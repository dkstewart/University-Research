\chapter{Compositional Minimal Cut Set Generation}
\label{chap:mcsGen}

Given the importance of minimal cut set computations for industrial sized systems, much research has been done on their generation (e.g.,~\cite{fta:survey,rauzy1993new,historyFTA,Bozzano:2010:DSA:1951720,rausand2003system}). As described in preliminary sections, the methods of cut set generation have varied greatly depending on how the system is modeled -- transition systems, state machines, decision diagrams, to name a few-- and what kind of model checking is performed -- symbolic algorithms, bounded model checking, etc. Compositional minimal cut set generation has not, to our knowledge, been previously explored. Due to the compositional nature of the verification it is difficult to see how faults that are {\em not} present in the current layer will affect the component behaviors and proofs of the current layer. This information needs to get passed through the layers of the model somehow. 

A contribution of this dissertation is providing a means for the compositional generation of minimal cut sets. In order to perform this kind of computation compositionally, a pre-existing framework was required. \danielle{Instead of having full paragraphs here for each piece, should I refer to the sections where these things are defined fully? Or provide a short preview here to eliminate the need to jump around in the document? Or both?}

(1) A rich modeling language was needed that allowed for the hierarchical definition of the model. The Architecture Analysis and Design Language (AADL)~\cite{aerospace2012sae} was chosen; AADL is an SAE International standard language that provides a unifying framework for describing the system architecture for performance-critical, embedded, real-time systems~\cite{AADL_Standard,FeilerModelBasedEngineering2012}. %Furthermore, AADL supports language extensions called {\em annexes} and there are open-source verification options that reason over AADL models.

(2) A compositional verification framework was required for the analysis of AADL models. The Assume-Guarantee Reasoning Environment (AGREE) is a tool for formal analysis of behaviors in AADL models~\cite{NFM2012:CoGaMiWhLaLu}.  It is implemented as an AADL annex and annotates AADL components with formal behavioral contracts. Each component's contracts can include assumptions and guarantees about the component's inputs and outputs respectively, as well as predicates describing how the state of the component evolves over time. AGREE translates an AADL model and the behavioral contracts into the dataflow programming language Lustre~\cite{Halbwachs91:IEEE} and then queries the model checker JKind~\cite{2017arXiv171201222G} to conduct the back-end analysis. The analysis can be performed compositionally or monolithically.

(3) A safety specific language was required that allows for the definitions of faults over component outputs and the model checker provides safety specific information and proofs about the fault model. In the early stages of this research, the Safety Annex for AADL was developed. The Safety Annex for AADL and its supporting extensions to the AADL tools provide the ability to reason about faults and faulty component behaviors in AADL models~\cite{Stewart17:IMBSA,stewart2020safety, nasaFinalReport}. In the Safety Annex approach, AGREE contracts are used to define the nominal behavior of system component and the nominal model is verified using JKind. The Safety Annex implementation weaves faults into the nominal model and analyzes the behavior of the system in the presence of faults. The tool supports behavioral specification of faults and their implicit propagation through behavioral relationships in the model and provides support to capture binding relationships between hardware and software components of the system. %For more information on the Safety Annex, see Chapter~\ref{chap:faultModeling}.

The remainder of this chapter describes compositional minimal cut set generation. First, the formal background and definitions required to understand the approach are supplied, then the proofs and algorithms are given. The implementation of these algorithms in the Safety Annex are then described.

\section{The High Level Idea and Approach}
Recently, Ghassabani et al. developed an algorithm that traces a safety property to a minimal set of model elements necessary for proof; this is called the \textit{all minimal inductive validity core} algorithm (\aivcalg)~\cite{GhassabaniGW16,Ghassabani2017EfficientGO,bendik2018online}. Inductive validity cores produce the minimal set of model elements necessary to prove a property. Each set contains the \emph{behavioral contracts} -- the requirement specifications for components -- of the model used in a proof. When the \aivcalg algorithm is run, this gives the minimal set of contracts required for proof of a safety property. If all of these sets are obtained, we have insight into every proof path for the property. Thus, if we violate at least one contract from every MIVC set, we have in essence ``broken" every proof path. This is the information that is used to perform fault analysis using MIVCs.

Safety analysts are often concerned with faults in the system, i.e., when components or subsystems deviate from nominal behavior, and the propagation of errors through the system. To this end, the model elements included in the reasoning process of the \aivcalg algorithm are not only the contracts of the system, but faults as well. This will provide additional insight into how an active fault may violate contracts that directly support the proof of a safety property. 

Before the specifics of the algorithm and proofs can be discussed, some background definitions are required. Throughout this chapter, a running example is referenced and we provide the description here.

\input{runningexampleMCS}

\input{backgroundMinCutSet}

\section{Formalization of the Method}
\danielle{Section desperately needs figures of some kind to break up the text. Will try to make a few small ones of PWR example results.}
Compositional analysis proceeds from the top layer of the architecture down through the system model; faults are defined on leaf level components and guarantees are defined on all components. Due to the difference in analysis per layer, this section focuses on the formalism per layer type we are in. 

Given an initial state $I$ and a transition relation $T$ consisting of conjunctive constraints as defined in section~\ref{sec:prelim}. The nominal guarantees of the system, $G$, consist of conjunctive constraints $g \in G$. Given no faults, each $g$ is one of the transition constraints $T_i$ where:

\begin{gather}
T_n = g_1 \land  g_2 \land \cdots \land g_n
\label{eq:Tn}
\end{gather}

We assume the property holds of the nominal relation $(I,T_n) \vdash P$. Given that our focus is on safety analysis in the presence of faults, let the set of all faults in the system be  denoted as $F$. A fault $f \in F$ is a deviation from the normal constraint imposed by a guarantee. Any ``faults" in a mid-layer are simply violated guarantees, or deviations from normal behavior.

\subsection{Top Layer of Compositional Analysis}
Since faults are defined at leaf layers of the architecture, the top (and middle) layers only contain guarantees in the analysis. The \aivcalg algorithm collects all {\em minimal unsatisfiable subsets} (MUSs) of a given transition system in terms of the \textit{negation} of the top level property~\cite{Ghassabani2017EfficientGO,bendik2018online}. Formally, an MUS of a constraint system $C$ is a set $M \subseteq C$ such that $M$ is unsatisfiable and $\forall c \in M$ : $M \setminus \{c\}$ is satisfiable. The MUSs are the minimal explanation of the infeasibility of this constraint system; equivalently, these are the minimal sets of model elements necessary for proof of the safety property.

Returning to our running example, this can be illustrated by the following. Given the constraint system $C = \{G_p, G_t, G_r, \neg P\}$, a minimal explanation of the infeasability of this system is the set $\{G_p, G_t, G_r,\}$. If all three guarantees hold, then $P$ is provable. 

A related set is a {\em minimal correction set} (MCS); a MCS $M$ of a constraint system $C$ is a subset $M\subseteq C$ such that $C \setminus M$ is satisfiable and $\forall S \subset M$ : $C \setminus S$ is unsatisfiable. A MCS can be seen to ``correct'' the infeasability of the constraint system by the removal from $C$ the constraints found in an MCS.

In the case of an UNSAT system, we may ask: what will correct this unsatisfiability? Returning to the PWR example, we can find the MCSs of the top level constraint system: $MCS_1 = \{G_t\}$, $MCS_2 = \{G_p\}$, $MCS_3 = \{G_r\}$. If any single guarantee is violated, a shut down from that subsystem will not get sent when it should and the safety property $P$ will be violated. 

A duality exists between the MUSs of a constraint system and the MCSs as established by Reiter \cite{reiter1987theory}. This duality is defined in terms of \textit{Minimal Hitting Sets} (\textit{MHS}). A hitting set of a collection of sets $A$ is a set $H$ such that every set in $A$ is ``hit'' by $H$; $H$ contains at least one element from every set in $A$. Every MUS of a constraint system is a minimal hitting set of the system's MCSs, and likewise every MCS is a minimal hitting set of the system's MUSs~\cite{liffiton2016fast, reiter1987theory, de1987diagnosing}.

For the PWR top level constraint system, it can be seen that each of the MCSs intersected with the MUS is nonempty. And now we have the minimal set of guarantees for which, if violated, will cause $P$ to be unprovable. 

\subsection{Leaf Layer of Compositional Analysis}
The faults in the safety annex are defined on leaf level components. Thus, for the lowest analysis layer, we must take into consideration faults and the guarantees their activation may violate. A fault $f \in F$ is a deviation from the normal constraint imposed by a guarantee. For the purposes of this paper, each guarantee at the leaf layer of analysis has an associated fault. Without loss of generality, we associate a single fault and an associated fault probability with a guarantee. Each fault $f_i$ is associated with an \emph{activation literal}, $af_i$, that determines whether the fault is active or inactive. 

To consider the system under the presence of faults, consider a set $GF$ of modified guarantees in the presence of faults and let a mapping be defined from activation literals $af_i \in AF$ to these modified guarantees $gf_i \in GF$. 
\begin{center}
$\sigma : AF \rightarrow GF$ \\
$gf_i = \sigma(af_i) =$ if $af_i$ then $f_i$ else $g_i$
\label{eq:sigma}
\end{center}

The transition system is composed of the set of modified guarantees $GF$ and a set of conjunctions assigning each of the activation literals $af_i \in AF$ to false: 

\begin{gather}
T = gf_1 \land gf_2 \land \cdots \land gf_n \land \neg af_1 \land \neg af_2 \land \cdots \land \neg af_n
\label{eq:T}
\end{gather}

\begin{lemma} If $(I,T_n) \vdash P$ for $T_n$ defined in equation~\ref{eq:Tn}, then $(I,T) \vdash P$ for $T$ defined in equation~\ref{eq:T}.
\begin{proof}
By application of successive evaluations of $\sigma$ on each constrained activation literal $\neg af_i$, the result is immediate.
\end{proof}
\end{lemma}

Consider the elements of $T$ as a set $GF \cup AF$, where $GF$ are the potentially faulty guarantees and $AF$ consists of the activation literals that determine whether a guarantee is faulty. This is a set that is considered by a SAT-solver for satisfiability during the $k$-induction procedures. The posited problem is thus: $GF \land AF \land \neg P$ for the safety property in question. Recall, if this is an \emph{unsatisfiable} constraint system, then $(I,T) \vdash P$. On the other hand, if it is \emph{satisfiable}, then we know that given the constraints in $GF$ and $AF$, $P$ is not provable. These are the exact constraints we wish to find. 

Let us view this in terms of the PWR system example and focus on the temperature sensor subsystem. The safety property to be proved is $G_t$, the supporting guarantees are found in each of the three temperature sensors, $g_{ti}$. Faults $f_{ti}$ are defined for each sensor. The transition system is: 
\begin{gather*}
T = gf_{t1} \land gf_{t2} \land gf_{t3}  \land \neg af_{t1} \land \neg af_{t2} \land \neg af_{t3}
\end{gather*}

The MIVCs for this subsystem layer correspond to all pairwise combinations of constrained activation literals. Intuitively, if any two sensor faults do {\em not} occur, then two of the three sensor guarantees are not violated and the system responds appropriately to high temperature; therefore, $G_t$ is provable. 

The MCSs for this subsystem layer happen to also correspond to all pairwise combinations of constrained activation literals. If any two sensor faults {\em do} occur, then two of the three sensor guarantees will be violated and the system does not respond to high temperature as required. This would result in the inability to prove $G_t$. (Note: it is not always the case that the MCSs are the same as the MIVCs -- in this case it is due to majority voting on three sensors.)

\subsection{Transforming MCS into Minimal Cut Set}
The MCSs contain the information needed to find minimal cut sets, but their elements consist of constrained activation literals and/or guarantees. The link between the activation literals, faults, and guarantees is defined through $\sigma$ mapping (equation~\ref{eq:sigma}). At the leaf layer, only activation literals are found in MCSs and $\sigma$ must be applied to each element in an MCS to map back to the associated fault. Without loss of generality, let $MCS = \{af_1, \cdots, af_m\}$. Let $\sigma (MCS) = \{\sigma (\neg af_{1}), \cdots, \sigma (af_{m})\}$ be a mapping where MCS is a minimal correction set with regard to some property $G$ and $MCS  \subseteq AF$. \danielle{Question: Does minimality need its own proof?}

\begin{lemma} $\sigma (MCS)$ is a minimal cut set of $G$. 
\begin{proof}
Assume towards contradiction that $\sigma (MCS)$ is not a cut set of $G$. Then $gf_1 \land \cdots \land gf_n \land af_1 \cdots \land af_m \land \neg af_{k+1} \land \neg af_n \land \neg G$ is unsatisfiable. Thus, the \emph{true} activation literals do not affect the provability of $G$. This contradicts $C \setminus MCS$ is satisfiable. 
Minimality follows directly from the definition of MCS.
\end{proof}
\end{lemma}

In terms of the PWR example, the minimal cut sets for the temperature subsystem property $G_t$ consist of all pairwise faults on the temperature sensors; if any two faults occur on the sensors at the same time, we violate the temperature subsystem guarantee. 

Once these lower level minimal cut sets are generated, it is a matter of simple set replacement to find the higher level minimal cut sets. This can be easily seen in our example. An MCS at the top level has the element $G_t$. We systematically replace the contract with the faults that cause their violation. This results in three distinct minimal cut sets for $P$ from the temperature subsystem: $\{f_{t1}, f_{t2}\}, \{f_{t1}, f_{t3}, \{f_{t2}, f_{t3}$. All minimal cut sets for $P$ are given as similar pairwise combinations from each subsystem and total 9 for the entire system.

\danielle{Seems I need a theorem to round it out: that replacement will give min cut sets of safety property. Will think about how to formulate this.}




\section{Implementation and Algorithms}
\label{sec:algs}
The implementation of this idea requires changing what information the \aivcalg algorithm uses to complete the proofs and generate MUSs. At each layer of analysis, the \aivcalg algorithm views the model as a constraint system consisting of the negation of the property in question (guarantees at lower levels, top-level properties at the highest level), and the supporting guarantees/assumptions from the direct child level. The information provided to this algorithm changes slightly when performing the minimal cut set algorithms.

\subsubsection{Implementation in Lustre}
In this approach, we use the all MIVCs algorithm and provide it a constraint system consisting of the negation of the top level safety property, the contracts of system components, as well as the faults in each layer constrained to false. It then collects the MUSs of this constraint system.

\begin{figure}[htbp]
	\hspace*{-2cm}
	\vspace{-0.1in} 
	\begin{center}
		\includegraphics[scale=0.5]{images/ivcElements1.png}
	\caption{IVC Elements used for Consideration in a Leaf Layer of a System}
		\label{fig:ivcElements1}
	\end{center}
\end{figure}

Different layers of the architecture (and hence proof) provide slightly different information to the \aivcalg algorithm. This is ``given" to the IVC algorithm by the insertion of a Lustre statement with the keyword \texttt{\%IVC} followed by the fault activation literal. \\ \texttt{--\%IVC \_\_fault\_\_independently\_\_active\_\_sensor}\\

The constraints on that literal are given through the use of an assert statement in Lustre.\\ \texttt{assert (\_\_fault\_\_independently\_\_active\_\_sensor = false)}\\

The leaf nodes contribute only constrained faults to the IVC elements as shown in Figure~\ref{fig:ivcElements1}. In the non-leaf layers of the program, both contracts and constrained faults are considered as shown in Figure~\ref{fig:ivcElements2}. The reason for this is that the contracts are used to prove the properties at the next highest level and are necessary for the verification of the properties. The faults are used to provide safety pertinant information for the minimal cut sets. 

\begin{figure}[htbp]
	\hspace*{-2cm}
	\vspace{-0.1in} 
	\begin{center}
		\includegraphics[scale=0.5]{images/ivcElements2.png}
	\caption{IVC Elements used for Consideration in a Middle Layer of a System}
		\label{fig:ivcElements2}
	\end{center}
\end{figure}

The all MIVCs algorithm returns the minimal set of these elements necessary to prove the properties. This equates to any contracts or inactive faults that must be present in order for the verification of properties in the model. From here, we perform a number of algorithms to transform all MIVCs into minimal cut sets.


%%%%%%%%%%%%%%%%%%%%%%%%%%%%%%%%%%%%%%%%%%%%%%%%%
%%%%%%%%%%%%%%%%%%%%            ALGORITHM DETAILS
\subsubsection{Algorithms}
The generation of \textit{MIVCs} traverses the program in a top down fashion. Likewise, the transformation of \textit{MIVCs} to MinCutSets traverses this program layer by layer if and only if all MIVCs have been generated. It is a requirement of the minimal hitting set algorithm that \textit{all} MUSs are used to find the MCSs~\cite{liffiton2016fast,gainer2017minimal,murakami2013efficient}. Thus, once all the MIVCs have been found and the minimal hitting set algorithm has completed, %our 
the MinCutSet Generation algorithm can begin. 

The MinCutSet Generation Algorithm begins with a list of MCSs specific to a top level property. These MCSs may contain a mixture of fault activation literals constrained to \textit{false} and %\textit{true} subcomponent contracts.
and subcomponent contracts constrained to \textit{true}. We remove all constraints from each MCS and call the resulting sets $I$, for \textit{Intermediate} set. Replacement of subcomponent contracts with their respective minimal cut sets can then proceed. For each of those contracts in $I$, we check to see if we have previously obtained a MinCutSet for that contract. If so, replacement is performed. If not, we recursively call this algorithm to obtain the list of all MinCutSets associated with this subcomponent contract. At a certain point, there will be no more contracts in the set $I$ in which case we have a minimal cut set for the current property. When this set is obtained, we store it in a lookup table keyed by the given property that this $I$ is associated with. 

A small example will illustrate this process. \danielle{Do I want to have a more concrete example? If so, pull from thesis proposal sensor example. If a shorter more abstract example is okay, do that.}







Algorithm~\ref{alg:generation_alg} describes this process.


\begin{algorithm}[htbp]
\SetKwFunction{FMain}{replace}
 \SetKwProg{Fn}{Function}{:}{}

	\Fn{\FMain{$P$}}{
		$List(I)$:= $List(MCS)$ for $P$ with all constraints removed \;
		\For{all $I \in List(I)$}{
			\eIf{there exists contracts $g \in I$}{
				\For{all constrained contracts $g \in I$}{
					\eIf{there exists $MinCutSets$ for $g$ in lookup table}{
						\For{all $minCut(g)$}{
							$I_{repl} = I$ \;
							$I_{repl} :=$ replace $g$ with $minCut(g)$ \;
							add $I_{repl}$ to $List(I)$ \;
						} %end for all cut sets of g
					}{
						replace($g$) \;
					} % end else if no cut sets in lookup table
				} % end for all constrained contracts in I
			}{
				add $I$ as $minCut(g)$ for $P$ \;
			} %end else if there exists contracts in I
		}%end for all I in list(I)
	}
%	\caption{Minimal Cut Set Generation Algorithm}
	\caption{MinCutSets Generation Algorithm}
	\label{alg:generation_alg}
\end{algorithm}

The number of replacements $R$ that are made in this algorithm are constrained by the number of minimal cut sets there are 
for all $\alpha$ contracts within the initial MCS. 

We call the set of all minimal cut sets for a contract $g$: $Cut(g)$. The following formula defines an upper bound on the number of replacements. The validity of this statement follows directly from the general multiplicative combinatorial principle. The number of replacements $R$ is bounded by the following formula:
\begin{equation}
\label{eq:bound}
  R \leq {\displaystyle \sum_{i=1}^{\alpha} }({\displaystyle \prod_{j=1}^{i} |Cut(g_j)|})  
\end{equation}


It is also important to note that the cardinality of $List(I)$ is bounded, i.e. the algorithm terminates. Every new $I$ that is generated through some replacement of a contract with its minimal cut set is added to $List(I)$ in order to continue the replacement process for all contracts in $I$. Adding to this set requires proof regarding termination.
\begin{theorem}
Algorithm~\ref{alg:generation_alg} terminates
\begin{proof}
No infinite sets are generated by the \aivcalg or minimal hitting set algorithms~\cite{Ghassabani2017EfficientGO,murakami2013efficient}; therefore, every MCS produced is finite. Thus, every $MinCutSet$ of every contract $g$ is finite. Furthermore, a bound exists on the number of additional intermediate sets $I$ that are added to $List(I)$: \\
$|List(I)| \leq R$ (Equation~\ref{eq:bound}).
\end{proof}
\end{theorem}

The reason for this upper bound is that for a contract $g_1$ in MCS, we make $|Cut(g_1)|$ replacements and add the resulting lists to $List(I)$. Then we move to the next contract $g_2$ in $I$. We must additionally make $|Cut(g_1)| \times |Cut(g_2)|$ replacements and add all of these resulting lists to $List(I)$, and so on throughout all contracts. Through the use of basic combinatorial principles, we end with the above formula for the upper bound on the number of additional intermediate sets.


\subsubsection{Pruning to Address Scalability}
The MinCutSets are filtered during this process based on a fault hypothesis given before analysis begins. The Safety Annex provides the capability to specify a type of verification in what is called a \textit{fault hypothesis statement}. These come in two forms: maximum number of faults or probabilistic analysis. Algorithm~\ref{alg:generation_alg} is the general approach, but the implementation changes slightly depending on which form of analysis is being performed. This pruning improves performance and diminishes the problem of combinatorial explosions in the size of minimal cut sets for larger models. \\

\textbf{Max $N$ Analysis Pruning} This statement restricts the number of faults that can be independently active simultaneously and verification is run with this restriction present. For example, if a max 2 fault hypothesis is specified, two or fewer faults may be active at once. In terms of minimal cut sets, this statement restricts the cardinality of minimal cut sets generated.

If the number of faults in an intermediate set $I$ exceeds the threshold $N$, any further replacement of remaining contracts in that intermediate set can never decrease the total number of faults in $I$; therefore, this intermediate set is eliminated from consideration.\\

\textbf{Probabilistic Analysis Pruning} The second type of hypothesis statement restricts the cut sets by use of a probabilistic threshold. Any cut sets with combined probability higher than the given probabilistic threshold are removed from consideration. The allowable combinations of faults are calculated before the transformation algorithm begins; this allows for a pruning of intermediate sets during the transformation. If the faults within an intermediate set are not a subset of any allowable combination, that intermediate set is pruned from consideration and no further replacements are made. 


