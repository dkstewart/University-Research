Model-based development tools are increasingly being used for system-level development of safety-critical systems. Architectural and behavioral models  provide important information that can be leveraged to improve the system safety analysis process. Model-based design artifacts produced in early stage development activities can be used to perform system safety analysis, reducing costs and providing accurate results throughout the system life-cycle.

Safety analysis is used to ensure that critical systems operate within some level of safety when failures are present. As critical systems become more dependent on software components, it becomes more challenging for safety analysts to comprehensively enumerate all possible failure causation paths. Any automated analyses should be sound to sufficiently prove that the system operates within the designated level of safety. This paper presents a compositional approach to the generation of fault forests (sets of fault trees) and minimal cut sets. We use a behavioral fault model to explore how errors may lead to a failure condition. The analysis is performed per layer of the architecture and the results are automatically composed. A complete formalization is given. We implement this by leveraging minimal inductive validity cores produced by an infinite state model checker. This research provides a sound alternative to a monolithic framework. This enables safety analysts to get a comprehensive enumeration of all applicable fault combinations using a compositional approach to generate while generating artifacts required for certification.


%The result is an extension to the Architecture Analysis and Design Language (AADL) that supports modeling of system behavior under failure conditions. This \emph{Safety Annex} enables the independent modeling of component failures and allows safety engineers to weave various types of fault behavior into the nominal system model. The accompanying tool support uses model checking to propagate errors from their source to their effect on safety properties without the need to add separate propagation specifications. 



