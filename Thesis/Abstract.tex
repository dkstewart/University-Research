Model-based development tools are increasingly being used for system-level development of safety-critical systems. Architectural and behavioral models  provide important information that can be leveraged to improve the system safety analysis process. Model-based design artifacts produced in early stage development activities can be used to perform system safety analysis, reducing costs and providing accurate results throughout the system life-cycle.

As critical systems become more dependent on software components, analysis regarding fault propagation through these software components becomes more important. The methods used to perform these analyses require understandability from the side of the analyst, scalability in terms of system size, and mathematical correctness in order to provide sufficient proof that a system is safe. Determination of the events that can cause failures to propagate through a system as well as the effects of these propagations can be a time consuming and error prone process. In this research, we introduce a safety analysis tool extension to the AADL modeling language, describe a technique for determining failure events with the use of Inductive Validity Cores (\ivc), and show how an analyst can use these methods to produce compositionally derived artifacts that encode pertinant system safety information.


%The result is an extension to the Architecture Analysis and Design Language (AADL) that supports modeling of system behavior under failure conditions. This \emph{Safety Annex} enables the independent modeling of component failures and allows safety engineers to weave various types of fault behavior into the nominal system model. The accompanying tool support uses model checking to propagate errors from their source to their effect on safety properties without the need to add separate propagation specifications. 



