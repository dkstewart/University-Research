\chapter{Introduction}
\label{chap:intro}
In our ever increasingly computerized world, the concept of system safety has become of great importance to many different fields of study. A \emph{complex safety critical system} is one whose safety cannot be shown only through testing, whose logic is difficult to comprehend without the aid of analytical tools, and that may contribute - directly or indirectly - to loss of life, damage of the environment, or large economical losses~\cite{SAE}. Critical systems can be found in aviation, automotive, nuclear, or medical industries and the process of designing such systems, from birth to use in society, presents numerous problems that researchers have been contending with for many years. 

System safety has been an important factor in the design of systems for many years, but the birth of system safety as we know it today is associated with the problems that the US Air Force experienced with accidents after World War II. Over 7,700 aircraft were lost between the years of 1952 and 1966 and over 8,000 people were killed~\cite{hammer}. At the time, many of the accidents were blamed on pilots, but many flight engineers did not believe the causes were so simple. They posited that safety must be designed and built into the aircraft~\cite{levesonWhitePaper}. Between the growth of nuclear capabilities, the defense industry complex, and the overall increase of computer capabilities, the need to abandon a "fly-fix-fly" approach to safety was imminent. The goal became to avoid accidents before they occur. 

Now, system safety analysis is crucial in the development life cycle of critical systems to ensure adequate safety as well as demonstrate compliance with applicable standards. The process that guides the development and certification of safety critical systems is highly controlled and standardized by competent authorities~\cite{SAE,SAE:ARP4761,SAE:ARP4754A}.

A prerequisite for any safety analysis is a thorough understanding of the system architecture and the behavior of its components; safety engineers use this understanding to explore the system behavior to ensure safe operation, assess the effect of failures on the overall safety objectives, and construct the accompanying safety analysis artifacts~\cite{SAE:ARP4761,SAE:ARP4754A}. These artifacts are used to show that system requirements are built into the system and correctly implemented. These artifacts are also used to show specific kinds of failures that may occur during normal use of the system and both qualitative and quantitative analyses can show that the system is safe for use.

The development life cycle of critical systems can be roughly seen as two main thrusts: one side focuses on the system development itself; the hardware design, the requirements of the system, and the overall behavior of the components and how they work together. The other side is the safety assessment of the designed system. Safety analysts use the artifacts generated during the system design and development process and analyze the system from the perspective of faults; in other words, what can go wrong. 

Due to the nature of this arrangement, these sides are not always done in strict parallel and are rarely synchronized perfectly, for example, the artifacts given to safety analysts from system engineers are not always formal in nature, come from various sources, and are rarely complete on their own. To address this concern, \emph{model-based system development} caught the attention of researchers in the safety critical system domains~\cite{Joshi05:Dasc,CAV2015:BoCiGrMa,info17:HaLuHo,5979344,Gudemann:2010:FQQ:1909626.1909813}. In model-based development, the development efforts are centered on a model of the intended system behavior and various techniques, for example, formal verification, testing, test generation, execution and animation, etc. can be used to validate and verify the proposed system behavior. Given this increase in model-based development in critical systems, leveraging the resultant models in the safety analysis process and automating the generation of safety analysis artifacts, holds great promise in terms of accuracy and efficiency. Many of these artifacts are representation of the functionality of the system in the presence of faults and can be effectively used in the certification process. An example of a commonly used artifact in the safety assessment process is all \emph{minimal cut sets}, or the minimal sets of faults that violate a system safety property. The automatic generation of these artifacts have been studied in depth, but have lacked in terms of scalability. Some research groups have introduced automating aspects of the safety assessment process and have developed tools to support this~\cite{Joshi05:SafeComp,CAV2015:BoCiGrMa,10.1007/978-3-319-11936-6-7}; nevertheless, there are gaps in current capabilities we aim to address. 

Many of the techniques proposed for \emph{model-based safety analysis} (MBSA) require the development of {\em fault models} specific for the safety analysis task; that is, the techniques do not rely on extension of existing system models, but instead require new models that are separate entities~\cite{symbAltaRica, DBLP:conf/tacas/BittnerBCCGGMMZ16, info8010007, Gudemann:2010:FQQ:1909626.1909813}. Thus there is a system model used by the system engineers; this model is used to create a new fault model by the safety analysts. This requires extra manual labor and clear understanding of the system in order to create a separate fault model that accurately describes the system in question. As systems become more complex, it becomes difficult to ensure that the fault model developed for safety analysis conforms with the the model created for the development efforts; just as it is difficult to show that the system model conforms to the actual implemented system. 

Part of the safety assessment process determines how faults can manifest themselves in a particular component, but also how a manifested fault (or \emph{error}) can propagate through a system. The propagation of faults can be handled a variety of ways; most commonly this is done through the use of signal flow diagrams, a deep understanding of the system components, and the intuition of a good analyst. Various research has attempted to address this gap by providing tools that work over a model and provide some form of propagation analysis. Often this propagation is done explicitly, but as the size and complexity of industrial sized systems grow, this form of propagation becomes quite unwieldy. To address this problem, \emph{behavioral} propagation has been introduced~\cite{DBLP:conf/tacas/BittnerBCCGGMMZ16,stewart2020safety}. Behavioral propagation automates the process of "moving" the error through the system and requires no explicit statements of what affect the error will have on components. In reality, both approaches are beneficial to an analyst. At times, there are effects that are known and easily captured explicitly. Other times, even within the same system, complex interactions make explicit propagation difficult to manage. The approach in this research uses both approaches in order to provide the most flexibility for an analyst. 

Another sticking point of MBSA is scalability of the \emph{verification} of the model and its requirements. Verification is the process of mathematically proving or disproving the correctness of a system with respect to certain properties or requirements. As a model and the number of system requirements grows large, a scalable approach is of utmost concern. \danielle{Cut mention of compositional here? Not needed... A \emph{compositional} approach to verification has been shown to be a more scalable approach to this problem. Compositional verification analyses the system components per layer of the architecture and combines the result. This breaks a large proof problem into a number of smaller proofs; a more manageable task for a model checker.} 

Leveraging valuable proof information about the system to use in the safety assessment process is of interest in this disseration. The scalable generation of the combinations of faults that can cause system failure is something that would be of great benefit to the safety analysis community and is currently a major bottleneck in using automated approaches in industry. Given recent research in the area of formal verification, this has become theoretically possible and is outlined extensively in this dissertation.

%Probabilistic evaluations of the faults and their overall impact on the system is a large part of the safety assessment process. Utilizing a compositional approach for the generation of minimal cut sets shows promise in compositional probabilistic computations as well. %Currently, probabilistic computations are not performed compositionally and 

In this dissertation, we outline a novel way to utilize the verification information used during the system development process in order to automate the generation of both qualitative and quantitative safety analysis artifacts. Furthermore, we describe a Model Based Safety Analysis (MBSA) approach to critical system development in which the safety analysis is tied directly to the system model and the flexibility of the analysis provides various ways of capturing error propagation information, single points of failures, and minimal cut sets.


\section{Objective and Summary of Contributions}
The \textbf{long range goal} is to increase system safety through the support of a model-based safety assessment process backed by formal methods to help safety engineers with early detection of design issues and automation of the artifacts required for certification. The \textbf{objective of this dissertation}, which is a logical step towards the goal, is to formally prove a relationship between formal verification results and safety analysis artifacts to allow for scalable generation of minimal cut sets, define a modeling process that allows for flexible fault modeling in the MBSA approach, and finally the support of these goals with relevant modeling environments.

The objectives of this dissertation were accomplished by pursuing the following aims: 

\begin{description}
\item[Defined and implemented a Safety Annex to enable fault modeling in AADL.] The modeling needs of a safety analyst were researched and a grammar that extends the Architecture Analysis and Design Language (AADL) was defined. A safety analysis platform was implemented that uses this grammar extension along with a backend SMT model checker for compositional verification purposes.
%\item \textbf{Define a grammar for a safety analysis tool that will work closely with a standard modeling language that has existing support for formal verification.} The grammar will formally define a user-friendly syntax that will still allow for rich fault models. The analyst will be able to wrap a fault around the output of a component such that the fault model has access to certain elements of the system components while still providing separation between the fault model and the nominal system model. The grammar should extend a standard system modeling language to provide ease in the integration of these languages and supporting tool framework. 
\item[Defined compositional verification capabilities in the face of component failures.] The analysis procedures were defined that allow an analyst to investigate system behaviour under a predetermined max number of faults or a max probability threshold for fault combinations.%, and develop and implement analysis procedures for these analyses. 
%\item \textbf{Determine how to capture behavioral and explicit error propagation through the use of formal verification techniques.} Given a complex model with numerous interactions, it becomes difficult to realize all possible effects of an active fault in the system; therefore, it is necessary to be able to provide behavioral error propagation within the system model. That being said, at times it is also helpful to explicitly state how an error propagates, for example when components are physically co-located but logically modeled separately. Thus having the capability to represent and model both types of propagation is valuable to an analyst. Research will need to be performed to determine how the effects of an active fault can be propagated through the contracts of the system's subcomponents; additionally, it is necessary to see exactly how this propagation will effect the system level properties. 


\item[Defined compositional computation of minimal cut sets.] Formally proved the relationship between verification results of compositional analysis and minimal cut sets, and implemented the algorithms for minimal cut set generation in the analysis tools supporting the Safety Annex.

\item[Researched how development of requirements could change analysis results.] Splitting a complex requirement into its constituent pieces can change how the model checker derives the artifacts used in the analysis.

\item[Illustrated the ideas using a case study] An industrial sized case study from the safety critical aerospace domain was used to illustrate all of the above.  
\end{description}

The beginning phase of this dissertation research included research into the needs of safety analysts and resulted in the definition of the Safety Annex~\cite{stewart2020safety, Stewart17:IMBSA}, the implementation of the Safety Annex parser in the OSATE tools~\cite{SEI:AADL}, and the support for analysis of AADL models extended with the Safety Annex~\cite{cofer2012compositional}. The final steps of this research included the compositional generation of minimal cut sets and a full implementation of these algorithms in the Safety Annex. 

\section{Structure of this Document}
This dissertation is organized into 7 chapters. The preliminaries in Chapter~\ref{chap:prelim} discusses the background and justification for this research, critical system development and the state of the field, and ends with an overview of formal verification. Chapter \label{chap:faultModeling} provides a detailed look at the Safety Annex and its implementation. Chapter~\ref{chap:mcsGen} describes the compositional generation of minimal cut sets. Chapter \label{chap:granularity} provides the initial research into how a particular form of contract definition can change the results of the analysis; it is followed by a chapter on case studies. Lastly, the conclusion summarizes the research approach in this dissertation.











