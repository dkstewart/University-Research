\chapter{Introduction}
\label{chap:intro}

System safety analysis is crucial in the development life cycle of critical systems to ensure adequate safety as well as demonstrate compliance with applicable standards. A prerequisite for any safety analysis is a thorough understanding of the system architecture and the behavior of its components; safety engineers use this understanding to explore the system behavior to ensure safe operation, assess the effect of failures on the overall safety objectives, and construct the accompanying safety analysis artifacts~\cite{SAE:ARP4761,SAE:ARP4754A}. Developing adequate understanding, especially for software components, is a difficult and time consuming endeavor which has traditionally been performed manually. Model-based development is getting increased adoption in the critical systems domain~\cite{Joshi05:Dasc,CAV2015:BoCiGrMa,info17:HaLuHo,5979344,Gudemann:2010:FQQ:1909626.1909813}. In model-based development, the development efforts are centered on a model of the intended system behavior and various techniques, for example, formal verification, testing, test generation, execution and animation, etc. can be used to validate and verify the proposed system behavior. Given this increase in model-based development in critical systems, leveraging the resultant models in the safety analysis process and automating the generation of safety analysis artifacts, holds great promise in terms of accuracy and efficiency. Many of these artifacts are representation of the functionality of the system in the presence of faults and can be effectively used to find \textit{minimal cut sets}, the minimal sets of faults that lead to a violation of a safety property. Some research groups have introduced automating the safety assessment process and have developed tools to support the automated generation of these artifacts~\cite{Joshi05:SafeComp,CAV2015:BoCiGrMa,10.1007/978-3-319-11936-6-7}. Nevertheless, there are gaps in current capabilities we aim to address. 

Many of the techniques developed require the development of models specific for the safety analysis task, that is, the techniques do not rely on extension of existing system models, but instead require new models that are separate entities~\cite{symbAltaRica, DBLP:conf/tacas/BittnerBCCGGMMZ16, info8010007, Gudemann:2010:FQQ:1909626.1909813}. This requires extra manual labor and clear understanding of the system in order to create a separate fault model that accurately describes the system in question. As systems become more complex, it becomes difficult to ensure that the fault model developed for safety analysis conforms with the the model created for the development efforts. 

The propagation of faults throughout the system can be handled in two main ways: one is to explicitly define the propagations and the other is to use a behavioral approach to the propagations. Given many possible faults in an industrial sized system, explicit propagation becomes complex and unwieldy. That being said, it is sometimes beneficial to also explicitly define fault propagations between components that are not logically connected, e.g., co-located components may fail together even though they are not logically connected, threads running on a processor will fail together if the processor fails, etc. Thus, it is beneficial to provide both options to an analyst. 

Using a compositional approach---analyze the system components individually and combine the results---to fault modeling improves scalability compared to a monolithic approach---the components are all analyzed together. The compositional generation of the combinations of faults that can cause system failure is something that no other research has fully addressed and a scalable automatic generation of these artifacts is a major bottleneck in using automated approaches in industry. Given recent research in the area of formal verification, this has become theoretically possible.

Probabilistic evaluations of the faults and their overall impact on the system is a large part of the safety assessment process. Utilizing a compositional approach for the generation of minimal cut sets shows promise in compositional probabilistic computations as well. %Currently, probabilistic computations are not performed compositionally and 

In this proposal, we outline a novel way to utilize the verification information used during the system development process in order to automate the generation of both qualitative and quantitative safety analysis artifacts. Furthermore, we describe a Model Based Safety Analysis (MBSA) approach to critical system development in which the safety analysis is tied directly to the system model and the flexibility of the analysis provides various ways of capturing error propagation information, single points of failures, and minimal cut sets.

Our \textbf{long range goal} is to increase system safety through the support of a model-based safety assessment process backed by formal methods to help safety engineers with early detection of design issues and automation of the artifacts required for certification. The \textbf{objective of this proposal}, which is a logical step towards our goal, is to formally prove a relationship between formal verification results and safety analysis artifacts to allow for compositional generation of minimal cut sets and associated probabilities, define a modeling process that allows for flexible fault modeling in the MBSA approach, and finally to support these goals with relevant tools.
%
\iffalse
%There are two pieces of this research work. 
The main focus of this research work is to define algorithms and formulate theoretical proofs that make use of verification results in order to extract information about the fault model. The secondary focus is the development of a safety analysis tool that works closely with existing verification engines and implements the theories in the first step.
\fi
%
We plan to accomplish the objectives of this proposal by pursuing the following aims:

\begin{description}
\item[Define and implement a Safety Annex to enable fault modeling in AADL.] Research the modeling needs of a safety analyst and define a grammar that extends the Architecture Analysis and Design Language (AADL). Implement a tool that uses this grammar extension and a backend SMT model checker for compositional verification purposes.
%\item \textbf{Define a grammar for a safety analysis tool that will work closely with a standard modeling language that has existing support for formal verification.} The grammar will formally define a user-friendly syntax that will still allow for rich fault models. The analyst will be able to wrap a fault around the output of a component such that the fault model has access to certain elements of the system components while still providing separation between the fault model and the nominal system model. The grammar should extend a standard system modeling language to provide ease in the integration of these languages and supporting tool framework. 


\item[Define compositional verification capabilities in the face of component failures.] Define analysis procedures to allow an analyst to investigate system behaviour under a predetermined max number of faults or a max probability threshold for fault combinations, and develop and implement analysis procedures for these analyses. 
%\item \textbf{Determine how to capture behavioral and explicit error propagation through the use of formal verification techniques.} Given a complex model with numerous interactions, it becomes difficult to realize all possible effects of an active fault in the system; therefore, it is necessary to be able to provide behavioral error propagation within the system model. That being said, at times it is also helpful to explicitly state how an error propagates, for example when components are physically co-located but logically modeled separately. Thus having the capability to represent and model both types of propagation is valuable to an analyst. Research will need to be performed to determine how the effects of an active fault can be propagated through the contracts of the system's subcomponents; additionally, it is necessary to see exactly how this propagation will effect the system level properties. 


\item[Define compositional computation of minimal cut sets.] Formally prove the relationship between verification results of compositional analysis and minimal cut sets, and implement the algorithms in the analysis tools supporting the Safety Annex.
%\item \textbf{Provide evidence through the use of representative system models how the fault analysis performed in tandem with formal verification methods can generate valuable information used in the safety assessment process.} Given that the tool will be closely related to the verification process, the model checker will be able to provide pertinant fault model information. This will allow feedback on system traces when faults are active, specific counterexamples showing the state of the system when a property is violated, and single points of failure of a system. These types of system traces will allow an analyst to see specifically and clearly which subcomponents are at risk, what the overall effect a subcomponent's behavior has on a system requirement, and will allow faster analysis than manual efforts can perform. 

\item[Determine accurate compositional probabilistic computations for the top-level events.] Define the computations, show that the computations are accurate, and implement the algorithms in the analysis tools supporting the Safety Annex.
%\item \textbf{Develop formal proofs that certain traceability information gathered through the nominal system model verification process can be leveraged to generate safety analysis artifacts.} Certain safety analysis artifacts require much more than a counterexample or an example of a single point of failure. It is not only useful, but often required for critical system certification, to show all combinations of failures that can contribute to violation of a safety property (\textit{minimal cut sets}). These sets of faults are often computationally expensive to find and no other safety analysis tool provides a way to perform this computation in a \textit{compositional} way. We will formally prove that a relationship exists between the proof cores of the nominal model and minimal cut sets, and then use these formal results to compositionally generate safety artifacts. 

\item[Evaluate the results using representative models in AADL.] Gather the safety analysis artifacts generated by these algorithms and discuss how they may be used within the safety assessment process.
%We will use elements available through the compositional proof process in order to capture fault model information and find all \textit{minimal cut sets} of a safety property (minimal combinations of faults that cause property violation). 
%\item \textbf{Research the possibility of finding the overall probability of the violation of a safety property, and generate upper and lower bounds of the system wide failure probability.} It is desirable to find the probability of system failure or property violation and these quantitative results are often required in the safety assessment process. We will discuss valuable approaches to computing the exact probability of property violation and possible difficulties in finding or using this number. Furthermore, we will explore methods and formal algorithms for leveraging the compositionally derived minimal cut sets in order to compute upper and lower bounds for the actual probability of system failure. 
\end{description}

The beginning phase of this dissertation research included research into the needs of safety analysts and resulted in the definition of the Safety Annex, the implementation of the Safety Annex parser in the OSATE tools, and the support for analysis of AADL models extended with the Safety Annex. The remaining steps of this research include completion of the theory behind the relationship between verification results and fault analysis artifacts and a full implementation of these algorithms in the tools supporting the Safety Annex. 

This proposal is organized in three chapters. The background in Chapter~\ref{chap:background} broadly discusses related work, the tools and modeling language used in this project, and some useful formal definitions. Chapter 3 describes the proposed approach and outlines the contributions of this dissertation. Lastly, the conclusion summarizes the approach of the project.


