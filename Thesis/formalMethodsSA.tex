\section{Formal Methods in Safety Analysis}
\label{sec:formalMethodsSA}
As the complexity of systems increase, the cost of development and validation consumes more time and resources than ever before; nevertheless, these processes are vital in safety critical systems when the loss of functionality of the system can result in loss of life. Authorities have put in place various thresholds for the likelihood of such events and it is the responsibility of the system developers to show that this is incredibly unlikely to occur~\cite{faaSA}. Utilizing the recent advancements in automated formal verification within the validation process has become essential to the certification of critical systems~\cite{deptOfDefense,standard1999,prasad2005survey}. This section provides a summary of the formal method techniques that are commonly used in the system development and safety assessment processes.

\subsection{Formal Validation and Verification}
Formal validation and verification is a proof-based methodology used to assess the correctness of requirements, system design, and implementation. In the past, this has been performed through manual means, but with the advancement in automated theorem proving and other formal methods, automated formal analyses not only guarantees a higher degree of confidence, but also reduces the time (and thus cost) of carrying out the proofs of correctness. Techniques used in formal validation and verification include automated theorem proving, model checking, and abstract interpretation. 

\subsubsection{Formal Specification}
Formal specification process translates the informal system requirements into a mathematical logic to determine if the system design is correct. This process guarantees an unambiguous description of the requirements which is not possible when using an informal natural language. This formal definition of system requirements includes the system design and its expected behavior as well as the assumptions on environment. A design or implementation can never be considered correct in isolation; it is only correct with respect to the specifications. The expected behavior, system design, and environmental assumptions change and are refined as the system goes through the various stages of development. 

\subsubsection{Formal Verification} 
Formal verification is the use of proof methods to show that given the environmental assumptions stated in the formal specification, the formal design of the system meets the requirements. The proofs are generated over an abstract mathematical model of the system, such as finite state machines, labeled transition systems, or timed automata. There are a couple of main approaches used to provide these proofs: deductive methods or an exhaustive exploration of the model known as model checking. 

Model checking was introduced in the early 1980's and consists of exploring the states and transitions of a model~\cite{clarke1981design,queille1982specification}. It takes as input an automaton based model of a system and a temporal logic property, then explores the entire state space of the system to determine if the model violates the property~\cite{fraser2009testing}. In recent years, model checking takes advantage of abstraction techniques specific to a domain to consider multiple states or transitions in a single operation; this lessens computation time considerably~\cite{d2008survey}. Nevertheless, the biggest limiting factor of model checking is scalability and many model checkers do not perform well given industrial sized models. 

Deductive methods of verification consists of generating proof obligations from the specifications of the system and using these obligations in a theorem prover setting. This includes theorem provers (e.g., Coq~\cite{coq}, Isabelle~\cite{isabelle}) and satisfiability modulo theories (e.g., SMTInterpol~\cite{smtInterpol}, Z3~\cite{z3}, Yices~\cite{yices}). 

%Interactive theorem provers such as Coq or Isabelle \danielle{cite} provide a human-machine collaborative effort to guide the verification. Automated theorem provers, on the other hand, produce proofs without human interaction. Many theorem provers are a hybrid approach. \danielle{check this and cite}

 


%In practice, when the proof methods show that the requirements are met, this allows the developers to move into the next refinement phase. On the other hand, when the system does not meet requirements, the specifications must be revisited and revised. 

