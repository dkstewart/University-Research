\chapter{Conclusion}
\label{chap:conclusion}
System safety analysis is crucial in the development of critical systems and the generation of accurate and useful results is invaluable to the assessment process. Having multiple ways to capture complex dependencies between faults and the behavior of the system in the presence of these faults is important throughout the development process. A model-based approach was proposed that allows for a tighter integration between system development and safety analysis. The existing system model is extended and safety specific definitions and information can be defined at all levels of the model architecture (e.g., software, hardware, system level, module level). This extended model, or fault model, can be analyzed using a model checker and various safety related artifacts can be generated. These include snapshots of the state of the system when a fault is active (automatically generated counterexamples), maximum active fault thresholds, and minimal cut sets. 

We also provided a formalization of the compositional generation of minimal cut sets through the use of inductive validity cores. This will open the door for future research work into more scalable options of minimal cut set generation through the use of SMT-solvers and possibly other verification engines. 

An introductory exploration into the concepts of granularity and mutation testing provides a framework for the application of other kinds of formal methods techniques on a fault model and what kinds of important information can be gleaned from such analyses. 

All of this contributes to the \textbf{long range goal of the research}: to increase system safety through the support of MBSA process backed by formal methods to help safety engineers with early detection of design issues and automation of the artifacts required for certification. 