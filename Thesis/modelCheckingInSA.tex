\section{Formal Methods in Safety Analysis: A Brief History and the State of the Practice}
\label{sec:modelCheckingInSA}
Safety analysis has traditionally been performed manually, but with the rise of model checking and the improvement of its capabilities, the world of safety analysis began to see its powerful benefits~\cite{hinchey2012industrial, liggesmeyer1998improving, coudert1993fault, Bozzano:2010:DSA:1951720,bozzano2003esacs}. There arose multiple ways of viewing the system and fault models, various ways of automating the capture of safety pertinent information, and a number of tools that addressed various issues that arose. In this section, we discuss the state of the practice and how formal methods has been applied in the domain of safety assessment research.

\subsection{Model Checking in Model Based Safety Analysis}
From the beginnings of model checking, there was a slow increase in its application to the domain of safety analysis, but a few research groups contributed immensely to this branch of study. Separately, these researchers began to contribute to safety analysis through the use of model checking starting in the '90's and are still contributing today (e.g., \cite{reese1997software,signoret1998altarica,chiappini1999formal,cimatti2000industrial}. 

One of the main methods was the abstraction of the system into a formal transition system; this provided a means of defining a precise mathematical model of the system and simplifying mathematical operations through the use of abstraction techniques on the transition system. This helped to shrink the entire state space into something more digestible by computational techniques~\cite{d2008survey}. 

In the early 2000's, model based safety assessment began to make an appearance in literature~\cite{Bozzano:2010:DSA:1951720,Joshi05:Dasc, Joshi05:SafeComp, Joshi07:Hase}. This applied model checking and model based system development to safety analysis at the same time.  In this approach, a safety analysis system model (SASM) is the central artifact in the safety analysis process, and traditional safety analysis artifacts, such as fault trees, are automatically generated by tools that analyze the SASM.

The contents and structure of the SASM differ significantly across different conceptions of MBSA.  We can draw distinctions between approaches along several different axes.  The first is whether they propagate faults explicitly through user-defined propagations, which we call {\em failure logic modeling} (FLM), or {\em explicit propagation}, or through existing behavioral modeling, which we call {\em failure effect modeling} (FEM), or {\em implicit propagation}.  The next is whether models and notations are {\em purpose-built} for safety analysis vs. those that extend {\em existing system models} (ESM).

For FEM approaches, there are several additional dimensions.  One dimension involves whether {\em causal} or {\em non-causal} models are allowed.  Non-causal models allow simultaneous (in time) bi-directional error propagations, which allow more natural expression of some failure types (e.g. reverse flow within segments of a pipe), but are more difficult to analyze.  A final dimension involves whether analysis is {\em compositional} across layers of hierarchically-composed systems or {\em monolithic}.  %Our approach is an extension of AADL (ESM), causal, compositional, mixed FLM/FEM approach.
\danielle{Make a figure here - that will help explain these distinctions.}

This literature overview is not a complete account of all safety analysis model checking tools available either in industry or research, but highlights some of the most influential safety assessment methods and tools currently available. 

\subsubsection{AltaRica}
AltaRica was one of the first model checking tools specifically aimed at safety analysis of critical systems. The first iteration of AltaRica (1.0) performed over a transition system of the model, used dataflow ({\em causal}) semantics, and could capture the hierarchy of a system~\cite{signoret1998altarica}. The key idea was that this transition system (more specifically {\em constraint automata}) could be compiled into Boolean formulae and transformed into a BDD~\cite{point1999altarica}. The literature for performing fault tree analysis over BDDs was rich with algorithms; this was how much of the safety analysis artifacts were generated. The dataflow dialect (AltaRica 1.0) has substantial tool support, including the commercial Cecilia OCAS tool from Dassault~\cite{bieber2004safety}. For this dialect, the safety assessment, fault tree generation, and functional verification can be performed with the aid of NuSMV model checking~\cite{symbAltaRica}.

The most recent language update (AltaRica 3.0) uses non-causal semantics~\cite{prosvirnova2013compilationfaulttrees,PROSVIRNOVA2013127}. Failure states are defined throughout the system and flow variables are updated through the use of assertions~\cite{Bieber04safetyassessment}.  AltaRica 3.0 has support for simulation and Markov model generation through the OpenAltaRica (www.openaltarica.fr) tool suite; it is a {\em FEM}-based, {\em purpose-built}, {\em monolithic} safety analysis language. 

AltaRica 3.0 provides automated fault tree generation by translating the model into a reachability graph and then further compiling it into Boolean formula in order to compute minimal cut sets~\cite{prosvirnova2015automated}. 

\subsubsection{FSAP, xSAP, and COMPASS}
The Formal Safety Analysis Platform (FSAP) was introduced in 2003~\cite{bozzano2003improving} and supported failure mode definitions, safety requirements in temporal logic formulae, automated fault tree construction, and counterexample traces. The platform used NuSMV, a BDD-based model checker~\cite{Cimatti2000}. The system model, written in NuSMV, and the fault model, developed graphically in FSAP, are together translated into a finite state machine and eventually into a BDD; fault tree analysis is performed using BDD algorithms implemented in NuSMV. 

By 2016, the researchers that developed FSAP (Foundation Bruno Kessler, FBK) released a similar tool called xSAP~\cite{DBLP:conf/tacas/BittnerBCCGGMMZ16}. xSAP extends FSAP in many ways: xSAP can handle infinite state machines, it is textual language rather than graphical, allows for richer fault modeling and definitions, and implements more than just BDD computations (e.g., SAT- and SMT-based routines). xSAP was integrated into the COMPASS toolsuite to take advantage of the algorithms it supports. More complex SAT-based algorithms were introduced to bypass the BDD method of minimal cut set generation, namely the ``anytime approximation" algorithms~\cite{CAV2015:BoCiGrMa, mattarei2016scalable}. These algorithms make clever use of bounded model checking algorithms to explore counterexamples provided to the query "the top level event never occurs." These explorations are done such that the cut sets generated are of increasing cardinality which allows for an approximation computation to be given even when the state space is too large to compute all minimal cut sets. These are implemented in xSAP~\cite{CAV2015:BoCiGrMa}.

COMPASS (Correctness, Modeling project and Performance of Aerospace Systems)~\cite{10.1007/978-3-642-04468-7_15} is a mixed {\em FLM/FEM}-based, {\em causal} {\em compositional} tool suite that uses the SLIM language, which is based on a subset of the Architecture Analysis and Design Language (AADL), for its input models~\cite{5185388, criticalembeddedsystems}. In SLIM, a nominal system model and the error model are developed separately and then transformed into an extended system model.  This extended model is automatically translated into input models for the NuSMV model checker~\cite{Cimatti2000, NuSMV}, MRMC (Markov Reward Model Checker)~\cite{Katoen:2005:MRM:1114692.1115230, MRMC}, and RAT (Requirements Analysis Tool)~\cite{RAT}. The safety analysis tool xSAP~\cite{DBLP:conf/tacas/BittnerBCCGGMMZ16} can be invoked in order to generate safety analysis artifacts such as fault trees and FMEA tables~\cite{compass30toolset}.  %COMPASS is an impressive tool suite, but some of the features that make AADL suitable for SW/HW architecture specification: event and event-data ports, threads, and processes, appear to be missing, which means that the SLIM language may not be suitable as a general system design notation (ESM).

\subsubsection{SmartIFlow}
SmartIFlow~\cite{info17:HaLuHo,honig2014new} is a {\em FEM}-based, {\em purpose-built}, {\em monolithic} {\em non-causal} safety analysis tool that describes components and their interactions using finite state machines and events. Verification is done through an explicit state model checker which returns sets of counterexamples for safety requirements in the presence of failures.  SmartIFlow allows {\em non-causal} models containing simultaneous (in time) bi-directional error propagations.  On the other hand, the tools do not yet appear to scale to industrial-sized problems, as mentioned by the authors: ``As current experience is based on models with limited size, there is still a long way to go to make this approach ready for application in an industrial context''~\cite{info17:HaLuHo}.

\subsubsection{SAML}
The Safety Analysis and Modeling Language (SAML)~\cite{Gudemann:2010:FQQ:1909626.1909813} is a {\em FEM}-based, {\em purpose-built}, {\em monolithic} {\em causal} safety analysis language that was developed in 2010.  System models constructed in SAML can be used used for both qualitative and quantitative analyses. It allows for the combination of discrete probability distributions and non-determinism. The SAML model can be automatically imported into several analysis tools like NuSMV~\cite{Cimatti2000}, PRISM (Probabilistic Symbolic Model Checker)~\cite{CAV2011:KwNoPa}, or the MRMC probabilistic model checker~\cite{Katoen:2005:MRM:1114692.1115230}. SAML itself does not provide the formal verification engines, but instead provides a platform to model the safety aspects of a system and then translate this into the input language for a formal verification engine~\cite{Gudemann:2010:FQQ:1909626.1909813}.

\subsubsection{Error Model Annex for AADL}
The SAE (Society of Automotive Engineers) released the
aerospace standard AS5506, named Architecture Analysis and Design Language (AADL), which is a mature industry-standard for embedded systems and has proved to be efficient for architecture modeling~\cite{aerospace2012sae,liu2016research}. AADL supports safety analysis by adding EMA (Error Model Annex) as an extension to the language. EMA allows the user to annotate system hardware and software architectures with hazard, error propagation, failure modes and effects due to failures. Around 2016, Version 2 of the Error Model Annex was released (EMV2)~\cite{EMV2}. EMV2 is an {\em FLM}-based {\em ESM} approach. The faults and error propagations are explicitly defined and the fault tree analysis is performed by traversing propagation paths in reverse to find the original fault that caused the problem~\cite{feiler2017automated}. 


