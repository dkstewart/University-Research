\section{Safety Critical Systems Development}
\label{sec:criticalSysDev}
As the capabilities of technology grows, so does the complexity and capabilities of mechanical and electrical systems. Many of these systems are safety critical; the loss of correct functioning leads to loss of life, substantial material or environmental damage, or large monetary losses. The development of such complex systems requires a process with clearly defined design and implementation phases which are subdivided into several sub-processes and phases. Certain sets of analyses are required for each of the phases and when the analyses provide satisfactory outcomes, the process transitions into the next phase. 

In general, each field relies on various interpretations of the development process. In the field of aerospace technologies, the Aerospace Recommended Practice (ARP) is what is commonly used. The Society of Automotive Engineers (SAE) is an association of engineers and professionals devoted to the standards that guide the development of transportation systems~\cite{SAE:ARP4761, SAE:ARP4754A}. 

\subsection{The V Model}
The development of safety critical systems is theoretically guided by the V model process as defined in ARP4754~\cite{SAE:ARP4754A}. The V model relates steps of the design phase with a post-implementation phase. It describes how the requirements are produced in the design phase and then how those requirements are verified against the implementation in the post-implementation phase. The left side of the V describes the requirements, architecture, and expected component behavior of the system (see Figure~\ref{fig:v1}). The right side of the V describes the evaluation of the system implementation in light of the requirements. 

\begin{figure}[!htb]
        \center{\includegraphics[width=0.85\textwidth] {images/v1.png}}
        \caption{\label{fig:v1} The V Model in System Development}
\end{figure}

\subsection{Traditional Safety Assessment Process}
ARP4754A, the Guidelines for Development of Civil Aircraft and Systems~\cite{SAE:ARP4754A}, provides guidance on applying development assurance at each hierarchical level throughout the development life cycle of highly-integrated/complex aircraft systems. It has been recognized by the Federal Aviation Administration (FAA) as an acceptable method to establish the assurance process. The safety assessment process is a starting point at each hierarchical level of the development life cycle and is tightly coupled with the system development and verification processes. It is used to show compliance with certification requirements and for meeting a company's internal safety standards. 

\begin{figure}[!htb]
        \center{\includegraphics[width=0.85\textwidth] {images/v2.png}}
        \caption{\label{fig:v2} The V Model in Safety Assessment}
\end{figure}

The safety assessment shown in Figure~\ref{fig:v2} integrates each phase of the V model with analyses specific to system hazards and their severity. It also shows how these hazards should be addressed within the design phase. The safety assessment proess is defined in ARP4754A by the following phases:

\begin{description}
\item[Functional Hazard Assessment (FHA)] examines the functions of the system to identify potential functional failures and classifies the potential hazards associated with them. This includes identification of failure conditions, identifying the effects of those failures, classification of each failure condition, and assignment to safety objectives.

\item[Common Cause Analysis (CCA)] verifies and establishes physical and functional separation, isolation, and independence requirements between subsystems and verifies that these requirements have been met.

\item[Preliminary Aircraft Safety Assessment (PASA)] establishes aircraft safety requirements and provide a preliminary indication that the aircraft can meet those safety requirements.

\item[Preliminary System Safety Assessment (PSSA)] examines the proposed architecture(s) to determine how failures could cause the failure conditions determined by the FHA. The objective is to complete the safety requirements of an aircraft or system and show that the proposed system architecture satisfies the safety requirements. The PSSA is an iterative process that is performed at multiple stages of system development. 

\item[Fault Tree Analysis (FTA)] is performed to find combinations of faults that lead to the violation of a safety requirement. The fault tree itself shows the logical relation between the sets of faults and the violation of a safety requirement.

\item[Common Mode Analysis (CMA)] analyzes designs and implementations for elements that may defeat the redundancy
or independence of functions within the design, i.e. if elements are shown as independent in FTA, make sure they are truly independent in the system under consideration.

\item[Failure Modes and Effect Analysis (FMEA)] aims at finding the causality relationship between sets of faults, intermediate events, and undesired states in the system. Usually this is represented in tabular form and called an \textit{FMEA table}.

\item[Aircraft Safety Assessment (ASA)/System Safety Assessment (SSA)] verifies that the system (or aircraft), as implemented, meets the safety requirements specified by the PSSA.

\end{description}

\section{Model Based Safety Assessment}
\label{sec:mbsa}

The lack of precise models of the system architecture and its failure modes often forces safety analysts to devote significant effort to gathering architectural details about the system behavior from multiple sources. Furthermore, this investigation typically stops at system level, leaving software function details largely unexplored. Typically equipped with the domain knowledge about the system, but not detailed knowledge of how the software applications are designed, practicing safety engineers find it a time consuming and involved process to acquire the knowledge about the behavior of the software applications hosted in a system and its impact on the overall system behavior.
Industry practitioners have come to realize the benefits of using models in the safety assessment process, and a revision of the ARP4761 to include Model Based Safety Analysis (MBSA) is under way. 

\subsection{Suggested Model Based Safety Assessment Process Supported by Formal Methods}
We propose a model-based safety assessment process backed by formal methods to help safety engineers with early detection of the design issues.  This process uses a single unified model to support both system design and safety analysis; this is shown in Figure~\ref{fig:proposed_safety_process} and is based on the following steps:

\begin{figure}[t!]
	\centering
	\includegraphics[trim=0 5 0 5,clip,width=0.85\textwidth]{images/process3.png}
	\caption{Use of the Shared System/Safety Model in the ARP4754A Safety Assessment Process}
	\label{fig:proposed_safety_process}
\end{figure}

\begin{enumerate}
	\item System engineers capture the critical information in a shared model:  high-level hardware and software architecture, nominal behavior at the component level, and safety requirements at the system level.
	\item System engineers use a model checker to check that the safety requirements are satisfied by the nominal design model. 
	\item Safety engineers augment the nominal model with the component failure modes. In addition, safety engineers specify the fault hypothesis for the analysis which corresponds to how many simultaneous faults the system must be able to tolerate.
	\item Safety engineers use a model checker to analyze if the safety requirements and fault tolerance objectives are satisfied by the design in the presence of faults. If the design does not tolerate the specified number of faults (or probability threshold of fault occurrence), then the tool produces counterexamples or minimal sets of fault combinations that can cause the safety requirement to be violated.
	\item The safety engineers examine the results to assess the validity of the fault combinations and the fault tolerance level of the system design. If a design change is warranted, the model will be updated with the latest design change and the above process is repeated.
\end{enumerate}



%There are other tools purpose-built for safety analysis, including AltaRica~\cite{PROSVIRNOVA2013127}, smartIFlow~\cite{info8010007} and xSAP~\cite{DBLP:conf/tacas/BittnerBCCGGMMZ16}. These tools and their accompanying notations are separate from the system development model. Other tools extend existing system models, such as HiP-HOPS~\cite{CHEN201391} and the AADL Error Model Annex, Version 2 (EMV2)~\cite{EMV2}. EMV2 uses enumeration of faults in each component and explicit propagation of faulty behavior to perform error analysis. The required propagation relationships must be manually added to the system model and can become complex, leading to mistakes in the analysis.

%In contrast, the Safety Annex supports model checking and quantitative reasoning by attaching behavioral faults to components and then using the normal behavioral propagation and proof mechanisms built into the AGREE AADL annex.  This allows users to reason about the evolution of faults over time, and produce counterexamples demonstrating how component faults lead to failures.
%Our approach adapts the work of Joshi et. al~\cite{Joshi05:Dasc} to the AADL modeling language.  Stewart, et. al provide more information on the approach~\cite{Stewart17:IMBSA}, and the tool and relevant documentation can be found at: \small \url{https://github.com/loonwerks/AMASE/}. \normalsize
subsection{Critical System Development Artifacts}
\label{subsec:crisysArtifacts}












\begin{comment}
ARP4754A, the Guidelines for Development of Civil Aircraft and Systems~\cite{SAE:ARP4754A}, provides guidance on applying development assurance at each hierarchical level throughout the development life cycle of highly-integrated/complex aircraft systems. It has been recognized by the Federal Aviation Administration (FAA) as an acceptable method to establish the assurance process. The safety assessment process is a starting point at each hierarchical level of the development life cycle and is tightly coupled with the system development and verification processes. It is used to show compliance with certification requirements and for meeting a company's internal safety standards. 

ARP4761, the Guidelines and Methods for Conducting Safety Assessment Process on Civil Airborne Systems and Equipment~\cite{SAE:ARP4761},  identifies a systematic means to show compliance. Among the industry accepted safety assessment processes are Preliminary System Safety Assessment (PSSA) and System Safety Assessment (SSA). PSSA evaluates the system design and defines safety requirements. SSA evaluates the implemented system to show that safety requirements defined in the PSSA are in fact satisfied.

A prerequisite of performing the safety assessment is understanding how the system is intended to work, primarily focusing on the integrity of the outputs and the availability of the system. The safety engineers then use the acquired understanding to conduct safety analysis, construct safety analysis artifacts, and compare the results with established safety objectives and requirements.
Typically equipped with the domain knowledge about the system, but not detailed knowledge of how the software applications are designed, practicing safety engineers find it a time consuming and involved process to acquire the knowledge about the behavior of the software applications hosted in a system and its impact on the overall system behavior.
Industry practitioners have come to realize the benefits of using models in the safety assessment process, and a revision of the ARP4761 to include Model Based Safety Analysis (MBSA) is under way.
Figure~\ref{fig:proposed_safety_process} presents our proposed use of a single unified model to support both system design and safety analysis. It describes both system design and safety-relevant information 
that are kept distinguishable and yet are able to interact with each other. The shared model maintains a living model that captures the current state of the system design as it moves through the development lifecycle, allowing all participants of the ARP4754A process to be able to communicate and review the system design. Safety analysis artifacts can be generated directly from the model, 
providing
the capability to more accurately analyze complex systems.

\begin{figure}[t!]
	
	\centering
	\includegraphics[trim=0 5 0 5,clip,width=0.85\textwidth]{images/process1.png}
	
	\caption{The ARP4754A Safety Assessment Process}
	\label{fig:safety_process}
\end{figure}

\begin{figure}[t!]
	
	\centering
	\includegraphics[trim=0 5 0 5,clip,width=0.85\textwidth]{images/process2.png}
	
	\caption{Use of the Shared System/Safety Model in the ARP4754A Safety Assessment Process}
	\label{fig:proposed_safety_process}
\end{figure}
\end{comment}