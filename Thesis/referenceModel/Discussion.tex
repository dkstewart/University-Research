\subsection{Requirements Satisfaction}\label{sec:discussion}

In this section we set out to construct our initial agent-based reference model
suitable for describing process simulation models.  At the outset we described
some requirements for the reference model; that the model must support modeling
agility and provide process-product independence.  In this section we discuss
how our model satisfies those requirements.

\subsubsection{Modeling Agility}
While our reference model was built around a specific scenario that we wish to
model, we believe the constructs and relationships that comprise the reference
model are suitable for modeling an arbitrary agile process as the reference
model has a means for capturing
\begin{inparaenum}[(1)]
    \item individuals and their interactions,
    \item processes with low process prescription, and
    \item in-process requirements changes.
\end{inparaenum}  We discuss our plans to validate this belief in later
sections.

Our reference model allows implementing models to represent individual behavior
using agents as abstractions of people.  While our reference model does not
prescribe the agent's behavior, it does capture the agent's relationships to
other model constructs.  The specific agent representation will dictate the
behaviors the agent may or may not express; however the established
relationships constrain the otherwise unbounded behavior of the agent.  We also
provide a means for agents to communicate or otherwise acquire knowledge when
its own knowledge is insufficient to complete an activity for a work package.

In our discussion about applying the reference model to our motivational
scenario, we described how an agent can perform work within a process with low
process prescription by empowering the agent to perform activities in whichever
order it chooses.  

Further, we described how we could model in-process requirements change through
a generator agent.  Because artifacts are composed of functions in our
reference model, we have the means to model incremental product development and 
function removal.


\subsubsection{Process Specification Independence}
Driven by our desire to provide a means for process experimentation, we set
out to decouple the process from the product within the simulation
model.  With our reference model, we achieve this by breaking direct
dependencies among activities and work packages.  The work package's artifact set and the
activity's artifact type properties together provide a means to look up the
target artifact of the activity.  If the specified artifact exists, it will
receive the function once produced; if it doesn't exist, the activity will be
skipped.  In this way, we separate the base product and process constructs.

Greater care is required to break coupling caused by knowledge requirements. 
Earlier, we proposed a division of knowledge requirements that we believe will
break the knowledge coupling by associating knowledge with the constructs that
need it the most.  In our reference model, we propose associating
\begin{inparaenum}[(1)]
    \item domain and application knowledge requirements with work
        packages,
    \item foundational skills (e.g. testing ability) with
        activities, and
    \item specific technology skills (e.g. languages or libraries) with
        artifacts.
\end{inparaenum}
Artifacts, representing the project deliverables, are encoded using specific
technologies.  In contrast, activities may be used by multiple processes and
work packages by multiple activities making both constructs poor candidates for
specifying specific technology knowledge requirements.  Work packages,
describing the properties of the work to be performed, require domain and
application knowledge to understand them.  Activities, describing how to
complete a task in a technology independent way, require generalized knowledge
to complete the specified steps.  Thus, we have a partitioning of the knowledge 
in our model that supports process-product independence.

We believe these constructs and relationships sufficiently break the direct
dependencies between the specific process and product concerns allowing a
modeler to change the details of one without altering the other.
