\subsection{Model Based Development}
\label{sec:mbd}
System safety analysis techniqes are well established and used extensively in the design of safety critical systems. These safety analysis techniques are often performed manually based on informal design models and various other documents~\cite{schatz2002model,Joshi05:Dasc}. Fault trees are one of the most common artifacts used by safety engineers, but different engineers may produce substantially different fault trees for the same system. It becomes clear that the analyses are highly subjective and dependent on the skill of the practitioner. Since the analyses are based on informal system documentation, researchers and practitioners have proposed a consolidation of the information into a central entity and use this to perform safety analysis~\cite{joshi2008behavioral, Joshi05:SafeComp, Joshi07:Hase, CAV2015:BoCiGrMa, Bozzano:2010:DSA:1951720, lisagor2011model}.

One way to achieve consolidation of information spread across various informal documents is through \emph{Model-based Development} (MBD)~\cite{schatz2002model}. In MBD, the development is centered around a formal specification or model of the system. This model can be analyzed for completeness and consistency~\cite{heimdahl1996completeness}, model checking~\cite{miller2010software,clarke2018model, grumberg1994model}, theorem proving~\cite{rayadurgam2003using}, test case generation~\cite{anand2013orchestrated,rayadurgam2001coverage}, etc. One can also automate aspects of the implementation from the formal specification. There are several modeling and verification notations that provide these capabilities. 

Model-based Development can also refer to a process that considers a non-formal model, such as SysML~\cite{friedenthal2014practical} or UML~\cite{fowler2003brief}, as the central development artifact. In this dissertation, we consider a formal model of the system in a language with well-defined semantics as the central artifact of the MBD process. 

While there are numerous modeling languages and specifications both in industry and research, we focus on one for this research: the Architecture Analysis and Design Language. 

\subsubsection{Architecture Analysis and Design Language}
The Architectural Analysis and Design Language (AADL) is an SAE International standard language that provides a unifying framework for describing the system architecture for performance-critical, embedded, real-time systems~\cite{AADL_Standard,FeilerModelBasedEngineering2012}. From its conception, AADL has been designed for the design and construction of avionics systems.  Rather than being merely descriptive, AADL models can be made specific enough to support system-level code generation.  
 
An AADL model describes a system in terms of a hierarchy of components and their interconnections, where each component can either represent a logical entity (e.g., application software functions, data) or a physical entity (e.g., buses, processors). An AADL model can be extended with language annexes to provide a richer set of modeling elements for various system design and analysis needs (e.g., performance-related characteristics, configuration settings, dynamic behaviors). The language definition is sufficiently rigorous to support formal analysis tools that allow for early phase error/fault detection. 

Further details regarding AADL will be introduced as needed throughout this dissertation. 