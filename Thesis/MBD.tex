\subsection{Model Based Development}
\label{sec:mbd}
System safety analysis techniqes are well established and used extensively in the design of safety critical systems. These safety analysis techniques are often performed manually based on informal design models and various other documents~\cite{schatz2002model,Joshi05:Dasc}. As mentioned previously, fault trees are one of the most common artifacts used by safety engineers, but different engineers may produce substantially different fault trees for the same system. It becomes clear that the analyses are highly subjective and dependent on the skill of the practitioner. Since the analyses are based on informal system documentation, researchers and practitioners have proposed a consolidation of the information into a central entity and use this entity to perform safety analysis~\cite{joshi2008behavioral, Joshi05:SafeComp, Joshi07:Hase, CAV2015:BoCiGrMa, Bozzano:2010:DSA:1951720, lisagor2011model}.

One way to achieve consolidation of information spread across various informal documents is through \emph{Model-based Development} (MBD)~\cite{schatz2002model}. In MBD, the development is centered on a formal specification or model of the system. This model can be analyzed for completeness and consistency~\cite{heimdahl1996completeness}, model checking~\cite{miller2010software,clarke2018model, grumberg1994model}, theorem proving~\cite{rayadurgam2003using}, test case generation~\cite{anand2013orchestrated,rayadurgam2001coverage}, etc. One can also automate aspects of the implementation from the formal specification. There are several modeling and verification notations that provide these capabilities. 

Model-based Development can also refer to a process that considers a non-formal model, such as SysML~\cite{friedenthal2014practical} or UML~\cite{fowler2003brief}, as the central development artifact. This is commonly referred to as Model-Driven Development (MDD). 

In this dissertation, we consider a formal model of the system in a language with well-defined semantics as the central artifact of the MBD process. 

