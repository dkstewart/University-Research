\subsection{Temporal Logic and Model Checking}
\label{sec:modelchecking}
The requirements have been specified (Section~\ref{sec:formalSpec}), the formal system model has been built (Section~\ref{sec:modeling}), and formal verification is desired (Section~\ref{sec:formalVer}). One of these approaches is called \emph{model checking}. Model checking is an automatic technique for verifying that concurrent systems meet their specified requirements~\cite{clarke2018model}.  

Temporal logics are useful for specifying complex system requirementss, because they can describe the ordering of events in time without introducing time explicitly. 


\textbf{Linear Temporal Logic}
Temporal logic can be used to express properties of reactive systems~\cite{Bozzano:2010:DSA:1951720}. System properties are usually classified into two main categories: {\em safety} properties and {\em liveness} properties. Safety properties express the idea that ``nothing bad ever happens" where liveness properties state that ``something good will eventually happen." 

An example of a safety property is: ``it is never the case that the brake pedal is pressed and no hydraulic pressure is supplied at the wheel." A liveness property, on the other hand, could state: ``eventually the process will complete it's execution." 

Traditionally, two types of temporal logic are used in model checking; Computational Tree Logic (CTL), which is based on a branching logic model, and Linear Temporal Logic (LTL), based on a linear representation of time. This research will focus on LTL. 

An LTL formula is built from a set of atomic propositions, logical operators, and basic temporal operators. The formula is evaluated over a linear path or sequence of states, $s_0, s_1, ..., s_i ,s_{i+1},...$. The following temporal operators are provided:
\begin{itemize}
    \item Globally (\textbf{G}): $G_p$ is true in a state $s_i$ if and only if $p$ is true in all states $s_j$ with $j \geq i$.
    
    \item Finally (\textbf{F}): $F_p$ is true in state $s_i$ if and only if $p$ is true in some state $s_j$ with $j \geq i$.
    
    \item Next (\textbf{X}): $X_p$ is true in state $s_i$ if and only if $p$ is true in the state $s_{i+1}$. 
    
    \item Until (\textbf{U}): $pUq$ is true in state $s_i$ if and only if $q$ is true in some state $s_j$ with $j \geq i$ and $p$ is true in all states $s_k$ such that $i \leq k < j$.
\end{itemize}

Other temporal operators can be defined on the basis of the operators above~\cite{sistla1985complexity}. Formal definitions and more information on LTL and CTL can be found in a number of research works~\cite{Bozzano:2010:DSA:1951720, clarke2018model}.