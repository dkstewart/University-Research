\subsection{Fault Trees and Minimal Cut Sets}
\label{sec:saArtifacts}
\danielle{Move this paragraph to where it makes sense.} Safety analysis has traditionally been performed manually, but with the rise of model checking and the improvement of its capabilities, the world of safety analysis began to see its powerful benefits~\cite{hinchey2012industrial, liggesmeyer1998improving, coudert1993fault, Bozzano:2010:DSA:1951720,bozzano2003esacs}. There arose multiple ways of viewing the system and fault models, various ways of automating the capture of safety pertinent information, and a number of tools that addressed various issues that arose. In this section, we discuss the state of the practice and how formal methods has been applied in the domain of safety assessment research.

Since the early days of safety engineering, fault tree analysis has been a primary method of determining safety of a system and showing the behavior of the system (with respect to its requirements) in the presence of faults~\cite{0f356f05e72f43018211b36f97c8854a,vesely1981fault}. Fault tree analysis requires one to explore the faults of the system and their effects on system behavior to determine minimal fault configurations -- \emph{minimal cut sets} -- that may violate requirements. From the beginning of fault tree analysis in the '60's, algorithms worked directly with the fault tree structure to produce minimal cut sets~\cite{10020219108,semanderes1971elraft}. As the years progressed, it was clear that this approach could not sufficiently address the problem of computation time. In 1993, Rauzy et al. developed a new approach that converted the fault tree structure into a binary decision diagram (BDD)~\cite{rauzy1993new}. This was a natural way to reduce the Boolean formula into something far more computationally efficient and reduceable to even simpler forms. Numerous algorithms were developed to perform variable ordering and minimization of the BDD; this resulted in better computation of MinCutSets and began the process of automating a complex manual safety analysis task~\cite{sinnamon1997new,bryant1986graph,aralia1996computation,reay2002fault,rauzy2007assessment}. BDDs are still commonly used to perform quantitative and qualitative fault tree analysis~\cite{ge2015quantitative,jiang2018algebraic,banov2019new}.

\subsubsection{Failure Mode and Effects Analysis}
\danielle{Is this section necessary? If I don't talk about FMEA tables again, cut this.} Failure Mode and Effects Analysis (FMEA) was one of the first systematic ways of performing dependability analysis and is used throughout the safety critical industries~\cite{rausand2003system,Bozzano:2011:SDP:1992983.1992988}. FMEA provides a structured way to list possible failures and their consequences systemwide. If probabilities of failures are known, quantitative analysis can be performed to estimate system reliability and to assign critical significance to potential failure modes or system components~\cite{MilStandardFMEA}. Performing FMEA is often the first step in the fault tree construction, for it shows possible component failures and hence basic events~\cite{0f356f05e72f43018211b36f97c8854a}. Typically, the failure modes of the components at a given level are considered; the objective it to identify the effects of the failure modes at that level - and usually higher levels - of the design. The FMEA results are often presented in tabular form (FMEA Table). FMEA tables vary in form, but almost always include failure mode definitions, the operational mode in which the failure can occur, and possible causes of the failure~\cite{Bozzano:2010:DSA:1951720}.
