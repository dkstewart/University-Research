\subsection{Granular Refinement}
Model elements considered for the \aivcalg algorithm are explicitly defined in the Lustre code -- these we call \emph{IVC elements}; the generation of Lustre from an AADL/AGREE model provides guarantees and assumptions as IVC elements and when fault analysis is run using the safety annex, constrained faults are also added to this set. This can be seen in the safety analysis modified Lustre code in Figure~\ref{fig:lustreSensorsSafety} which is nominally the same as Figure~\ref{fig:lustreSensors}, but has faulty information added to outputs and IVC elements\footnote{For more information on safety modified Lustre programs, see Chapter XX, Section XX \danielle{ADD THIS TO IMPL}}. 

\danielle{Add figure of lustre code. Highlight IVC elements.}

\danielle{Wording in this paragraph.} We explored granularity within the context of the Lustre language; this programming language provides a nice formalism for discussion because it is top-level conjunctive, equational, and \emph{referentially transparent}: the behavior or a Lustre program is defined by a system of equations, and any subexpression on the right side of an equation can be extracted and assigned to a fresh variable which is substituted into the original equation without changing the meaning of a program~\cite{}(69). In this context, we can define a \emph{granular refinement} as an extraction of a subexpression into a new equation assigning a new variable. 

The maximal factorization of the model can be obtained by assigning each instance of a subexpression and each use of an input to its own variable. This results in a \emph{totally decomposed} Lustre model: (1) each computed (non-input) variable is used at most once in the right side of an equation, (2) each equation is either a single operator or a constant expression, and (3) each model input is directly assigned to one or more fresh variables and is not used elsewhere in the model~\cite{ghassabani_2018}.

Ghassabani did a preliminary analysis on maximally factored models for IVC coverage and found that analysis performed ``significantly slower" for proofs and the \ivcmust algorithm. For our purposes in safety analysis, our concern is both the faults and the guarantees in the Lustre model; therefore, we are able to weaken the factorization performed. In this research, a \emph{partially decomposed} Lustre model has the properties that (1) each computed variable is used at most once in the right side of a equation, and (2) each equation is either a single operator or a constant expression. 

