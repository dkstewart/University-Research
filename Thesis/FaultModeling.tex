\chapter{Fault Modeling and the Safety Annex}
\label{chap:faultModeling}
Early on in Section 2.2.1, a model-based safety assessment process was proposed. This process was backed by formal methods and incorporates a shared model into the development and safety analysis processes. A high level description of this cyclical process is shown in Figure~\ref{fig:SACycle} for your convenience. 

\begin{figure}[h]
	\begin{center}
		\includegraphics[width=\textwidth]{images/SACycle.PNG}
	\end{center}
	\caption{Proposed Steps of the Safety Assessment Process}
	\label{fig:SACycle}
\end{figure}

There are certain capabilities that are required in order to fully perform all steps of this process. In beginning this research, we outlined what those pieces were and investigated related work to determine if a gap still existed. Based on the related work summary found in Section 2.7, it is clear that this work fills certain gaps that no previous research has addressed:

\begin{description}[nosep]

    \item[Shared model] using a language expressive enough to describe HW and SW components.
    \item[Flexible error propagations] through both behavioral and explicit means.
    \item[Flexible fault modeling] with support for a/symmetric faults, in/dependent faults, etc.
    \item[Model checker] used to assess and verify the design with or without faults active.
    \item[Ability to generate artifacts] used in the safety assessment process.
\end{description}

In previous chapters, we discussed how a model checker can be used to generate minimal cut sets in a compositional fashion. In this chapter, the fault modeling process using the Safety Annex for the Architecture Analysis and Design Language (AADL)~\cite{AADL_Standard} is described. The Safety Annex was developed with these two broad ideas in mind: (1) how this supports the proposed safety assessment process, and (2) what missing pieces need to be addressed in order to support safety analysts. 

\section{Fault, Failure, and Error Terminology}
The usage of the terms error, failure, and fault are defined in ARP4754A and are described here for ease of understanding~\cite{SAE:ARP4754A}. An \textit{error} is a mistake made in implementation, design, or requirements. A \textit{fault} is the manifestation of an error and a \textit{failure} is an event that occurs when the delivered service of a system deviates from correct behavior. If a fault is activated under the right circumstances, that fault can lead to a failure. The terminology used in the Error Model Annex version 2 for AADL (EMV2)~\cite{EMV2}, differs slightly for an error: an error is a corrupted state caused by a fault. The error propagates through a system and can manifest as a failure. In this dissertation, we use the ARP4754A terminology with the added definition of \textit{error propagation} as used in EMV2. An error is a mistake made in design or code and an error propagation is the propagation of the corrupted state caused by an active fault. 

\section{Component Fault Modeling}
The Safety Annex is used to add possible faulty behaviors to a component model. Within the AADL component instance model, an annex is added which contain the fault definitions for the given component. The flexibility of the fault definitions allows the user to define numerous types of fault \textit{nodes} by utilizing the AGREE node syntax. %A library of common fault nodes has been written and is available in the project GitHub repository~\cite{SAGithub}. 
Examples of such faults include valves being stuck open or closed, output of a software component being nondeterministic, or power being cut off.  When the fault analysis requires fault definitions that are more complex, these nodes can easily be written and used in the model. 

When a fault is activated by its specified triggering conditions, it modifies the output of the component. This faulty behavior may violate the contracts of other components in the system, including assumptions of downstream components. The impact of a fault is computed by the AGREE model checker when the safety analysis is run on the fault model. 

The majority of faults that are connected to outputs of components are known as \textit{symmetric}. That is, whatever components receive this faulty output will receive the same faulty output value. Thus, this output is seen symmetrically. An alternative fault type is \textit{asymmetric}. This pertains to a component with a 1-n output: one output which is sent to many receiving components. This fault can present itself differently to the receiving components. For instance, in a boolean setting, one component might see a true value and the rest may see false. This is also possible to model using the keyword \textit{asymmetric}. For more details on fault definitions and fault modeling capabilities, we refer readers to the Safety Annex Users Guide\cite{SAGithub}.

\danielle{Need to either explain the example here or use a different example than WBS. WBS is used A LOT in this section, so perhaps a short description is okay...}
As an illustration of fault modeling using the Safety Annex, we look at one of the components important to the inadvertent braking property: the brake pedal. When the mechanical pedal is pressed, a sensor reads this information and passes an electronic signal to the BSCU which then commands hydraulic pressure to the wheels. 

%Figure~\ref{fig:sensor} 
Figure~\ref{fig:sensor} shows the AADL pedal sensor component with a contract for its nominal behavior. The sensor has only one input, the mechanical pedal position, and one output, the electrical pedal position. 
A property that governs the behavior of the component is that the mechanical position should always equal the electronic position. (The expression \textit{true $\rightarrow$ property} in AGREE is true in the initial state and then afterwards it is only true if property holds.)

\begin{figure}[h!]
	\hspace*{-2cm}
	%\vspace{-0.55in} 
	\begin{center}
		\includegraphics[trim=0 640 -10 70,clip,width=1.5\dimexpr\textwidth-2cm\relax]{images/system_sensor.pdf}
		\vspace{-0.3in}
		\caption{An AADL System Type: The Pedal Sensor}
		\label{fig:sensor}
	\end{center}
	\vspace{-0.2in}
\end{figure}

One possible failure for this sensor is inversion of its output value. This fault can be triggered with probability $5.0\times 10^{-6}$ as described in AIR6110 (in reality, the component failure probability is 
collected from hardware specification sheets).  
The Safety Annex definition for this fault is shown in Figure~\ref{fig:sensorFault}. Fault behavior is defined through the use of a fault node called \textit{inverted\_fail}.  When the fault is triggered, the nominal output of the component (\textit{elec\_pedal\_position}) is replaced with its failure value (\textit{val\_out}). 

\begin{figure}[h!]
	\hspace*{-2cm}
	%\vspace{-0.5in} 
	\begin{center}
		\includegraphics[trim=0 680 -10 70,clip,width=1.5\dimexpr\textwidth-2cm\relax]{images/safetyannex_sensorfault.pdf}
		\vspace{-0.2in}
		\caption{The Safety Annex for the Pedal Sensor}
		\label{fig:sensorFault}
	\end{center}
	\vspace{-0.2in}
\end{figure}

The WBS fault model expressed in the Safety Annex contains a total of 33 different fault types and 141 fault instances. The large number of fault instances is due to the redundancy in the system design and its replication to control 8 wheels.

\section{Error Propagation}
As systems become larger and more complex, it can be difficult knowing all possible error propagations within a model; using a purely explicit approach to error propagation is difficult. To this end, we developed the Safety Annex to primarily use \textit{behavioral} propagation. In this approach, the faults are attached to a component's output and ``turned on" in a manner of speaking. The effects and propagation of the active fault is revealed through the behavioral contracts of the system by use of the model checker. 

This section outlines the Safety Annex approach to implicit error propagation and also describes how one can model an explicit propagation by defining dependent faults. 

\subsection{Implicit Propagation}
In the Safety Annex approach, faults are captured as faulty behaviors that augment the system behavioral model in AGREE contracts. No explicit %fault
error propagation is necessary since the faulty behavior itself propagates through the system just as in the nominal system model. The effects of any triggered fault are manifested through analysis of the AGREE contracts. 

On the contrary, in the AADL Error Model Annex, Version 2 (EMV2)~\cite{EMV2} approach, all errors must be explicitly propagated through each component (by applying fault types on each of the output ports) in order for a component to have an impact on the rest of the system. To illustrate the key differences between implicit %failure
error propagation provided in the Safety Annex and the explicit %failure 
error propagation provided in EMV2, we use a simplified behavioral flow from the WBS example using code fragments from EMV2, AGREE, and the Safety Annex. 

\begin{figure}[h]
	%\hspace*{-2cm}
%	\vspace{-0.19in}
	\centering
	\includegraphics[trim=0 9 0 5,clip,width=\textwidth]{images/Comparison_with_EMV2.pdf}
	%\vspace{-0.3in}
	\caption{Differences between Safety Annex and EMV2}
	\label{fig:comparison_with_EMV2}
	%\vspace{-0.2in}
\end{figure} 

In this simplified WBS system, the physical signal from the Pedal component is detected by the Sensor and the pedal position value is passed to the Braking System Control Unit (BSCU) components.  The BSCU generates a pressure command to the Valve component which applies hydraulic brake pressure to the Wheels. 

In the EMV2 approach (top half of Figure~\ref{fig:comparison_with_EMV2}), the ``NoService'' fault is explicitly propagated through all of the components. These fault types are essentially tokens that do not capture any analyzable behavior. At the system level, analysis tools supporting the EMV2 annex can aggregate the propagation information from different components to compose an overall fault flow diagram or fault tree. 

When a fault is triggered in the Safety Annex (bottom half of Figure~\ref{fig:comparison_with_EMV2}), the output behavior of the Sensor component is modified. In this case the result is a ``stuck at zero'' error. The behavior of the BSCU receives a zero input and proceeds as if the pedal has not been pressed. This will cause the top level system contract to fail: {\em pedal pressed implies brake pressure output is positive}.


\subsection{Explicit Propagation} 
%Faults
Failures in hardware (HW) components can trigger behavioral faults in the system components that depend on them. For example, a CPU %fault
Failure may trigger faulty behavior in the threads bound to that CPU. In addition, a %fault
failure in one HW component may trigger %faults
failure in other HW components located nearby, such as overheating, fire, or explosion
in the containment location. 
The Safety Annex provides the capability to explicitly model the impact of hardware %faults
failures on other faults, behavioral or non behavioral. The explicit propagation to non behavioral faults is similar to that provided in EMV2.

To better model %HW dependent faults 
faults at the system level dependent on HW failures, a fault model element is introduced called a \textit{hardware fault}. Users are not required to specify behavioral effects for the HW faults, nor are data ports necessary on which to apply the fault definition. An example of a model component fault declaration is shown below:
\begin{figure}[h!]
	%\vspace{-0.1in}
	\begin{center}
	\includegraphics[width=.6\textwidth]{images/hw_fault2.png}
	\end{center}
	\vspace{-0.1in}
	\caption{Hardware Fault Definition}
	\label{fig:hwFault}
	%\vspace{-0.2in}
	%\vspace{-0.1in}
\end{figure}

Users specify dependencies between the HW component faults and faults that are defined in other components, either HW or SW. The hardware fault then acts as a trigger for dependent faults. This allows a simple propagation from the faulty HW component to the SW components that rely on it, affecting the behavior on the outputs of the affected SW components.

In the WBS example, assume that both the green and blue hydraulic pumps are located in the same compartment in the aircraft and an explosion in this compartment rendered both pumps inoperable. 
The HW fault definition can be modeled first in the green hydraulic pump component as shown in Figure~\ref{fig:hwFault}. The activation of this fault triggers the activation of related faults as seen in the \textit{propagate\_to} statement shown in Figure~\ref{fig:hwFaultProp}. 
Notice that these pumps need not be connected through a data port in order to specify this propagation. %Furthermore, the probability of the HW fault activation can be specified. 

\begin{figure}[h!]
	%\vspace{-0.1in}
	\begin{center}
		\includegraphics[width=1.0\textwidth]{images/hw_prop_stmt.png}
	\end{center}
	\vspace{-0.1in}
	\caption{Hardware Fault Propagation Statement}
	\label{fig:hwFaultProp}
	%\vspace{-0.2in}
	%\vspace{-0.1in}
\end{figure}

The fault dependencies are specified in the system implementation where the system configuration that causes the dependencies becomes clear (e.g., binding between SW and HW components, co-location of HW components). 
%This is because fault propagations are typically tied to the way components are connected or bound together; this information may not be available when faults are being specified for individual components. Having fault propagations specified outside of a component’s fault statements also makes it easier to reuse the component in different systems. 



\section{Fault Analysis Statements}
%An annotation in the AADL model determines the fault analysis statement (also referred to in this report as the fault hypothesis). This may specify either a maximum number of faults that can be active at any point in execution:
The fault analysis statement (also referred to as the fault hypothesis) resides in the AADL system implementation that is selected for verification. This may specify either a maximum number of faults that can be active at any point in execution:

\begin{figure}[h!]
	\vspace{-0.1in}
	%\begin{center}
		\includegraphics[width=0.4\textwidth]{images/hypothesisMaxN.png}
	%\end{center}
	\vspace{-0.1in}
	%%\caption{Max N Faults Analysis Statement}
	\label{fig:hypothesisMaxN}
\end{figure}
or that the only faults to be considered are those whose probability of simultaneous occurrence is above some probability threshold: 

\begin{figure}[h!]
	\vspace{-0.1in}
	%\begin{center}
		\includegraphics[width=0.5\textwidth]{images/hypothesisProb.png}
	%\end{center}
	\vspace{-0.1in}
	%\caption{Probability Analysis Statement}
	%\label{fig:hypothesisProb}
\end{figure}

Tying back to the fault tree analysis in traditional safety analysis, the former is analogous to restricting the cutsets to a specified maximum number of terms, and the latter is analogous to restricting the cutsets to only those whose probability is above some set value. In the former case, we assert that the sum of the true {\em fault\_\_trigger} variables is at or below some integer threshold.  In the latter, we determine all combinations of faults whose probabilities are above the specified probability threshold, and describe this as a proposition over {\em fault\_\_trigger} variables. 
%
With the introduction of dependent faults, active faults are divided into two categories: independently active (activated by its own triggering event) and dependently active (activated when the faults they depend on become active). The top level fault hypothesis applies to independently active faults. Faulty behaviors augment nominal behaviors whenever their corresponding faults are active (either independently active or dependently active).

\section{Asymmetric Fault Modeling}
\label{sec:byzantine}
A \textit{asymmetric} or \textit{Byzantine} fault is a fault that presents different symptoms to different observers~\cite{Driscoll-Byzantine-Fault}. In our modeling environment, asymmetric faults may be associated with a component that has a 1-n output to multiple other components. In this configuration, a \textit{symmetric} fault will result in all destination components seeing the same faulty value from the source component. To capture the behavior of asymmetric faults (``different symptoms to different observers''), it was necessary to extend our fault modeling mechanism in AADL. 

\textbf{Implementation of Asymmetric Faults}
To illustrate our implementation of asymmetric faults, assume a source component A has a 1-n output connected to four destination components (B-E) as shown in Figure~\ref{fig:commNodes} under ``Nominal System.'' If a symmetric fault was present on this output, all four connected components would see the same faulty behavior. An asymmetric fault should be able to present arbitrarily different values to the connected components. 

To this end, ``communication nodes'' are inserted on each connection from component A to components B, C, D, and E (shown in Figure~\ref{fig:commNodes} under ``Fault Model Architecture.'' From the users perspective, the asymmetric fault definition is associated with component A's output and the architecture of the model is unchanged from the nominal model architecture. Behind the scenes, these communication nodes are created to facilitate potentially different fault activations on each of these connections. The fault definition used on the output of component A will be inserted into each of these communication nodes as shown by the red circles at the communication node output in Figure~\ref{fig:commNodes}.
\begin{figure}[!htb]
        \center{\includegraphics[width=\textwidth] {images/commNodes.png}}
        \caption{\label{fig:commNodes} Communication Nodes in Asymmetric Fault Implementation}
\end{figure}

\begin{figure}[!htb]
        \center{\includegraphics[width=\textwidth] {images/asymFaultDef.png}}
        \caption{\label{fig:asymFaultDef} Asymmetric Fault Definition in the Safety Annex}
\end{figure}

An asymmetric fault is defined for Component A as in Figure~\ref{fig:asymFaultDef}. This fault defines an asymmetric failure on Component A that when active, is stuck at a previous value (\textit{prev(Output, 0)}). This can be interpreted as the following: some connected components may only see the previous value of Comp A output and others may see the correct (current) value when the fault is active. This fault definition is injected into the communication nodes and which of the connected components see an incorrect value is completely nondeterministic. Any number of the communication node faults (0…all) may be active upon activation of the main asymmetric fault.

\textbf{Process ID Example}
The illustration of asymmetric fault implementation can be seen through a simple example where 4 nodes report to each other their own process ID (PID). Each node has a 1-3 connection and thus each node is a candidate for an asymmetric fault. Given this architecture, a top level contract of the system is simply that all nodes report and see the correct PID of all other nodes. Naturally in the absence of faults, this holds. But when one asymmetric fault is introduced on any of the nodes, this contract cannot be verified. What is desired is a protocol in which all nodes agree on a value (correct or arbitrary) for all PIDs. 

\textbf{The Agreement Protocol Implementation in AGREE}
In order to mitigate this problem, special attention must be given to the behavioral model. Using the strategies outlined in previous research~\cite{bracha1987asynchronous,Driscoll-Byzantine-Fault}, the agreement protocol is specified in AGREE to create a model resilient to one active Byzantine fault. 

The objective of the agreement protocol is for all correct (non-failed) nodes to eventually reach agreement on the PID values of the other nodes. There are $n$ nodes, possibly $f$ failed nodes. The protocol requires $n > 3f$ nodes to handle a single fault. The point is to achieve distributed agreement and coordinated decisions.
The properties that must be verified in order to prove the protocol works as desired are as follows: 
\begin{itemize}
	\item All correct (non-failed) nodes eventually reach a decision regarding the value they have been given.  In this solution, nodes will agree in $f+1$ time steps or rounds of communication.   
	\item If the source node is correct, all other correct nodes agree on the value that was originally sent by the source.  
	\item If the source node is failed, all other nodes must agree on some predetermined default value.  
\end{itemize}

The updated architecture of the PID example is shown in Figure~\ref{fig:PIDArch}. 
\begin{figure}[!htb]
        \center{\includegraphics[width=\textwidth] {images/PIDArch.png}}
        \caption{\label{fig:PIDArch} Updated PID Example Architecture}
\end{figure}

Each node reports its own PID to all other nodes in the first round of communication. In the second round, each node informs the others what they saw in terms of %everyones
everyone's PIDs. %Thus, new connections are added for this second round of communication.
The outputs from a node are described in Figure~\ref{fig:NodeOutputsPID}. %\janet{The figure needs to be updated, as node 2 sends to node 1, 2, 3; and node 3 sends to node 1, 2, 4; and node 4 sends to node 1, 2, 3}
\begin{figure}[!htb]
        \center{\includegraphics[width=0.9\textwidth] {images/NodeOutputsPID.png}}
        \caption{\label{fig:NodeOutputsPID} Description of the Outputs of Each Node in the PID Example}
\end{figure}
%These new connections 
These outputs are modeled as a nested data implementation in AADL and each field corresponds to a PID from a node. The AADL code fragment defining this data implementation is shown in Figure~\ref{fig:PIDNodeData}.

\begin{figure}[!htb]
        \center{\includegraphics[width=0.7\textwidth] {images/PIDNodeData.png}}
        \caption{\label{fig:PIDNodeData} Data Implementation in AADL for Node Outputs}
\end{figure}

The fault definition for %that is inserted explicitly into the model is connected to 
each node's output and can effect the data fields arbitrarily. This is a nondeterministic fault in two ways. It is nondeterministic how many receiving nodes see incorrect values and it is nondeterministic how many of the data fields are affected by this fault. This can be accomplished through the fault definition shown in Figure~\ref{fig:PIDFaultNode} and the fault node definition in Figure~\ref{fig:PIDFaultDef}. 

\begin{figure}[!htb]
        \center{\includegraphics[width=0.9\textwidth] {images/PIDFaultNode.png}}
        \caption{\label{fig:PIDFaultNode} Fault Definition on Node Outputs for PID Example}
\end{figure}

\begin{figure}[!htb]
        \center{\includegraphics[width=0.9\textwidth] {images/PIDFaultDef.png}}
        \caption{\label{fig:PIDFaultDef} Fault Node Definition for PID Example}
\end{figure}

Once the fault model is in place, the implementation in AGREE of the agreement protocol is developed. As stated previously, there are two cases that must be considered in the contracts of this system. 
\begin{itemize}
	\item In the case of no active faults, all nodes must agree on the correct PID of all other nodes. 
	\item In the case of an active fault on a node, all non-failed nodes must agree on a PID for all other nodes. 
\end{itemize}

These requirements are encoded in AGREE through the use of the following contracts. Figure~\ref{fig:PIDContract1} and Figure~\ref{fig:PIDContract2} show example contracts regarding Node 1 PID. There are similar contracts for each node's PID. 
\begin{figure}[!htb]
        \center{\includegraphics[width=0.9\textwidth] {images/PIDContract1.png}}
        \caption{\label{fig:PIDContract1} Agreement Protocol Contract in AGREE for No Active Faults}
\end{figure}

\begin{figure}[!htb]
        \center{\includegraphics[width=0.9\textwidth] {images/PIDCOntract2.png}}
        \caption{\label{fig:PIDContract2} Agreement Protocol Contract in AGREE Regarding Non-failed Nodes}
\end{figure}


\textbf{Referencing Fault Activation Status}
To fully implement the agreement protocol, it must be possible to describe whether or not a subcomponent is failed by specifying if any faults defined for the subcomponents is activated. In the Safety Annex, this is made possible through the use of a \textit{fault activation} statement. Users can declare boolean \textit{eq} variables in the AGREE annex of the AADL system where the AGREE verification applies to that system's implementation. %in AGREE and in the implementations Safety Annex, 
Users can then assign the activation status of specific faults to those \textit{eq} variables in Safety Annex of the AADL system implementation (the same place where the fault analysis statement resides).
%assigns this boolean \textit{eq} statement to a fault specific to a subcomponent. 
This assignment links each specified AGREE boolean variable with the 
activation status of the specified fault activation literal. %It is 
The AGREE boolean variable is true when and only when the fault is active. An example of this for the PID example is shown in Figure~\ref{fig:PID_faultActivationStmt}. %The \textit{eq} \janet{variables declared } %statements defined in AGREE are: \textit{n1\_failed, n2\_failed, n3\_failed, n4\_failed}. %, the fault name is \textit{Asym\_Fail\_Any\_PID\_To\_Any\_Value} \janet{and the fault definition applies to each of the node subcomponent instance of the top level system: \textit{node1}, \textit{node2}, \textit{node3}, \textit{node4}}. 
Each of the \textit{eq} variables declared in AGREE (i.e., \textit{n1\_failed, n2\_failed, n3\_failed, n4\_failed}) is linked to the fault activation status of the \textit{Asym\_Fail\_Any\_PID\_To\_Any\_Value} fault defined in a node subcomponent instance of the %top level 
AADL system implementation (i.e., \textit{node1}, \textit{node2}, \textit{node3}, \textit{node4}).

\begin{figure}[!htb]
        \center{\includegraphics[width=0.9\textwidth] {images/PID_faultActivationStmt.png}}
        \caption{\label{fig:PID_faultActivationStmt} Fault Activation Statement in PID Example}
\end{figure}

\textbf{PID Example Analysis Results}
The nominal model verification shows that all properties are valid. Upon running verification of the fault model (\textit{Verify in the Presence of Faults}) with one active fault, the first four properties stating that all nodes agree on the correct value (Figure~\ref{fig:PIDContract1}) fail. This is expected since this property is specific to the case when no faults are present in the model. The remaining 4 top level properties (Figure~\ref{fig:PIDContract2}) state that all non-failed nodes reach agreement in two rounds of communication. These are verified valid when any one asymmetric fault is present. This shows that the agreement protocol was successful in eliminating a single point of asymmetric failure from the model. Furthermore, when changing the number of allowed faults to two, these properties do not hold. This is expected given the theoretical result that $3f+1$ nodes are required in order to be resilient to $f$ faults and that $f+1$ rounds of communication are needed for successful protocol implementation. %, this is not surprising. 
%\janet{Since w}e only have 4 nodes and 2 rounds of communication%. 
%\janet{, t}o be resilient to two faults would require $7$ nodes and $3$ rounds of communication. 
A summary of the results follows. 
\begin{itemize}
	\item Nominal model: All top level guarantees are verified. All nodes output the correct value and all agree. 
	\item Fault model with one active fault: The first four guarantees fail (when no fault is present, all nodes agree: shown in Figure~\ref{fig:PIDContract1}). This is expected if faults are present. The last four guarantees (all non-failed nodes agree) are verified as true with one active fault. 
	\item Fault model with two active faults: All 8 guarantees fail. This is expected since in order to be resilient up to two active faults ($f=2$), we would need $3f + 1 = 7$ nodes and $f+1 = 3$ rounds of communication. 
\end{itemize}
This model is in Github and is called PIDByzantineAgreement~\cite{SAGithub}.


\section{Implementation of the Safety Annex}
The Safety Annex is written in Java as a plug-in for the OSATE AADL toolset, which is built on Eclipse.  It is not designed as a stand-alone extension of the language, but works with behavioral contracts specified using the AGREE annex for AADL~\cite{NFM2012:CoGaMiWhLaLu}. 
The architecture of the Safety Annex is shown in Figure~\ref{fig:plugin-arch}.

\begin{figure}[h]
	\begin{center}
		%\includegraphics[trim=0 400 430 0,clip,width=0.85\textwidth]{images/arch.png}
		\includegraphics[width=\textwidth]{images/arch.png}
	\end{center}
	%\vspace{-0.2in}
	\caption{Safety Annex Plug-in Architecture}
	\label{fig:plugin-arch}
	%\vspace{-0.2in}
\end{figure}

AGREE contracts are used to define the nominal behaviors of system components as {\em guarantees} that hold when {\em assumptions} about the values the component's environment are met. When an AADL model is annotated with AGREE contracts and the fault model is created using the Safety Annex, the model is transformed through AGREE into a Lustre model~\cite{Halbwachs91:IEEE} containing the behavioral extensions defined in the AGREE contracts for each system component. 

When performing fault analysis, the Safety Annex extends the AGREE contracts to allow faults to modify the behavior of component inputs and outputs. An example of a portion of an initial AGREE node and its extended contract is shown in Figure~\ref{fig:lustre}. The left column of the figure shows the nominal Lustre pump definition is shown with an AGREE contract on the output; and the right column shows the additional local variables for the fault (boxes 1 and 2), the assertion binding the fault value to the nominal value (boxes 3 and 4), and the fault node definition (box 5). Once augmented with fault information, the AGREE model (translated into the Lustre dataflow language~\cite{Halbwachs91:IEEE}) follows the standard translation path to the model checker JKind~\cite{2017arXiv171201222G}, an infinite-state model checker for safety properties. 

\begin{figure}[h!]
	%\hspace*{-2cm}
	%\vspace{-0.3in} 
	\begin{center}
		%\includegraphics[trim=0 690 -10 70,clip,width=1.5\dimexpr\textwidth-2cm\relax]{images/lustre.pdf}
		\includegraphics[scale=0.3]{images/lustre.jpg}
		%\caption{Nominal AGREE node and its extension with faults}
		\caption{Nominal AGREE Node and Extension with Faults}
		\label{fig:lustre}
	\end{center}
	%\vspace{-0.3in}
\end{figure}

There are two different types of fault analysis that can be performed on a fault model. The Safety Annex plugin intercepts the AGREE program and add fault model information to the model depending on which form of fault analysis is being run.

\textbf{Verification in the Presence of Faults}: This analysis returns one counterexample when fault activation per the fault hypothesis can cause violation of a property. The augmentation from Safety Annex to the AGREE program includes traceability information so that when counterexamples are displayed to users, the active faults for each component are visualized.

\textbf{Generate Minimal Cut Sets}: This analysis collects all minimal set of fault combinations that can cause violation of a property. As described in Chapter~\ref{chap:mcsGen}, the first step of MinCutSet generation is to collect the minimal IVCs for each property. Given the compositional nature of the verification, each level of the system is extended in a slightly different way. The leaf nodes of a system contribute only constrained faults to the \aivcalg algorithm as shown in Figure~\ref{fig:ivcElements1}. 

\begin{figure}[h!]
	\hspace*{-2cm}
	\vspace{-0.1in} 
	\begin{center}
		\includegraphics[scale=0.5]{images/ivcElements1.png}
	\caption{IVC Elements used for Consideration in a Leaf Layer of a System}
		\label{fig:ivcElements1}
	\end{center}
\end{figure}

In the non-leaf layers of the program, both contracts and constrained faults are considered as shown in Figure~\ref{fig:ivcElements2}. The reason for this is that the contracts are used to prove the properties at the next highest level and are necessary for the verification of the properties. 

\begin{figure}[h!]
	\hspace*{-2cm}
	\vspace{-0.1in} 
	\begin{center}
		\includegraphics[scale=0.5]{images/ivcElements2.png}
	\caption{IVC Elements used for Consideration in a Middle Layer of a System}
		\label{fig:ivcElements2}
	\end{center}
\end{figure}

The \aivcalg algorithm returns the minimal set of these elements necessary to prove the properties. This equates to any contracts or inactive faults that must be present in order for the verification of properties in the model. From here, we transform all MIVCs into minimal cut sets.






