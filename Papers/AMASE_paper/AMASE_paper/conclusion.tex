\section{Conclusions \& Future Work}
In this paper, we describe our initial work towards performing MBSA using the AADL architecture description language using a failure effect modeling approach.  Our goal is to be able to perform safety analysis on common models used by systems and safety engineers for functional and non-functional analyses, schedulability, and perhaps system image generation.  To perform this analysis, we use existing capabilities within AADL to describe the structure of the system, and build on the existing AGREE framework for compositional analysis of components.  

As part of our exploration, we are interested in examining the strengths and weaknesses of our FEM and the AADL Error Annex FLM-based approach.  We believe that the FEM approach has advantages both in terms of brevity of specifications and accuracy of results, and can build on existing analyses performed for systems engineering.  However, there are also risks in the FEM approach involving incomplete or mis-specified properties.  

We illustrated the ideas using architecture models based on the Wheel Braking System model in SAE AIR 6110 \cite{AIR6110} and use this in the evaluation of our approach. Using assume-guarantee compositional reasoning techniques, we prove a top level property of the wheel brake system that states when the brake pedals are pressed in the absence of skidding, there will be hydraulic pressure supplied to the brakes.  

Starting from the error model notions of error types, two main faults were defined: \textit{fail\_to} which will describe failures of valves and pressure regulators and \textit{inverted\_fail} which describes the failures occurring to components that output boolean values. Using the AADL behavioral model of the WBS, these permanent faults were tied into the nominal model in order to reason about how this model behaves in the presence of specific kinds of faults.

In order to demonstrate that the system was resilient to single faults, we modified the model to allow feedback from the wheel pressure to the BSCU.   This changed the way the system responded to faults that were further downstream of the BSCU or Selector and created a chance for the system to switch to alternate forms of hydraulic pressure. We also reasoned about the initialization values of the system in regards to which mode is the starting mode. It is crucial for the system to begin in Normal mode in order to function successfully in the presence of faults.  After model modification and a small weakening of our original property to account for feedback delay, the model does fulfill the top level contract even when a permanent fault of one of the high level components is introduced.

The current capabilities of AGREE are well-suited to specifying faults.  Our approach allows for scalar types of unbounded integers and reals, as well as composite types such as tuples and structures.  It is possible to model systems and reason about them in either discrete time or real-time.  However, adding faults to existing components is cumbersome and can obscure the nominal behaviors of the model.  We are currently examining several fault specification languages, giving special consideration to the xSAP modeling language.

Future research work will involve the continuation of development of the methods and tools needed to perform model-based safety analysis at the system architecture level. By introducing a common set of models for both nominal system design and safety analysis, we hope to reduce the cost of development and improve safety. Our hope is to demonstrate the practicality of formal analysis for early detection of safety issues that would be prohibitively expensive to find through testing and inspection. We will base this research on industry standard notations that are being used in airborne and ground-based avionics in order to ensure transition of this technology.

\subsection*{Acknowledgements} This research was funded by NASA AMASE NNL16AB07T and University of Minnesota College of Science and Engineering Graduate Fellowship.


