\subsection{Safety Critical Systems}
\label{sec:SA_background}
A safety critical system is a system whose safety cannot be shown solely by test, whose logic is difficult to comprehend without the aid of analytical tools, and whose failure can directly or indirectly cause significant loss of life or property\cite{SAE:ARP4761}. Guaranteeing safety and reliability of safety critical systems is mandatory. The process that guides this guarantee is highly standardized and controlled~\cite{RTCA:StdC,SAE:ARP4761}. Due to the complexity of critical systems, the field of safety analysis has in recent decades turned to formal methods~\cite{mattarei,Bozzano:2010:DSA:1951720}. In practice, a systems behavior can be described in a variety of ways that include diagrams, textual descriptions, and operational procedures~\cite{SAE:ARP4754A}. These descriptions must be clear and well defined in order to avoid ambiguous interpretation. The formal definition of system behavior has a unique interpretation and is therefore a good candidate for automated analysis in order to validate requirements and spot design flaws~\cite{Joshi05:Dasc}. 

Model checking is a technique used to allow exhaustive and automatic checking of whether a system model (formal system definition) meets a set of formal requirements. As early as the '90's, using model checking for safety requirements began to surface in the critical systems literature\cite{DBLP:conf/safecomp/CimattiGMRTT98,DBLP:conf/edcc/BernardeschiFGM96}. Current tools in safety analysis use model checking techniques during the development and assessment of safety critical systems, \cite{mattarei,CAV2015:BoCiGrMa,symbAltaRica,DBLP:conf/tacas/BittnerBCCGGMMZ16}.






