\section{Preliminaries}
\label{sec:prelim}
\newcommand{\bool}[0]{\mathit{bool}}
\newcommand{\reach}[0]{\mathit{R}}
\newcommand{\ite}[3]{\mathit{if}\ {#1}\ \mathit{then}\ {#2}\ \mathit{else}\ {#3}}

In this paper we consider \emph{safety properties} over infinite-state machines. The states are vectors of boolean variables that define the values of state variables. We assume there are a set of legal \emph{initial states} and the safety property is specified as a propositional formula over state variables. A \emph{reachable state space} means that all states are reachable from the initial state. 

Given a state space $U$, a transition system $(I,T)$ consists of an
initial state predicate $I : U \to \bool$ and a transition step
predicate $T : U \times U \to \bool$.
We define the notion of
reachability for $(I, T)$ as the smallest predicate $\reach : U \to
\bool$ which satisfies the following formulas:
\begin{gather*}
  \forall u.~ I(u) \Rightarrow \reach(u) \\
  \forall u, u'.~ \reach(u) \land T(u, u') \Rightarrow \reach(u')
\end{gather*}
A safety property $P : U \to \bool$ is a state predicate. A safety
property $P$ holds on a transition system $(I, T)$ if it holds on all
reachable states, i.e., $\forall u.~ \reach(u) \Rightarrow P(u)$,
written as $\reach \Rightarrow P$ for short. When this is the case, we
write $(I, T)\vdash P$.

\subsection{Induction}
\danielle{This section  is likely far longer than what it needs to be for this paper. We really just need a brief intro on induction before hitting sec 3.2. Will do a shortening - just wanted to write it all out for my own sake (and thesis sake). But I think we still need a paragraph linking transition systems to the SAT problem. Easy to do without all this verbage.}
For an arbitrary transition system $(I, T)$, computing reachability
can be very expensive or even impossible. Thus, we need a more
effective way of checking if a safety property $P$ is satisfied by the
system. The key idea is to over-approximate reachability. If we can
find an over-approximation that implies the property, then the
property must hold. Otherwise, the approximation needs to be refined.

A good first approximation for reachability is the property itself.
That is, we can check if the following formulas hold:
\begin{gather}
  \forall s.~ I(s) \Rightarrow P(s)
  \label{eq:1-ind-base} \\
  \forall s, s'.~ P(s) \land T(s, s') \Rightarrow P(s')
  \label{eq:1-ind-step}
\end{gather}
If both formulas hold then $P$ is {\em inductive} and holds over the
system. If (\ref{eq:1-ind-base}) fails to hold, then $P$ is violated
by an initial state of the system. If (\ref{eq:1-ind-step}) fails to
hold, then $P$ is too much of an over-approximation and needs to be
refined.

The JKind model checker used in this research uses {\em
  $k$-induction} which unrolls the property over $k$ steps of the
transition system. For example, 1-induction consists of formulas
(\ref{eq:1-ind-base}) and (\ref{eq:1-ind-step}) above, whereas
2-induction consists of the following formulas:
\begin{gather*}
\forall s.~ I(s) \Rightarrow P(s) \\
\forall s, s'.~ I(s) \land T(s, s') \Rightarrow P(s') \\
\forall s, s', s''.~ P(s) \land T(s, s') \land P(s') \land T(s',
  s'') \Rightarrow P(s'')
\end{gather*}
That is, there are two base step checks and one inductive step check.
In general, for an arbitrary $k$, $k$-induction consists of $k$
base step checks and one inductive step check as shown in
Figure~\ref{fig:k-induction} (the universal quantifiers on $s_i$ have
been elided for space). We say that a property is $k$-inductive if it
satisfies the $k$-induction constraints for the given value of $k$.
The hope is that the additional formulas in the antecedent of the
inductive step make it provable.

\begin{figure}
\begin{gather*}
I(s_0) \Rightarrow P(s_0) \\[-2pt]
%
\vdots \\[2pt]
%
I(s_0) \land T(s_0, s_1) \land \cdots \land T(s_{k-2}, s_{k-1})
\Rightarrow P(s_{k-1}) \\[2pt]
%
P(s_0) \land T(s_0, s_1) \land \cdots \land P(s_{k-1}) \land
T(s_{k-1}, s_k) \Rightarrow P(s_k)
\end{gather*}
\caption{$k$-induction formulas: $k$ base cases and one inductive
  step}
\label{fig:k-induction}
\end{figure}

In practice, inductive model checkers often use a combination of the
above techniques. Thus, a typical conclusion is of the form ``$P$ with
lemmas $L_1, \ldots, L_n$ is $k$-inductive''.

\subsection{The SAT Problem}
Boolean Satisfiability (SAT) solvers attempt to determine if there exists a total truth assignment to a given propositional formula, that evaluates to TRUE. Generally, a propositional formula is any combination of the disjunction and conjunction of literals (as an example, $a$ and $\neg a$ are literals). For a given unsatisfiable problem, solvers try to generate a proof of unsatisfiability; this is generally more useful than a proof of satisfiability. Such a proof is dependent on identifying a subset of clauses that make the problem unsatisfiable (UNSAT). 

SAT solvers in model checking work over a constraint system to determine satisfiability. A \textit{constraint system} $C$ is an ordered set of $n$ abstract constraints $\{C_1, C_2, ..., C_n\}$ over a set of variables. The constraint $C_i$ restricts the allowed assignments of these variables in some way~\cite{liffiton2016fast}. Given a constraint system, we require some method of determining, for any subset $S \subseteq C$, whether $S$ is \textit{satisfiable} (SAT) or \textit{unsatisfiable} (UNSAT). When a subset $S$ is SAT, this means that there exists an assignment allowed by all $C_i \in S$; when no such assignment exists, $S$ is considered UNSAT. 

There are several ways of translating a propositional formula into clauses such that satisfiability is preserved~\cite{een2003temporal}. By performing this translation, $k$-inductive model checkers are able to utilize parallel SAT-solving engines to glean information about the proof of a safety property at each inductive step. Expression of the base and induction steps of a temporal induction proof as SAT problems is straightforward. As an example, we look at an arbitrary base case from Figure~\ref{fig:k-induction}.

\begin{gather*}
I(s_0) \land T(s_0, s_1) \land \cdots \land T(s_{k-2}, s_{k-1})
\land \neg P(s_{k-1})
\end{gather*}

When proving correctness it is shown that the formulas are \emph{unsatisfiable}. If an $n^{th}$ inductive-step is unsatisfiable, that means following an $n$-step trace where the property holds, there exists no next state where it fails, i.e., the property $P$ is provable.

%\textbf{Background Information on Toolsuite}
\danielle{My suggestion is to place this at the front of the implementation section. It just seems like it is taking longer to get to the interesting part of the paper by having it here.}

The algorithms in this paper are implemented in the Safety Annex for the Architecture Analysis and Design Language (AADL) and require the Assume-Guarantee Reasoning Environment (AGREE)~\cite{NFM2012:CoGaMiWhLaLu} to annotate the AADL model in order to perform verification using the back-end model checker \jkind~\cite{2017arXiv171201222G}. 

\textbf{Architecture Analysis and Design Language}
We are using the Architectural Analysis and Design Language (AADL) to construct system architecture models of performance-critical, embedded, real-time systems~\cite{AADL_Standard,FeilerModelBasedEngineering2012}. %An AADL model describes a system in terms of a hierarchy of components and their interconnections, where each component can either represent a logical entity (e.g., application software functions, data) or a physical entity (e.g., buses, processors). 
Language annexes to AADL provide a richer set of modeling elements for various system design and analysis needs, and the language definition is sufficiently rigorous to support formal analysis tools that allow for early phase error/fault detection. 

\textbf{Compositional Analysis} 
One way to structure compositional verification is hierarchically: layers of the system architecture are analyzed independently and their composition demonstrates a system property of interest. Compositional verification partitions the formal analysis of a system architecture into verification tasks that correspond into the decomposition of the architecture~\cite{clarke1989compositional}.  A proof consists of demonstrating that the system property is provable given the contracts of its direct subcomponents and the system assumptions~\cite{cofer2012compositional,clarke1989compositional}. When compared to monolithic analysis (i.e., analysis of the flattened model composed of all components), the compositional approach allows the analysis to scale to much larger systems~\cite{NFM2012:CoGaMiWhLaLu,heckel1998compositional,cofer2012compositional}.

\textbf{Assume Guarantee Reasoning Environment}
The Assume Guarantee Reasoning Environment (AGREE) is a tool for formal analysis of behaviors in AADL models and supports compositional verification~\cite{NFM2012:CoGaMiWhLaLu}.  It is implemented as an AADL annex and is used to annotate AADL components with formal behavioral contracts. Each component's contracts includes assumptions and guarantees about the component's inputs and outputs respectively. AGREE translates an AADL model and the behavioral contracts into Lustre~\cite{Halbwachs91:IEEE} and then queries the \jkind model checker to conduct the back-end analysis~\cite{2017arXiv171201222G}. 

\textbf{JKind}
JKind is an open-source industrial infinite-state inductive model checker for safety properties~\cite{2017arXiv171201222G}. Models and properties in JKind are specified in Lustre~\cite{Halbwachs91:IEEE}, a synchronous dataflow language, using the theories of linear real and integer arithmetic. JKind uses SMT-solvers to prove and falsify multiple properties in parallel.

\textbf{Safety Annex for AADL}
The Safety Annex for AADL provides the ability to reason about faults and faulty component behaviors in AADL models~\cite{Stewart17:IMBSA,stewart2020safety}. In the Safety Annex approach, AGREE is used to define the nominal behavior of system components, faults are introduced into the nominal model, and JKind is used to analyze the behavior of the system in the presence of faults. Faults describe deviations from the nominal behavior and are attached to the outputs of components in the system.%The Safety Annex supports behavioral specification of faults and their implicit propagation through behavioral relationships in the model as well as explicit propagation through dependencies. 