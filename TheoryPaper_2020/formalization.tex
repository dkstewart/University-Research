\section{Formalization}
\label{sec:formalization}
Given an initial state $I$ and a transition relation $T$ consisting of conjunctive constraints as defined in section~\ref{sec:prelim}. The set of all nominal guarantees of the system $G$ consists of conjunctive constraints $g \in G$. Given no faults (i.e., nominal system), each $g$ is one of the transition constraints $T_i$ where:

\begin{gather}
T_n = g_1 \land  g_2 \land \cdots \land g_n
\label{eq:Tn}
\end{gather}

We consider an arbitrary layer of analysis of the architecture and assume the property holds of the nominal relation $(I,T_n) \vdash P$. Given that our focus is on safety analysis in the presence of faults, let the set of all faults in the system be  denoted as $F$. A fault $f \in F$ is a deviation from the normal constraint imposed by a guarantee. Without loss of generality, we associate a single fault and an associated fault probability with a guarantee. Each fault $f_i$ is associated with an \emph{activation literal}, $\mathit{af}_i$, that determines whether the fault is active or inactive. %Any ``faults" in a mid-layer are simply violated guarantees, or deviations from normal behavior. 

%The faults in the safety annex are defined on leaf level components. Thus, for the lowest analysis layer, we must take into consideration faults and the guarantees their activation may violate. A fault $f \in F$ is a deviation from the normal constraint imposed by a guarantee. For the purposes of this paper, each guarantee at the leaf layer of analysis has an associated fault. 

To consider the system under the presence of faults, consider a set $GF$ of modified guarantees in the presence of faults and let a mapping be defined from activation literals $\mathit{af}_i \in AF$ to these modified guarantees $\mathit{gf}_i \in GF$. 
\begin{center}
$\mathit{gf}_i =$ \textit{if} $\mathit{af}_i$ \textit{then} $f_i$ \textit{else} $g_i$\\
\label{eq:sigma}
\end{center}

The transition system is composed of the set of modified guarantees $GF$ and a set of conjunctions assigning each of the activation literals $\mathit{af}_i \in AF$ to false: 

\begin{gather}
T = \mathit{gf}_1 \land \mathit{gf}_2 \land \cdots \land \mathit{gf}_n \land \neg \mathit{af}_1 \land \neg \mathit{af}_2 \land \cdots \land \neg \mathit{af}_n
\label{eq:T}
\end{gather}

\begin{theorem} If $(I,T_n) \vdash P$ for $T_n$ defined in equation~\ref{eq:Tn}, then $(I,T) \vdash P$ for $T$ defined in equation~\ref{eq:T}.
\begin{proof}
By the mapping of each constrained activation literal $\neg \mathit{af}_i$ to the associated guarantee $g_i$ and the weakening of the antecedent by introduction of the activation literals, the result is immediate.
\end{proof}
\end{theorem}

Consider the elements of $T$ as a set $GF \cup AF$, where $GF$ are the potentially faulty guarantees and $AF$ consists of the activation literals that determine whether a guarantee is faulty. This is a set that is considered by an SMT-solver for satisfiability during the $k$-induction procedures. The posited problem is thus: $GF \land AF \land \neg P$ for the safety property in question. Recall, if this is an \emph{unsatisfiable} constraint system, then $P$ is provable given these constraints. On the other hand, if it is \emph{satisfiable}, then we know that given the constraints in $GF$ and $AF$, $P$ is not provable. These satisfiability constraints contain the information we wish to find. 

Let us view this in terms of the PWR system example and focus on the temperature sensor subsystem. The safety property to be proved is $G_t$, the supporting guarantees are found in each of the three temperature sensors, $g_{ti}$. Faults $f_{ti}$ are defined for each sensor. The transition system is: 
\begin{gather*}
T = \mathit{gf}_{t1} \land \mathit{gf}_{t2} \land \mathit{gf}_{t3}  \land \neg \mathit{af}_{t1} \land \neg \mathit{af}_{t2} \land \neg \mathit{af}_{t3}
\end{gather*}

The MIVCs for this subsystem layer correspond to all pairwise combinations of constrained activation literals. Intuitively, if any two sensor faults do {\em not} occur, then two of the three sensor guarantees are not violated and the system responds appropriately to high temperature; therefore, $G_t$ is provable. 

The MCSs for this subsystem layer happen to also correspond to all pairwise combinations of constrained activation literals. If any two sensor faults {\em do} occur, then two of the three sensor guarantees will be violated and the system does not respond to high temperature as required. This would result in the inability to prove $G_t$. (Note: it is not always the case that the MCSs are the same as the MIVCs -- in this case it is due to majority voting on three sensors.)

\subsection{Transforming MCS into Minimal Cut Set}
The MCSs contain the information needed to find minimal cut sets, but their elements consist of constrained activation literals and guarantees, whereas minimal cut sets contain faults. An activation literal $\mathit{af}_i$ represents a potentially active fault and $f_i$ is the manifestation of the fault in the transition system. For ease of notation, we define a mapping $\sigma : AF \rightarrow F$ where $\sigma (\mathit{af}_i) = f_i$. Let $MCS = \{\mathit{af}_1, \cdots, \mathit{af}_m\}$ and let $\sigma (MCS) = \{\sigma (\mathit{af}_{1}), \cdots, \sigma (\mathit{af}_{m})\} = \{f_1, \dots, f_m\}$ be a mapping where MCS is a minimal correction set with regard to some property $G$ and $MCS  \subseteq AF$. 

\begin{lemma} $\sigma (MCS)$ is a cut set of $G$. 
\begin{proof}
Assume towards contradiction that $\sigma (MCS)$ is not a cut set of $G$. Then $\mathit{gf}_1 \land \cdots \land \mathit{gf}_n \land \mathit{af}_1 \cdots \land \mathit{af}_m \land \neg \mathit{af}_{k+1} \land \neg \mathit{af}_n \land \neg G$ is unsatisfiable. Thus, the \emph{true} activation literals do not affect the provability of $G$. This contradicts $C \setminus MCS$ is satisfiable. 
\label{lemma:cut}
\end{proof}
\end{lemma}

Notice in lemma~\ref{lemma:cut} that the constraint system requires that a subset of the activation literals to be constrained; these faults \textit{do not} occur. This is a subtlety that allows the model checker to choose the assignment for each unconstrained activation literal for each step of the transition system that leads to failure, and may not involve entirely simultaneous failures.

\begin{lemma} $\sigma(MCS)$ is minimal.
\begin{proof}
Assume toward contradiction that $\sigma(MCS)$ is not minimal with regard to $G$. Then there exists $S \subset MCS$ such that $\sigma(S)$ is a minimal cut set of $G$. This implies that the corresponding constraint system $C \setminus S$ is satisfiable. This contradicts the minimality of MCS.
\label{lemma:min}
\end{proof}
\end{lemma}

Minimal cut sets generated by monolithic analysis look only at explicitly defined faults throughout the architecture and attempt through various techniques to find the minimal violating set for a particular property. In this approach, explicit faults are analyzed as well as supporting guarantees. We view violated guarantees as deviations from nominal behavior and refer to them as ``faulty". Thus, this approach provides an overapproximation of the minimal cut sets compared to a monolithic approach. %Let $\mathit{MonoCuts}$ be the set of all minimal cut sets using a monolithic approach and let $\mathit{CompCuts}$ be the set of minimal cut sets using the above approach. 

\section{Mike: Read this. This is a draft.}
%\danielle{Since monolithic approach doesn't factor in the replacement and then conversion back to DNF, we should probably have something here about that. }
%In a compositional proof, the analysis is performed top down. Given the assumptions, a property $P$ is proven (or disproven) based on the guarantees at the lower level: $gf_1, \dots, gf_n \vdash P$. The next level of analysis iterates through each guarantee and proves it by use of the guarantees $gf'$ at the next lower level: $gf'_1, \dots, gf'_k \vdash gf_1$, and $gf'_1, \dots, gf'_k \vdash gf_2$, and so on. This process continues until the leaf level guarantees been proven and there are no lower level guarantees. 
A compositional proof is performed from the top hierarchical layer down and the analysis results are composed to form a proof of the system level guarantees. This means that at each intermediate layer, the ``top" level guarantees are subsystem guarantees that require their own proof given the system and component assumptions and the properties of each subcomponent. MIVC generation is performed compositionally in this way as well; the system properties as well as any intermediate layer guarantees each have their own minimal inductive validity cores, and by extension their own minimal correction sets. The formulation described above shows the validity of this approach for a single layer; now we show that these results may be composed. 

Recall that the translated system consists of the constrained formulas of the transition system and the {\em negation} of the property; therefore, the MCSs of a property are actually in reference to the negated property. We call the minimal correction sets of a constraint system with a negated property $P$, $MCS(P)$. 

For a safety property $P$, the set of all MCSs are understood as $\lor^{n}_{i=1} MCS_i(P)$; intuitively, this means if all constraints found in any MCS are removed from the constraint system, $\neg P$ is satisfied. For each element $gf_j \in MCS_i$, this is understood as $\land^{m}_{i=1} gf_i$ and speaks to the minimality of the correction set. Thus, for each layer the MCSs form a disjunctive normal formula over the guarantees $gf_i$ of the hierarchical layer. As the proof proceeds down the hierarchy, each of the subcomponent guarantees become a property to be proven and thus MIVCs/MCSs are generated. The composition of the MCSs consists of replacing a contract in a higher layer MCS with the disjunctive normal formula of its own MCSs. After all replacements have been made, the system property formula is converted back into disjunctive normal form. Since a requirement of the minimal correction set generation is to compute {\em all} MIVCs, this requirement holds for each guarantee in the system. If at any point in MIVC generation, not all can be found, the minimal cut sets likewise cannot be computed.

We want to compose the MCS results starting from the leaf contracts (those whose implementation is specified outside of the architectural model) through several layers of the architecture. 


At this time, a simple mapping from $af_i$ to $f_i$ suffices to make these into minimal cut sets.


\begin{comment}
The composition of these results is performed top down and shown in Algorithm~\ref{alg:compose}. For each guarantee found in an MCS, a replacement must be made with its own MCSs. This is done recursively until all replacements have been made. At that time, the formula must be converted back into disjunctive normal form to finish the translation into minimal cut sets. The \texttt{init} method (line 1) is called once per property per layer of analysis after the conversion of MIVCs into MCSs for that property.

\begin{algorithm}[h]
\SetKwFunction{Init}{init}
\SetKwFunction{Resolve}{resolve}
\SetKwFunction{Replace}{replace}
 \SetKwProg{Fn}{Function}{:}{}

	
	\Fn{\Init{$\mathit{\amcs(P)}$}}{
		$\mathit{\amcs(P)}$ in Disjunctive Normal Form\\
		$\amcs(P) \gets \Resolve(\amcs(P))$ \;
		$\mathit{transformToDNF(\amcs(P))}$
	}

	\Fn{\Resolve{$\mathit{\amcs(P)}$}}{
		$\mathit{\amcs(P)}$ in Disjunctive Normal Form
		
		\For{all $MCS_i \in \mathit{\amcs(P)}$}{
			$\Replace(\mathit{MCS_i}, \mathit{\amcs(P)})$\;
		}
		return : \amcs
	}

	\Fn{\Replace{$\mathit{MCS, \amcs(P)}$}}{
		\For{all $gf_j \in MCS$}{
			\eIf{ $\neg \exists MCS(gf_j)$ }{
				$\mathit{\amcs} \gets $replace $gf_j$ in $MCS(P)$ with $af_j$ \;
			}{
				$\mathit{\amcs} \gets $replace $gf_j$ with $\lor^{n}_{i=1} \Resolve(\mathit{MCS(gf_j))}$	\;		
			} 
		}
	}
	\caption{Compose Results}
	\label{alg:compose}
\end{algorithm}




At the end of this algorithm, we have a formula in disjunctive normal form containing the composed results of the MCSs where the elements refer to activation literals. A simple mapping from $af_i$ to $f_i$ suffices to make these into minimal cut sets. Let $\mathit{MonoCuts}$ be the set of all minimal cut sets using a monolithic approach and let $\mathit{CompCuts}$ be the set of minimal cut sets using the compositional approach.

\begin{theorem} $\mathit{MonoCuts} \subseteq \mathit{CompCuts}$.
\begin{proof}
%Let $M \in \mathit{MonoCuts}$ where $M$ is a mimimal cut set for safety property $P$. Let $M = \{f_1, \dots, f_k\}$. Based on equation~\ref{eq:sigma}, this can be written as $\{af_1, \dots, af_k\}$. 



%Then all $f_i \in M$, if active simultaneously, violate the property $P$. A direct translation of this system to a $\sigma$ form constraint system gives: $g_1  \land \cdots \land g_n  \land \neg \mathit{af}_{1} \land \cdots \land \neg \mathit{af}_{n} \land \neg P$. 

%Without loss of generality, let $M = \{f_1, \dots, f_k\}$. Then we know that $g_1  \land \cdots \land g_n  \land \mathit{af}_1 \land \cdots \land \mathit{af}_k \land \neg \mathit{af}_{k+1} \land \cdots \land \neg \mathit{af}_{n} \land \neg P$ is satisfiable. Then $\{\mathit{af}_1, \dots,\mathit{af}_k\}$ is a correction set for the system and can be mapped by $\sigma$ to a minimal cut set by Lemmas \ref{lemma:cut} - \ref{lemma:min}.
\end{proof}
\end{theorem}
\danielle{Proof sketch for feedback.}

Let $M \in MonoCut$, then $M = \{af_1, \dots, af_n\}$. 

Case 1: If $\forall af_i \in M$, $af_i$ is in the highest level of proof, then the corresponding MCS is a top level MCS of $P$ in a compositional proof. Then $M \in CompCut$.

Case 2: If $\exists af_i \in M$ such that $af_i$ is found at a lower level of the proof, then we know that $\exists af_i$ in the model such that $af_i \in MCS$ for some $MCS(gf_j)$. This element was composed as per line 13 in the algorithm into the set of all MCSs of $P$. Then $af_i$ is in some final MCS for $P$ at the end of the composition results of the algorithm. 

\danielle{It seems like this is the only way I can really approach this problem. I have tried a number of other paths -- all of which end in stale mate. By being able to refer to the replacement process we make, I am able to connect a lower level element with the final results of the analysis. Suggestions??}
\end{comment}





\begin{comment}
\danielle{Seems abrupt here - need lead in to describe WHY the cut sets are pairwise faults.} In terms of the PWR example, the minimal cut sets for the temperature subsystem property $G_t$ consist of all pairwise faults on the temperature sensors; if any two faults occur on the sensors at the same time, we violate the temperature subsystem guarantee. 

Once these lower level minimal cut sets are generated, it is a matter of simple set replacement to find the higher level minimal cut sets. This can be easily seen in our example. An MCS at the top level has the element $G_t$. We systematically replace the contract with the faults that cause their violation. This results in three distinct minimal cut sets for $P$ from the temperature subsystem: $\{f_{t1}, f_{t2}\}, \{f_{t1}, f_{t3}, \{f_{t2}, f_{t3}$. All minimal cut sets for $P$ are given as similar pairwise combinations from each subsystem and total 9 for the entire system.

\danielle{Seems I need a theorem to round it out: that replacement will give min cut sets of safety property. Will think about how to formulate this.}

\end{comment}













\begin{comment}
Given an initial state $I$ and a transition relation $T$ consisting of conjunctive constraints as defined in section~\ref{sec:prelim}. The nominal guarantees of the system, $G$, consist of conjunctive constraints $g \in G$. Given no faults, each $g$ is one of the transition constraints $T_i$ where:

\begin{gather}
T_n = g_1 \land  g_2 \land \cdots \land g_n
\label{eq:Tn}
\end{gather}

We assume the property holds of the nominal relation $(I,T_n) \vdash P$. 

Given that our focus is on safety analysis in the presence of faults, let the faults in the system be the set $F$. A fault $f \in F$ is a deviation from the normal constraint imposed by a guarantee. For the purposes of this paper, each guarantee has an associated fault. Without loss of generality, we associate a single fault and an associated fault probability with a guarantee. Each fault $f_i$ is associated with an \emph{activation literal}, $af_i$, that determines whether the fault is active or inactive. 

To consider the system under the presence of faults, consider a set $GF$ of modified guarantees in the presence of faults and let a mapping be defined from activation literals $af_i \in AF$ to these modified guarantees $gf_i \in GF$. 
\begin{center}
$\sigma : AF \rightarrow GF$ \\
$gf_i = \sigma(af_i) =$ if $af_i$ then $f_i$ else $g_i$
\end{center}

The transition system is composed of the set of modified guarantees $GF$ and a set of conjunctions assigning each of the activation literals $af_i \in AF$ to false: 

\begin{gather}
T = gf_1 \land gf_2 \land \cdots \land gf_n \land \neg af_1 \land \neg af_2 \land \cdots \land \neg af_n
\label{eq:T}
\end{gather}

\begin{lemma} If $(I,T_n) \vdash P$ for $T_n$ defined in equation~\ref{eq:Tn}, then $(I,T) \vdash P$ for $T$ defined in equation~\ref{eq:T}.
\begin{proof}
By application of successive evaluations of $\sigma$ on each constrained activation literal $\neg af_i$, the result is immediate.
\end{proof}
\end{lemma}

Consider the elements of $T$ as a set $GF \cup AF$, where $GF$ are the potentially faulty guarantees and $AF$ consists of the activation literals that determine whether a guarantee is faulty. This is a set that is considered by a SAT-solver for satisfiability during the $k$-induction procedures. The posited problem is thus: $GF \land AF \land \neg P$ for the safety property in question. Recall, if this is an \emph{unsatisfiable} constraint system, then $(I,T) \vdash P$. On the other hand, if it is \emph{satisfiable}, then we know that given the constraints in $GF$ and $AF$, $P$ is not provable. These are the exact constraints we wish to find. 

\subsection{Transform the MIVCs into Minimal Cut Sets}
The \aivcalg algorithm collects all {\em minimal unsatisfiable subsets} (MUSs) of a given transition system in terms of the \textit{negation} of the top level property~\cite{Ghassabani2017EfficientGO,bendik2018online}. Formally, an MUS of a constraint system $C$ is a set $M \subseteq C$ such that $M$ is unsatisfiable and $\forall c \in M$ : $M \setminus \{c\}$ is satisfiable. The MUSs are the minimal explanation of the infeasibility of this constraint system; equivalently, these are the minimal sets of model elements necessary for proof of the safety property.

Returning to our running example, this can be illustrated by the following. Given the constraint system $C = \{g_p, g_t, g_r, \neg P\}$, a minimal explanation of the infeasability of this system is the set $\{g_p, g_t, g_r,\}$. If all three guarantees hold, then $P$ is provable. 

A related set is a {\em minimal correction set} (MCS); a MCS $M$ of a constraint system $C$ is a subset $M\subseteq C$ such that $C \setminus M$ is satisfiable and $\forall S \subset M$ : $C \setminus S$ is unsatisfiable. A MCS can be seen to ``correct'' the infeasability of the constraint system by the removal from $C$ the constraints found in an MCS.

In the case of an UNSAT system, we may ask: what will correct this unsatisfiability? Returning to the PWR example, we can find the MCSs of the constraint system: $MCS_1 = \{g_t\}$, $MCS_2 = \{g_p\}$, $MCS_3 = \{g_r\}$. If any single guarantee is violated, a shut down from that subsystem will not get sent when it should and the safety property $P$ will be violated. 

A duality exists between the MUSs of a constraint system and the MCSs as established by Reiter \cite{reiter1987theory}. This duality is defined in terms of \textit{Minimal Hitting Sets} (\textit{MHS}). A hitting set of a collection of sets $A$ is a set $H$ such that every set in $A$ is ``hit'' by $H$; $H$ contains at least one element from every set in $A$. Every MUS of a constraint system is a minimal hitting set of the system's MCSs, and likewise every MCS is a minimal hitting set of the system's MUSs~\cite{liffiton2016fast, reiter1987theory, de1987diagnosing}.

For the PWR top level constraint system, it can be seen that each of the MCSs intersected with the MUS is nonempty. 

Since we are interested in sets of active faults that cause violation of the safety property, we turn our attention to Minimal Cut Sets. A \textit{Minimal Cut Set} (MinCutSet) is a minimal collection of faults that lead to the violation of the safety property. Furthermore, any subset of a MinCutSet will not cause this property violation. %We define a minimal cut set consistently with much of the research in this field~\cite{fta:survey,historyFTA}.
In this running example, the critical guarantees were seen in the first (top) layer of compositional analysis; the violation of the guarantees found in the MCSs provided a minimal set of supporting contracts that contribute to a top level event (violation of a safety property). We desire to compute the minimal cut sets and so a natural question is how to get from a minimal set of violated guarantees to the faults that cause their violation. 

\subsection{Compositionality}
Compositional analysis proceeds from the top layer downward through the architecture of the system model. Faults are defined on leaf level components. Any middle level analysis will provide the MCSs in terms of violated guarantees, which are not valid elements of a minimal cut set. The lowest level of analysis will contain the faults $f_i$ that violate guarantees $g_i$. 

For illustration, the PWR lowest level of analysis is performed per sensor subsystem. We focus on the temperature subsystem which has the guarantee under analysis $g_t = temp > T_t \iff SHUTDOWN$. Each leaf level component (temperature sensors) have associated fault $f_{ti}: $ fail low. Due to the majority voting mechanism, the MIVCs show all possible pairs of faults restricted to \emph{false}. This means, if any combination of two faults do not occur, then the guarantee at the temperature sensor subsystem level is satisfied. 
\begin{gather*}
		MIVC_1(g_t) = \{\neg f_{t1}, \neg f_{t2}\} \\
		MIVC_2(g_t) = \{\neg f_{t1}, \neg f_{t3}\} \\
		MIVC_3(g_t) = \{\neg f_{t2}, \neg f_{t3}\}
\end{gather*}


The hitting set algorithm produces the following MCSs; if any two faults occur, then the guarantee cannot be proven.
\begin{gather*}
		MCS_1(leaf) = \{\neg f_{t1}, \neg f_{t2}\} \\
		MCS_2(leaf) = \{\neg f_{t1}, \neg f_{t3}\} \\
		MCS_3(leaf) = \{\neg f_{t2}, \neg f_{t3}\}
\end{gather*}

And now we have the information required to determine the faults that cause the violation of guarantees at upper levels. 
\end{comment}




