\section{Introduction}
\label{sec:intro}

Risk and safety analyses are important activities used to ensure that critical systems operate in an expected way. From nuclear power plants and airplanes to heart monitors and automobiles, critical systems are ubiquitous in our society. These systems are required to operate safely under nominal and faulty conditions. Guaranteeing that system safety properties hold in the presence of faults is an important aspect of critical systems development and falls under the discipline of safety analysis. Safety analysis produces various safety related artifacts that are used during development and certification of critical systems~\cite{SAE:ARP4761,SAE:ARP4754A}. Examples include {\em minimal cut sets} -- the minimal sets of faults that may violate a safety property and {\em fault trees} -- the Boolean formula whose literals are minimal cut sets. Since the introduction of minimal cut sets in the field of safety analysis~\cite{vesely1981fault}, much research has been performed to address the generation of these sets and associated formulae~\cite{fta:survey,rauzy1993new,historyFTA}. As critical systems get larger, more minimal cut sets are possible with increasing cardinality. In recent years, symbolic model checking has been used to address scaling the analysis of systems with millions of minimal cut sets~\cite{bieber2002combination,schafer2003combining,fta:survey,symbFTA}. 

The state space explosion problem often prevents formal verification from being used on large systems. This problem can arise from combining parallel processes together and attempting to reason monolithically over them. Compositional reasoning takes advantage of the hierarchical organizaton of a system model. A compositional approach verifies each component of the system in isolation and allows global properties to be inferred about the entire system~\cite{berezin1997compositional}. The {\em assume-guarantee} paradigm is commonly used in compositional reasoning where the assumed behavior of the environment implies the guaranteed behavior of the component ~\cite{NFM2012:CoGaMiWhLaLu}.

Using an assume-guarantee reasoning framework, we extend the definition of the nomimal transition system to allow for unconstrained guarantees. We use this idea to reason about all counterexamples to a proof for each layer of analysis, and then transform the results into a Boolean formula describing the satisfiability of the violation of a property. These results are then composed. 

After we provide the formalization, we describe the implementation in the OSATE tool for the Architecture and Analysis Lanugage (AADL)~\cite{FeilerModelBasedEngineering2012}. AADL has two annexes that are of interest to us: the Assume-Guarantee Reasoning Environment (AGREE)~\cite{NFM2012:CoGaMiWhLaLu} and the safety annex~\cite{stewart2020safety}. AGREE provides the assume-guarantee reasoning required for the transition system extension, and the safety annex allows us to define faults on component outputs. To implement the formalization, we look to recent work in formal verification. Ghassabani et al. developed an algorithm that traces a safety property to a minimal set of model elements necessary for proof; this is called the \textit{all minimal inductive validity core} algorithm (\aivcalg)~\cite{GhassabaniGW16,Ghassabani2017EfficientGO,bendik2018online}. Inductive validity cores produce the minimal set of model elements necessary to prove a property. Each set contains the behavioral contracts -- the requirement specifications for components -- used in a proof. We collect all MIVCs per layer to generate the minimal cut sets and thus the fault trees to be composed.

%The \aivcalg algorithm gives the minimal set of contracts required for proof of a safety property. If all of these sets are obtained, we have insight into every proof path for the property. Thus, if we violate at least one contract from every MIVC set, we have in essence ``broken" every proof path. This is the information that is used to perform fault analysis using MIVCs.

%If all of these sets are obtained, we have insight into not only what is necessary for the verification of the property, but we can also find what combination of contracts, if \emph{violated}, will provide a state of the system which makes the safety property unprovable. 

%Safety analysts are often concerned with faults in the system, i.e., when components or subsystems deviate from nominal behavior, and the propagation of errors through the system. To this end, the model elements included in the reasoning process of the \aivcalg algorithm are not only the contracts of the system, but faults as well. This will provide additional insight into how an active fault may violate contracts that directly support the proof of a safety property. 

%In complex critical systems, safety analysts are concerned with hardware faults, how these may propagate to software components reliant on the failed hardware, and other faults whose propagation requires insight into system dynamics. Scaling model checking of complex hardware and software is challenging;  one way to address this problem is to take advantage of the architecture of the system model through a \textit{compositional} approach~\cite{anderson1996model, clarke1989compositional,mcmillan1999verification}. Compositional model checking reduces the verification of a large system into multiple smaller verification problems that can be solved independently and which together guarantee correctness of the original problem.  One way to structure compositional verification is hierarchically: layers of the system architecture are analyzed independently and their composition demonstrates a system property of interest.

This paper presents a compositional approach to generating fault forests (sets of fault trees) and associated minimal cut sets, allowing us to reason uniformly about faults in hardware and software and their impact on system properties. The main contributions of this research include the formalization of the composition of fault forests and its implementation.


The organization of the paper is as follows.  Section 2 describes a running example, Section 3 outlines the formalization of this approach. The implementation of the formalism is discussed in Sections 4 and 5 and related work follows in Section 6. The paper ends with a conclusion and discussion of future work.
