%\documentclass{sig-alternate-05-2015}
%\documentclass[10pt,conference]{IEEEtran}
\documentclass{llncs}

\let\proof\relax
\let\endproof\relax
\let\example\relax
\let\endexample\relax

\usepackage[margin=1.5in]{geometry}
\usepackage{makeidx}
\usepackage{tabularx,colortbl}
\usepackage[dvipsnames]{xcolor}
\usepackage{flushend}
\usepackage{cite}
\usepackage{amsmath}
\usepackage{amsthm}
\usepackage{amssymb}
\usepackage{bm}
\usepackage{epsfig}
\usepackage{stmaryrd}
\usepackage{url}
\usepackage{multirow}
\usepackage{latexsym}
\usepackage{graphics}
\usepackage{graphicx}
\usepackage{enumitem}
\usepackage{comment}
\usepackage{longtable}
\usepackage{supertabular}
\usepackage{times}
\usepackage{listings}
\usepackage{subfigure}
\usepackage{booktabs}
\usepackage{color}
\usepackage{balance}
\usepackage{xspace}
\usepackage{hyperref}
\usepackage[ruled, vlined, linesnumbered]{algorithm2e}
\usepackage[autostyle]{csquotes}
\usepackage[]{algorithm2e}
\usepackage{IEEEtrantools}
%\usepackage{fourier} 
\usepackage{array}
\usepackage{makecell}


%\theoremstyle{Definition}
%\newtheorem{definition}{Definition}
%%%
%\theoremstyle{Theorem}
%\newtheorem{theorem}{Theorem}


%\newcommand{\definition}{\noindent \textbf{Definition} \citation{}}
%\newcommand{\theorem}{\noindent \textbf{Theorem} \citation{}}
%\newcommand{\lemma}{\noindent \textbf{Lemma} \citation{}}

%\newdef{lemma}{Lemma}
%\newdef{definition}{Definition}
%\newdef{theorem}{Theorem}
%\newdef{corollary}{Corollary}
%\newdef{note}{Note}
%\newdef{axiom}{Axiom}

%\newtheorem{theorem}{Theorem}
%\newtheorem{definition}{Definition}

\newcommand{\mkeyword}[1]{\mbox{\texttt{#1}}}
\DeclareMathOperator{\kuop}{uop}
\DeclareMathOperator{\kbop}{bop}
\DeclareMathOperator{\kite}{ite}
\DeclareMathOperator{\kpre}{pre}
\DeclareMathOperator{\dom}{dom}
\DeclareMathOperator{\ktrue}{true}
\DeclareMathOperator{\kfalse}{false}
\DeclareMathOperator{\kselect}{select}
\DeclareMathOperator{\ran}{range}
\newcommand{\lbb}{[\![}
\newcommand{\rbb}{]\!]}
\newcommand{\jkind}{\texttt{JKind}\xspace}
\newcommand{\lustre}{\texttt{Lustre}\xspace}
\newcommand{\agree}{\texttt{AGREE}\xspace}
\newcommand{\ivc}{\textit{IVC}\xspace}
\newcommand{\unsat}{\texttt{UNSAT}\xspace}
\newcommand{\sat}{\texttt{SAT}\xspace}
\newcommand{\aivcalg}{\texttt{\small{All\_MIVCs}}\xspace}
\newcommand{\amcs}{\texttt{\small{All\_MCSs}}\xspace}

\renewcommand\theadalign{bc}
\renewcommand\theadfont{\bfseries}
\renewcommand\theadgape{\Gape[4pt]}
\renewcommand\cellgape{\Gape[4pt]}

\newcommand{\danielle}[1]{\textcolor{blue}{#1}}
\newcommand{\mats}[1]{\textcolor{orange}{#1}}
\newcommand{\mike}[1]{\textcolor{red}{#1}}
\newcommand{\darren}[1]{\textcolor{purple}{#1}}
\newcommand{\janet}[1]{\textcolor{green}{#1}}

\sloppypar



\begin{document}

\definecolor{gold}{rgb}{0.90,.66,0}
\definecolor{dgreen}{rgb}{0,0.6,0}
\newcommand{\stateequiv}{\equiv_{s}}
\newcommand{\traceequiv}{\equiv_{\sigma}}
\newcommand{\ta}{\text{TA}}
\newcommand{\cta}{\text{TA$_{C}$}}
\newcommand{\tta}{\text{TA$_{T}$}}
\newcommand{\ucalg}{\texttt{\small{IVC\_UC}}}
\newcommand{\ucbfalg}{\texttt{\small{IVC\_UCBF}}}
\newcommand\doesnotentail{\mkern-2mu\not\mkern2mu\vdash}
\newcommand{\bool}[0]{\mathit{bool}}
\newcommand{\reach}[0]{\mathit{R}}
\newcommand{\ite}[3]{\mathit{if}\ {#1}\ \mathit{then}\ {#2}\ \mathit{else}\ {#3}}


\title{Composition of Fault Forests}

%\author{Danielle Stewart  \and Michael Whalen \and Mats Heimdahl} 

\author{Danielle Stewart\inst{1}  \and Michael Whalen\inst{1} \and Mats Heimdahl\inst{1} \and Jing (Janet) Liu\inst{2} \and Darren Cofer\inst{2}}

\institute{University of Minnesota, Minneapolis, MN, USA,\\
\email{\{dkstewar, mwwhalen, heimdahl\}@umn.edu} \and Collins Aerospace -- Collins Advanced Technology Center, Cedar Rapids, IA, USA,\\
\email{\{jing.liu, darren.cofer\}@collins.com}}



\maketitle

\begin{abstract}
Safety analysis is used to ensure that critical systems operate within some level of safety when failures are present. As critical systems become more dependent on software components, it becomes more challenging for safety analysts to comprehensively enumerate all possible failure causation paths. Any automated analyses should be sound to sufficiently prove that the system operates within the designated level of safety. This paper presents a compositional approach to the generation of fault forests (sets of fault trees) and minimal cut sets. We use a behavioral fault model to explore how errors may lead to a failure condition. The analysis is performed per layer of the architecture and the results are automatically composed. A complete formalization is given. We implement this by leveraging minimal inductive validity cores produced by an infinite state model checker. This research provides a sound alternative to a monolithic framework. This enables safety analysts to get a comprehensive enumeration of all applicable fault combinations using a compositional approach to generate while generating artifacts required for certification.
%A compositional approach in model checking has provided a means for more scalable model checking analyses than with monolithic approaches. Commonly used safety artifacts include the set of all \emph{minimal cut sets}, minimal sets of faults that may lead to a violation of a safety property as well as the associated fault trees that graphically depict how the activation of faults leads to a top level event. In this paper, we present a novel compositional approach to computing fault trees and minimal cut sets. This research proves that the composition of fault trees is sound and demonstrates an implementation in the OSATE tool suite for AADL. 
\end{abstract}

\section{Introduction}
\label{sec:intro}

Risk and safety analyses are important activities used to ensure that critical systems operate in an expected way. From nuclear power plants and airplanes to heart monitors and automobiles, critical systems are ubiquitous in our society. These systems are required to operate safely under nominal and faulty conditions. Guaranteeing that system safety properties hold in the presence of faults is an important aspect of critical systems development and falls under the discipline of safety analysis. Safety analysis produces various safety related artifacts that are used during development and certification of critical systems~\cite{SAE:ARP4761,SAE:ARP4754A}. Examples include {\em minimal cut sets} -- the minimal sets of faults that may violate a safety property and {\em fault trees} -- the Boolean formula whose literals are minimal cut sets. Since the introduction of minimal cut sets in the field of safety analysis~\cite{vesely1981fault}, much research has been performed to address the generation of these sets and associated formulae~\cite{fta:survey,rauzy1993new,historyFTA,Bozzano:2010:DSA:1951720,rausand2003system}. As critical systems get larger, more minimal cut sets are possible with increasing cardinality. In recent years, symbolic model checking has been used to address scaling the analysis of systems with millions of minimal cut sets~\cite{bieber2002combination,schafer2003combining,fta:survey,contractBasedDesign,symbFTA,DBLP:conf/cav/BozzanoCPJKPRT15}. 

The state space explosion problem often prevents formal verification from being used on large systems. This problem can arise from combining parallel processes together and attempting to reason monolithically over them. Compositional reasoning takes advantage of the hierarchical organizaton of a system model. A compositional approach verifies each component of the system in isolation and allows global properties to be inferred about the entire system~\cite{berezin1997compositional}. The {\em assume-guarantee} paradigm is commonly used in compositional reasoning where the assumed behavior of the environment implies the guaranteed behavior of the component ~\cite{NFM2012:CoGaMiWhLaLu}.

Using an assume-guarantee reasoning framework, we extend the definition of the nomimal transition system to allow for unconstrained guarantees. We use this idea to generate all counterexamples to a proof for each layer of analysis, and then transform the results into a Boolean formula describing the satisfiability of the violation of a property. These results are then composed. 

After we provide the formalization, we describe the implementation in the OSATE tool for the Architecture and Analysis Lanugage (AADL)~\cite{FeilerModelBasedEngineering2012}. AADL has two annexes that are of interest to us: the Assume-Guarantee Reasoning Environment (AGREE)~\cite{NFM2012:CoGaMiWhLaLu} and the safety annex~\cite{stewart2020safety}. AGREE provides the assume-guarantee reasoning required for the transition system extension and the safety annex allows us to define faults on component outputs. 

Recently, Ghassabani et al. developed an algorithm that traces a safety property to a minimal set of model elements necessary for proof; this is called the \textit{all minimal inductive validity core} algorithm (\aivcalg)~\cite{GhassabaniGW16,Ghassabani2017EfficientGO,bendik2018online}. Inductive validity cores produce the minimal set of model elements necessary to prove a property. Each set contains the behavioral contracts -- the requirement specifications for components -- used in a proof. We collect all MIVCs per layer to generate the minimal cut sets and thus the fault trees to be composed.

%The \aivcalg algorithm gives the minimal set of contracts required for proof of a safety property. If all of these sets are obtained, we have insight into every proof path for the property. Thus, if we violate at least one contract from every MIVC set, we have in essence ``broken" every proof path. This is the information that is used to perform fault analysis using MIVCs.

%If all of these sets are obtained, we have insight into not only what is necessary for the verification of the property, but we can also find what combination of contracts, if \emph{violated}, will provide a state of the system which makes the safety property unprovable. 

%Safety analysts are often concerned with faults in the system, i.e., when components or subsystems deviate from nominal behavior, and the propagation of errors through the system. To this end, the model elements included in the reasoning process of the \aivcalg algorithm are not only the contracts of the system, but faults as well. This will provide additional insight into how an active fault may violate contracts that directly support the proof of a safety property. 

%In complex critical systems, safety analysts are concerned with hardware faults, how these may propagate to software components reliant on the failed hardware, and other faults whose propagation requires insight into system dynamics. Scaling model checking of complex hardware and software is challenging;  one way to address this problem is to take advantage of the architecture of the system model through a \textit{compositional} approach~\cite{anderson1996model, clarke1989compositional,mcmillan1999verification}. Compositional model checking reduces the verification of a large system into multiple smaller verification problems that can be solved independently and which together guarantee correctness of the original problem.  One way to structure compositional verification is hierarchically: layers of the system architecture are analyzed independently and their composition demonstrates a system property of interest.

This paper presents a compositional approach to generating fault forests (finite sequences of fault trees) and minimal cut sets, allowing us to reason uniformly about faults in hardware and software and their impact on system properties. The main contributions of this research include the formalization of the composition of fault forests and its implementation.


The organization of the paper is as follows.  Section 2 describes a running example, Section 3 outlines the formalization of this approach. The implementation of the algorithms is discussed in Section 4 and 5 and related work follows in Section 6. The paper ends with a conclusion and discussion of future work.

\section{Running Example}
\label{sec:example}
We present a running example of a simplified sensor system in a Pressurized Water Reactor (PWR). In a typical PWR, the core inside of the reactor vessel produces heat. Pressurized water in the primary coolant loop carries the heat to the steam generator. Within the steam generator, heat from the primary coolant loop vaporizes the water in a secondary loop, producing steam. The steamline directs the steam to the main turbine, causing it to turn the turbine generator, which produces electricity. There are a few important factors that must be considered during safety assessment and system design. An unsafe climb in temperature can cause high pressure and hence pipe rupture, and high levels of radiation could indicate a leak of primary coolant. 

\begin{figure*}[h!]
	%\vspace{-2em}
	\begin{center}
		\includegraphics[width=0.8\textwidth]{images/sensorSysAADL.png}
	\end{center}
	\vspace{-2em}
	\caption{PWR Sensor System}
	\label{fig:sensorSys}
	%\vspace{-2em}
\end{figure*}

The following sensor system can be thought of as a subsystem within a PWR that monitors these factors. A diagram of the model is shown in Figure~\ref{fig:sensorSys} and represents a highly simplified version of a safety critical system. The temperature subsystem details are shown at the bottom of Figure~\ref{fig:sensorSys}; each of the subsystems have a similar architecture.

The subsystems each contain three sensors that monitor pressure, temperature, and radiation. Environmental inputs are fed into each sensor in the model and the redundant sensors monitor temperature, pressure, or radiation respectively. If temperature, pressure, or radiation is too high, a shut down command is sent from the sensors to the parent components. 

\subsection{PWR Nominal Model}
The temperature, pressure, and radiation sensor subsystems use a majority voting mechanism on the sensor values and will send a shut down command based on this output. The safety property of interest in this system is: \emph{shut down when and only when we should}; the AGREE guarantee stating this property is shown in Figure~\ref{fig:shutdownGuar}. 

\begin{figure*}[h!]
	\vspace{-2em}
	\begin{center}
		\includegraphics[width=0.7\textwidth]{images/sensorGuar.PNG}
	\end{center}
	\vspace{-2em}
	\caption{Sensor System Safety Property}
	\label{fig:shutdownGuar}
	%\vspace{-2em}
\end{figure*}

The safety of the system requires a shut down to take place if the temperature, pressure, or radiation levels climb beyond safe levels; thus, a threshold for each subsystem is introduced. If any sensor subsystem reports passing that threshold, a shutdown command is sent. Supporting guarantees are located in each sensor subsystem and correspond to temperature, pressure, and radiation sending a shut down command if sensed inputs are above a given threshold. Each sensor has a similar guarantee. For reference throughout this paper, we provide Figure~\ref{fig:sensorSysContracts} which shows the guarantees and faults of interest for this running example. 

\begin{figure*}[h!]
	%\vspace{-2em}
	\begin{center}
		\includegraphics[width=1.0\textwidth, trim={0 7.5cm 0 0},clip]{images/PWRFigureContracts.png}
	\end{center}
	\vspace{-6em}
	\caption{Sensor System Nominal and Fault Model Details}
	\label{fig:sensorSysContracts}
	%\vspace{-2em}
\end{figure*}

Note: the thresholds vary for pressure, temperature, and radiation. These are given as constants $T_p$, $T_t$, and $T_r$ respectively. The overall (or ``top level") shutdown command is defined notationally as $S$; each sensor subsystem provides their own shutdown command, $S_p$ for example.  The faults are shown as ``fail low" which correspond to the temp (or pressure or radiation) being high, but the sensor reports safe ranges. We also do not list all guarantees and assumptions that are in the model, but only the ones of interest for the illustration. 

\subsection{PWR Fault Model}
The faults that are of interest in this example system are any one of the sensors failing high or low. If sensors report high and a shut down command is sent, we shut down when we should not. On the other hand, if sensors report low when it should be high, a shut down command is not sent and we do not shut down when we should. From the perspective of safety, a false report of low temperature is the main concern. For simplification in this paper, we focus on the failures when sensors report low when they should not.

A fault is defined for each sensor in the system using the safety annex. An example of a temperature sensor fault stuck at high is shown in Figure~\ref{fig:tempSensorFault}.

\begin{figure*}[h!]
	%\vspace{-2em}
	\begin{center}
		\includegraphics[width=0.9\textwidth]{images/tempSensorFault.PNG}
	\end{center}
	\vspace{-2em}
	\caption{Fault on Temperature Sensor Defined in the Safety Annex for AADL}
	\label{fig:tempSensorFault}
	\vspace{-2em}
\end{figure*}

The Safety Annex provides a way to weave the faults into the nominal model by use of the \emph{inputs} and \emph{outputs} keywords. This allows users to define a fault and attach it to the output of a component. The fault shown in Figure~\ref{fig:tempSensorFault} is defined to be a {\em permanent} fault and has probability of occurrence set at $1.0 \times 10^{-5}$. If the fault is active, the error can possibly violate the guarantees of this component or the assumptions of downstream components~\cite{stewart2020safety}. The activation of a fault is not up to the user, but instead left up to the model checker, JKind, to determine if the activation of this fault will contribute to a violation of higher level guarantees. If so, it can be activated during the analysis.




\begin{comment}
\begin{center}
\resizebox{0.5\textwidth}{!}{%
    \begin{tabular}{ | c | c | c |}
      \hline
      \thead{Component} & \thead{Layer of Analysis} & \thead{Guarantee}\\
      \hline
      ReactorSys & Top &  \makecell{Safety Property $P$: \\ $((temp$ $>$ $T_t)$ $\lor$ $ (pressure$ $>$ $ T_p)$  $\lor$ $ (radiation$ $>$ $ T_r))$ \\ $\iff SHUTDOWN$}    \\
      \hline
      TempSys & Leaf  &  \makecell{Guarantee $G_t$: \\  $temp$ $>$ $ T_t \iff SHUTDOWN$}   \\
      \hline
      PressureSys & Leaf  &  \makecell{Guarantee $G_p$: \\ $pressure$ $>$ $ T_p \iff SHUTDOWN$}    \\
	\hline
      RadiationSys & Leaf  &  \makecell{Guarantee $G_r$: \\ $radiation$ $>$ $ T_r \iff SHUTDOWN$}   \\
      \hline
    \end{tabular}}
  \end{center}

\begin{center}
\resizebox{0.5\textwidth}{!}{%
    \begin{tabular}{ | c | c | c |}
      \hline
      \thead{Component} & \thead{Layer of Architecture} & \thead{Faults}\\
      \hline
      Temp Sensors (3) & \makecell{Leaf \\ Components} & \makecell{$f_{t}$: fail low}  \\
      \hline
	Pressure Sensors (3) & \makecell{Leaf \\ Components} & \makecell{$f_{p}$: fail low}  \\
      \hline
	Radiation Sensors (3) & \makecell{Leaf \\ Components} & \makecell{$f_{r}$: fail low}  \\
      \hline
    \end{tabular}}
  \end{center}



\subsection{Using this Example in the Generation of Minimal Cut Sets}
Step by step, we outline how minimal cut sets are generated through the \aivcalg algorithm using the sensor system as an example. For ease of reference, a table is provided giving model elements of interest in the sensor example. We refer to these throughout this section. Note: the thresholds vary for pressure, temperature, and radiation. These are given as constants $T_p$, $T_t$, and $T_r$ respectively. We also do not list all guarantees and assumptions that are in the model, but only the ones of interest for this analysis.

\begin{center}
\resizebox{\textwidth}{!}{%
    \begin{tabular}{ | c | c | c |}
      \hline
      \thead{Component} & \thead{Layer of Analysis} & \thead{Guarantee}\\
      \hline
      ReactorSys & Top &  \makecell{Safety Property $P$: \\ $((temp$ $>$ $T_t)$ $\lor$ $ (pressure$ $>$ $ T_p)$  $\lor$ $ (radiation$ $>$ $ T_r))$ \\ $\iff SHUTDOWN$}    \\
      \hline
      TempSys & Leaf  &  \makecell{Guarantee $g_t$: \\  $temp$ $>$ $ T_t \iff SHUTDOWN$}   \\
      \hline
      PressureSys & Leaf  &  \makecell{Guarantee $g_p$: \\ $pressure$ $>$ $ T_p \iff SHUTDOWN$}    \\
	\hline
      RadiationSys & Leaf  &  \makecell{Guarantee $g_r$: \\ $radiation$ $>$ $ T_r \iff SHUTDOWN$}   \\
      \hline
    \end{tabular}}
  \end{center}

\begin{center}
%\resizebox{\textwidth}{!}{%
    \begin{tabular}{ | c | c | c |}
      \hline
      \thead{Component} & \thead{Layer of Architecture} & \thead{Faults}\\
      \hline
      Temp Sensors (3) & \makecell{Leaf \\ Components} & \makecell{$f_{t}$: fail low}  \\
      \hline
	Pressure Sensors (3) & \makecell{Leaf \\ Components} & \makecell{$f_{p}$: fail low}  \\
      \hline
	Radiation Sensors (3) & \makecell{Leaf \\ Components} & \makecell{$f_{r}$: fail low}  \\
      \hline
    \end{tabular}
  \end{center}

The first two steps of this process are performed top down in conjunction with JKind analysis over a program. Thus, the top layer is analyzed first, then the next and so on. Steps 3 and 4 proceed after all MIVCs for each layer have been generated. We walk through this sensor system example in this fashion. 

\textbf{Step 1a: Preprocessing top layer.} The preprocessing step inserts specific MIVC elements into the Lustre program. The MIVC elements are the model elements considered in the constraint system for a given property. The Safety Annex provides the means to define a fault over the output of a component. This fault is given an unassigned \emph{trigger} Boolean literal in Lustre. If the trigger literal is true, the output of the component is changed. If not, the output remains equivalent to the nominal output of this component~\cite{stewart2020safety, Stewart17:IMBSA}. This trigger in Lustre is called a \emph{fault activation literal}. The IVC elements required in order to perform this transformation are these fault activation literals as well as guarantees. The basic rules used to insert these additional literals into Lustre depend on the analysis layer of that is being formed in Lustre and are as follows. 
\begin{itemize}
\item Leaf layer of analysis: only fault activation literals are added.
\item Middle or top layers: guarantees are added and if a direct subcomponent is a leaf component of the architecture and faults are defined on its outputs, then these faults are also added.
\end{itemize}

At the top level, guarantees of the sensor subsystems are the IVC elements.
$$\boxed{g_t, g_p, g_r}$$

There are distinct constraint systems, one for each property being proved. In this system at the top layer, there is a single property $P$; this results in the following constraint system. 
$$\boxed{C = \{g_t, g_p, g_r, \neg P\}}$$

\textbf{Step 2a: Generate all MIVCs for the constraint system at the top layer.} In order to prove $P$, all three guarantees from the sensor subsystem level are required. 
$$\boxed{MIVC(P) = \{g_t, g_p, g_r\}}$$. 

\textbf{Step 1b: Preprocessing leaf layer.} Model elements for the IVC algorithm consideration are the faults for each sensor, for instance temperature sensors:
$$\boxed{f_{t1}, f_{t2}, f_{t3}}$$ 

The resulting constraint system for the temperature sensor subsystem layer is:
$$\boxed{C = \{\neg f_{t1}, \neg f_{t2}, \neg f_{t3}, \neg g_t\}}$$

\textbf{Step 2b: Generate all MIVCs for the constraint system at the leaf layer.} Due to the majority voting mechanism, the MIVCs show all possible pairs of faults restricted to \emph{false}. This means, if any combination of two faults do not occur, then the guarantees at the temperature sensor subsystem level are satisfied. 
$$\boxed{
	\begin{aligned}
		MIVC_1(g_t) = \{\neg f_{t1}, \neg f_{t2}\} \\
		MIVC_2(g_t) = \{\neg f_{t1}, \neg f_{t3}\} \\
		MIVC_3(g_t) = \{\neg f_{t2}, \neg f_{t3}\}
	\end{aligned}
}$$

At this point, all MIVCs have been successfully generated (which is a requirement of the following algorithms) and we can move on to the generation of minimal cut sets. 

\textbf{Step 3a: Generate MCSs using a hitting set algorithm at the top layer.} Our single MIVC in this case will reveal three associated MCSs. (Notice: $MCS_1 \cap MIVC(P) \neq \emptyset$, and same for $MCS_2$ and $MCS_3$, thus these are hitting sets.)
 $$\boxed{
	\begin{aligned}
		MCS_1(top) = \{g_t\} \\
		MCS_2(top) = \{g_p\} \\
		MCS_3(top) = \{g_r\}
	\end{aligned}
}$$

\textbf{Step 4a: Transform MCSs into MinCutSets at the top layer.} Given that only guarantees are found in the MCSs at this layer, recursion is used to find the faults that cause violation of these guarantees. Using this recursion on $MCS_1$, we show the process further. 

\textbf{Step 3b: Generate MCSs using a hitting set algorithm at the leaf layer.} In step 2b, we found all MIVCs for the contract $g_t$ and send these to the hitting set algorithm. The resulting MCSs are: 
 $$\boxed{
	\begin{aligned}
		MCS_1(leaf) = \{\neg f_{t1}, \neg f_{t2}\} \\
		MCS_2(leaf) = \{\neg f_{t1}, \neg f_{t3}\} \\
		MCS_3(leaf) = \{\neg f_{t2}, \neg f_{t3}\}
	\end{aligned}
}$$

Only constrained faults are found in these MCSs, so we simply remove those constraints and have found the MinCutSets for the contracts $g_t$. These are returned and replace this contract in the top layer $MCS_1(top)$. Here is the end result for $MCS_1(top)$; this can be understood as three of the total minimal cut sets for $P$. 

\begin{equation*}
MCS_1(top) \rightarrow \left\{ \,
\begin{IEEEeqnarraybox}[][c]{l?s}
\IEEEstrut
MinCutSet_1(P) = \{f_{t1}, f_{t2}\}, \\
MinCutSet_2(P) = \{f_{t1},  f_{t3}\}, \\
MinCutSet_3(P) = \{f_{t2}, f_{t3}\}
\IEEEstrut
\end{IEEEeqnarraybox}
\right.
\end{equation*}

After all replacements have been made, we are left with all minimal cut sets for the property of interest ($P$ in this example). 

\end{comment}











%\section{Background}
\label{sec:background}
In order to formalize the approach of this research, some background information must be supplied. Boolean Satisfiability (SAT) solvers attempt to determine if there exists a total truth assignment to a given propositional formula, that evaluates to TRUE. Generally, a propositional formula is any combination of the disjunction and conjunction of literals (as an example, $a$ and $\neg a$ are literals). For a given unsatisfiable problem, solvers try to generate a proof of unsatisfiability; this is generally more useful than a proof of satisfiability. Such a proof is dependent on identifying a subset of clauses that make the problem unsatisfiable (UNSAT). 

SAT solvers in model checking work over a constraint system to determine satisfiability. A \textit{constraint system} $C$ is an ordered set of $n$ abstract constraints $\{C_1, C_2, ..., C_n\}$ over a set of variables. The constraint $C_i$ restricts the allowed assignments of these variables in some way~\cite{liffiton2016fast}. Given a constraint system, we require some method of determining, for any subset $S \subseteq C$, whether $S$ is \textit{satisfiable} (SAT) or \textit{unsatisfiable} (UNSAT). When a subset $S$ is SAT, this means that there exists an assignment allowed by all $C_i \in S$; when no such assignment exists, $S$ is considered UNSAT. 

The \aivcalg algorithm adapts the MARCO scheme~\cite{liffiton2016fast} which generates all MUSs and applies the idea to sequential model checking~\cite{Ghassabani2017EfficientGO}. Thus, MIVCs are MUSs for inductive systems. \danielle{<-- wording?}

Such a constraint system can encode our example. Consider the first layer of analysis (top level of the system): $C = \{g_p, g_t, g_r, \neg P\}$. This constraint system describes the supporting guarantees at the lower level and the constrained safety property at the top level. The reason for the constraint on $P$ is due to the formulation of the constraint system by the model checker we use, JKind. JKind attempts to look at the unsatisfiability of the problem in order to produce proof results. \danielle{I don't like this description - it can be better. Look over these sentences.}

Given a constraint system $C$, there are certain subsets of $C$ that are of interest in terms of satisfiability. Definitions 2-4 are taken from research by Liffiton et al.,~\cite{liffiton2016fast}. 

\begin{definition} : A Minimal Unsatisfiable Subset (MUS) $M$ of a constraint system $C$ is a subset $M \subseteq C$ such that $M$ is unsatisfiable and $\forall c \in M$ : $M \setminus \{c\}$ is satisfiable. 
\end{definition}
\noindent
An MUS can be intuitively understood as the minimal explanation of the constraint systems infeasability. 

Returning to our running example, this can be illustrated by the following. Given the constraint system $C = \{g_p, g_t, g_r, \neg P\}$, one minimal explanation of the infeasability of this system is the set $\{g_p, g_t, g_r,\}$. If all three guarantees can be violated, then there exists an assignment for all elements such that the system is satisfiable. Given that $P$ is an ``OR" statement of all three guarantees, in order to satisfy $\neg P$, all three guarantees must be violated. Due to the level of the architecture under analysis, there are no faults in the system, but instead only guarantees that may be violated. This is always the case with logical subsystems in a compositional approach. 

\begin{definition} : A Minimal Correction Set (MCS) $M$ of a constraint system $C$ is a subset $M\subseteq C$ such that $C \setminus M$ is satisfiable and $\forall S \subset M$ : $C \setminus S$ is unsatisfiable. 
\end{definition}
\noindent
A MCS can be seen to ``correct'' the infeasability of the constraint system by the removal from $C$ the constraints found in an MCS.

Returning to the PWR example, we can see that if all three guarantees hold, then the safety property $P$ will also hold, i.e., the nominal model is verified. But the constraint system is in terms of $\neg P$ which clearly gives an unsatisfiable constraint system. In this case, we may ask: what will correct this unsatisfiability? These are the MCSs for this constraint system: $MCS_1 = \{g_t\}$, $MCS_2 = \{g_p\}$, $MCS_3 = \{g_r\}$. 

A duality exists between the MUSs of a constraint system and the MCSs as established by Reiter \cite{reiter1987theory}. This duality is defined in terms of \textit{Minimal Hitting Sets} (\textit{MHS}). A hitting set of a collection of sets $A$ is a set $H$ such that every set in $A$ is ``hit'' by $H$; $H$ contains at least one element from every set in $A$. Every MUS of a constraint system is a minimal hitting set of the system's MCSs, and likewise every MCS is a minimal hitting set of the system's MUSs~\cite{liffiton2016fast, reiter1987theory, de1987diagnosing}.

For the PWR top level constraint system, it can be seen that each of the MCSs intersected with the MUS is nonempty. 

Since we are interested in sets of active faults that cause violation of the safety property, we turn our attention to Minimal Cut Sets. 
\begin{definition}
A \textit{Minimal Cut Set} (MinCutSet) is a minimal collection of faults that lead to the violation of the safety property. Furthermore, any subset of a MinCutSet will not cause this property violation. 
\end{definition}
\noindent
We define a minimal cut set consistently with much of the research in this field~\cite{fta:survey,historyFTA}.

\subsection{Inductive Validity Cores}
Given a safety property, a model checker can be invoked in order to construct a proof of the property.  It is often useful to extract traceability information related to the proof, in
other words, which portions of the model were necessary to construct the proof.  Minimal Inductive Validity Cores (MIVCs) describe the minimal
model elements necessary for the inductive proof of a safety property for a sequential system~\cite{GhassabaniGW16}.  MIVCs can be considered a generalization of UNSAT cores for sequential systems.  Further work extended the MIVC algorithms to produce all mimimal sets of MIVC elements (\aivcalg)~\cite{Ghassabani2017EfficientGO,bendik2018online}.  These algorithms are incorporated into the JKind model checker~\cite{2017arXiv171201222G} which is given a model in the dataflow programming language Lustre~\cite{Halbwachs91:IEEE} 

The \aivcalg algorithm collects all minimal unsatisfiable subsets of a given transition system in terms of the \textit{negation} of the top level property~\cite{Ghassabani2017EfficientGO,bendik2018online}. Assuming that the nominal model proves (no faults are active), it is not surprising that the model elements (guarantees and assumptions of components) and the negation of the safety property is UNSAT. The MUSs are the minimal explanation of the infeasibility of this constraint system; equivalently, these are the minimal sets of model elements necessary for proof of the safety property.

\subsection{High Level Overview of the Main Idea}

We utilize the \aivcalg algorithm by providing not only component contracts as model elements, but also faults constrained to \textit{false}, i.e., the faults are inactive. We first check that the property is true of the system when no faults occur; if not, then the system must be repaired until the property is true before starting this analysis.  With this assumption, the resulting the resulting MIVCs (MUSs) will contain the required contracts and constrained faults necessary to prove the safety property. A high level summary of the steps of this transformation are shown in Figure~\ref{fig:trans}. 

\begin{figure*}[h!]
	%\vspace{-0.1in}
	\begin{center}
		\includegraphics[width=0.6\textwidth]{images/highLevelIdea.PNG}
	\end{center}
	\caption{Steps of the Transformation Process}
	\label{fig:trans}
\end{figure*}

Returning to the PWR example, we focus now on a lower layer of analysis -- proving the guarantee at the temperature subsystem level. The constraint system will now consist of constrained faults of the sensors and the negation of the property of interest $g_t$: $C = \{\neg f_{t1}, \neg f_{t2}, \neg f_{t3}, \neg g_t\}$. 

The MIVCs (or MUSs) of this sytem correspond to all pairs of faults constrained to faults due to the majority voting mechanism. Simply put, if any two faults \emph{occur}, then the negation of $g_t$ is satisfiable. 

Because of the duality between MUSs and MCSs, all MCSs can be obtained by finding the hitting sets of all MUSs. The MCS can be seen to correct the infeasibility of the constraint system and provides the minimal such correction. By removing the constraints from $C$ that are found in any MCS, $C$ becomes satisfiable. In terms of the constraint system that includes fault activation literals, by \textit{activating} the faults in the MCS and \textit{violating} the contracts in the MCS, we can demonstrate the \textit{negation} of the property $P$. 

To illustrate, notice that in the PWR example, all MIVCs are given as: 
$$\boxed{
	\begin{aligned}
		MIVC_1(g_t) = \{\neg f_{t1}, \neg f_{t2}\} \\
		MIVC_2(g_t) = \{\neg f_{t1}, \neg f_{t3}\} \\
		MIVC_3(g_t) = \{\neg f_{t2}, \neg f_{t3}\}
	\end{aligned}
}$$

The hitting sets of all MIVCs are: 
 $$\boxed{
	\begin{aligned}
		MCS_1 = \{\neg f_{t1}, \neg f_{t2}\} \\
		MCS_2 = \{\neg f_{t1}, \neg f_{t3}\} \\
		MCS_3 = \{\neg f_{t2}, \neg f_{t3}\}
	\end{aligned}
}$$

By removing the constraints from the elements in each MCS, we have the ``correction" to the infeasible constraint system; an assignment exists such that $\neg g_t$ is satisfiable. These are the minimal cut sets for $g_t$. 

In summary, if the contracts in the MCS are replaced with the faults that cause its violation, the MCS can be transformed into a MinCutSet. 

%The algorithms in this paper are implemented in the Safety Annex for the Architecture Analysis and Design Language (AADL) and require the Assume-Guarantee Reasoning Environment (AGREE)~\cite{NFM2012:CoGaMiWhLaLu} to annotate the AADL model in order to perform verification using the back-end model checker \jkind~\cite{2017arXiv171201222G}. 

\textbf{Architecture Analysis and Design Language}
We are using the Architectural Analysis and Design Language (AADL) to construct system architecture models of performance-critical, embedded, real-time systems~\cite{AADL_Standard}. %An AADL model describes a system in terms of a hierarchy of components and their interconnections, where each component can either represent a logical entity (e.g., application software functions, data) or a physical entity (e.g., buses, processors). 
Language annexes to AADL provide a rich set of modeling elements for various system design and analysis needs, and the language definition is sufficiently rigorous to support formal analysis tools that allow for early fault detection. Figure~\ref{fig:tempSensor} shows a temperature sensor component defined in AADL. It has a single input (\texttt{Env\_Temp}) and a single output (\texttt{High\_Temp\_Indicator}).  

\begin{figure*}[h!]
	\vspace{-2em}
	\begin{center}
		\includegraphics[width=0.9\textwidth]{images/tempSensoraadlannex.png}
	\end{center}
	\vspace{-2em}
	\caption{Temperature sensor in AADL with AGREE and safety annexes}
	\label{fig:tempSensor}
	\vspace{-2em}
\end{figure*}

%\textbf{Compositional Analysis} 
%One way to structure compositional verification is hierarchically: layers of the system architecture are analyzed independently and their composition demonstrates a system property of interest. Compositional verification partitions the formal analysis of a system architecture into verification tasks that correspond into the decomposition of the architecture~\cite{clarke1989compositional}.  A proof consists of demonstrating that the system property is provable given the contracts of its direct subcomponents and the system assumptions~\cite{cofer2012compositional,clarke1989compositional}. When compared to monolithic analysis (i.e., analysis of the flattened model composed of all components), the compositional approach allows the analysis to scale to much larger systems~\cite{NFM2012:CoGaMiWhLaLu,heckel1998compositional,cofer2012compositional}.

\textbf{Assume Guarantee Reasoning Environment}
The Assume Guarantee Reasoning Environment (AGREE) is a tool for formal analysis of behaviors in AADL models and supports compositional verification~\cite{NFM2012:CoGaMiWhLaLu}.  It is implemented as an AADL annex and is used to annotate AADL components with formal behavioral contracts. Each component's contracts includes assumptions and guarantees about the component's inputs and outputs respectively. AGREE translates an AADL model and the behavioral contracts into Lustre~\cite{Halbwachs91:IEEE} and then queries the \jkind model checker to conduct the back-end analysis~\cite{2017arXiv171201222G}. Figure~\ref{fig:tempSensor} shows the guarantee defined in the temperature sensor written in the AGREE annex.

%\textbf{JKind}
%JKind is an open-source industrial infinite-state inductive model checker for safety properties~\cite{2017arXiv171201222G}. Models and properties in JKind are specified in Lustre~\cite{Halbwachs91:IEEE}, a synchronous dataflow language, using the theories of linear real and integer arithmetic. JKind uses SMT-solvers to prove and falsify multiple properties in parallel.

\textbf{Safety Annex for AADL}
The Safety Annex for AADL provides the ability to reason about faults and faulty component behaviors in AADL models~\cite{Stewart17:IMBSA,stewart2020safety}. In the Safety Annex approach, AGREE is used to define the nominal behavior of system components, faults are introduced into the nominal model, and JKind is used to analyze the behavior of the system in the presence of faults. Faults describe deviations from the nominal behavior and are attached to the outputs of components in the system. An example of a safety annex fault definition on the temperature sensor is shown in Figure~\ref{fig:tempSensor}. The fault is associated with the \texttt{High\_Temp\_Indicator} output and has a probability of occurrence of $1.0 \times 10^{-5}$. The behavior of the output when the fault is active is to report low temperature. 



%\section{Running Example}
\label{sec:example}
We present a running example of a simplified sensor system in a Pressurized Water Reactor (PWR). In a typical PWR, the core inside of the reactor vessel produces heat. Pressurized water in the primary coolant loop carries the heat to the steam generator. Within the steam generator, heat from the primary coolant loop vaporizes the water in a secondary loop, producing steam. The steamline directs the steam to the main turbine, causing it to turn the turbine generator, which produces electricity. There are a few important factors that must be considered during safety assessment and system design. An unsafe climb in temperature can cause high pressure and hence pipe rupture, and high levels of radiation could indicate a leak of primary coolant. 

\begin{figure*}[h!]
	%\vspace{-2em}
	\begin{center}
		\includegraphics[width=0.8\textwidth]{images/sensorSysAADL.png}
	\end{center}
	\vspace{-2em}
	\caption{PWR Sensor System}
	\label{fig:sensorSys}
	%\vspace{-2em}
\end{figure*}

The following sensor system can be thought of as a subsystem within a PWR that monitors these factors. A diagram of the model is shown in Figure~\ref{fig:sensorSys} and represents a highly simplified version of a safety critical system. The temperature subsystem details are shown at the bottom of Figure~\ref{fig:sensorSys}; each of the subsystems have a similar architecture.

The subsystems each contain three sensors that monitor pressure, temperature, and radiation. Environmental inputs are fed into each sensor in the model and the redundant sensors monitor temperature, pressure, or radiation respectively. If temperature, pressure, or radiation is too high, a shut down command is sent from the sensors to the parent components. 

\subsection{PWR Nominal Model}
The temperature, pressure, and radiation sensor subsystems use a majority voting mechanism on the sensor values and will send a shut down command based on this output. The safety property of interest in this system is: \emph{shut down when and only when we should}; the AGREE guarantee stating this property is shown in Figure~\ref{fig:shutdownGuar}. 

\begin{figure*}[h!]
	\vspace{-2em}
	\begin{center}
		\includegraphics[width=0.7\textwidth]{images/sensorGuar.PNG}
	\end{center}
	\vspace{-2em}
	\caption{Sensor System Safety Property}
	\label{fig:shutdownGuar}
	%\vspace{-2em}
\end{figure*}

The safety of the system requires a shut down to take place if the temperature, pressure, or radiation levels climb beyond safe levels; thus, a threshold for each subsystem is introduced. If any sensor subsystem reports passing that threshold, a shutdown command is sent. Supporting guarantees are located in each sensor subsystem and correspond to temperature, pressure, and radiation sending a shut down command if sensed inputs are above a given threshold. Each sensor has a similar guarantee. For reference throughout this paper, we provide Figure~\ref{fig:sensorSysContracts} which shows the guarantees and faults of interest for this running example. 

\begin{figure*}[h!]
	%\vspace{-2em}
	\begin{center}
		\includegraphics[width=1.0\textwidth, trim={0 7.5cm 0 0},clip]{images/PWRFigureContracts.png}
	\end{center}
	\vspace{-6em}
	\caption{Sensor System Nominal and Fault Model Details}
	\label{fig:sensorSysContracts}
	%\vspace{-2em}
\end{figure*}

Note: the thresholds vary for pressure, temperature, and radiation. These are given as constants $T_p$, $T_t$, and $T_r$ respectively. The overall (or ``top level") shutdown command is defined notationally as $S$; each sensor subsystem provides their own shutdown command, $S_p$ for example.  The faults are shown as ``fail low" which correspond to the temp (or pressure or radiation) being high, but the sensor reports safe ranges. We also do not list all guarantees and assumptions that are in the model, but only the ones of interest for the illustration. 

\subsection{PWR Fault Model}
The faults that are of interest in this example system are any one of the sensors failing high or low. If sensors report high and a shut down command is sent, we shut down when we should not. On the other hand, if sensors report low when it should be high, a shut down command is not sent and we do not shut down when we should. From the perspective of safety, a false report of low temperature is the main concern. For simplification in this paper, we focus on the failures when sensors report low when they should not.

A fault is defined for each sensor in the system using the safety annex. An example of a temperature sensor fault stuck at high is shown in Figure~\ref{fig:tempSensorFault}.

\begin{figure*}[h!]
	%\vspace{-2em}
	\begin{center}
		\includegraphics[width=0.9\textwidth]{images/tempSensorFault.PNG}
	\end{center}
	\vspace{-2em}
	\caption{Fault on Temperature Sensor Defined in the Safety Annex for AADL}
	\label{fig:tempSensorFault}
	\vspace{-2em}
\end{figure*}

The Safety Annex provides a way to weave the faults into the nominal model by use of the \emph{inputs} and \emph{outputs} keywords. This allows users to define a fault and attach it to the output of a component. The fault shown in Figure~\ref{fig:tempSensorFault} is defined to be a {\em permanent} fault and has probability of occurrence set at $1.0 \times 10^{-5}$. If the fault is active, the error can possibly violate the guarantees of this component or the assumptions of downstream components~\cite{stewart2020safety}. The activation of a fault is not up to the user, but instead left up to the model checker, JKind, to determine if the activation of this fault will contribute to a violation of higher level guarantees. If so, it can be activated during the analysis.




\begin{comment}
\begin{center}
\resizebox{0.5\textwidth}{!}{%
    \begin{tabular}{ | c | c | c |}
      \hline
      \thead{Component} & \thead{Layer of Analysis} & \thead{Guarantee}\\
      \hline
      ReactorSys & Top &  \makecell{Safety Property $P$: \\ $((temp$ $>$ $T_t)$ $\lor$ $ (pressure$ $>$ $ T_p)$  $\lor$ $ (radiation$ $>$ $ T_r))$ \\ $\iff SHUTDOWN$}    \\
      \hline
      TempSys & Leaf  &  \makecell{Guarantee $G_t$: \\  $temp$ $>$ $ T_t \iff SHUTDOWN$}   \\
      \hline
      PressureSys & Leaf  &  \makecell{Guarantee $G_p$: \\ $pressure$ $>$ $ T_p \iff SHUTDOWN$}    \\
	\hline
      RadiationSys & Leaf  &  \makecell{Guarantee $G_r$: \\ $radiation$ $>$ $ T_r \iff SHUTDOWN$}   \\
      \hline
    \end{tabular}}
  \end{center}

\begin{center}
\resizebox{0.5\textwidth}{!}{%
    \begin{tabular}{ | c | c | c |}
      \hline
      \thead{Component} & \thead{Layer of Architecture} & \thead{Faults}\\
      \hline
      Temp Sensors (3) & \makecell{Leaf \\ Components} & \makecell{$f_{t}$: fail low}  \\
      \hline
	Pressure Sensors (3) & \makecell{Leaf \\ Components} & \makecell{$f_{p}$: fail low}  \\
      \hline
	Radiation Sensors (3) & \makecell{Leaf \\ Components} & \makecell{$f_{r}$: fail low}  \\
      \hline
    \end{tabular}}
  \end{center}



\subsection{Using this Example in the Generation of Minimal Cut Sets}
Step by step, we outline how minimal cut sets are generated through the \aivcalg algorithm using the sensor system as an example. For ease of reference, a table is provided giving model elements of interest in the sensor example. We refer to these throughout this section. Note: the thresholds vary for pressure, temperature, and radiation. These are given as constants $T_p$, $T_t$, and $T_r$ respectively. We also do not list all guarantees and assumptions that are in the model, but only the ones of interest for this analysis.

\begin{center}
\resizebox{\textwidth}{!}{%
    \begin{tabular}{ | c | c | c |}
      \hline
      \thead{Component} & \thead{Layer of Analysis} & \thead{Guarantee}\\
      \hline
      ReactorSys & Top &  \makecell{Safety Property $P$: \\ $((temp$ $>$ $T_t)$ $\lor$ $ (pressure$ $>$ $ T_p)$  $\lor$ $ (radiation$ $>$ $ T_r))$ \\ $\iff SHUTDOWN$}    \\
      \hline
      TempSys & Leaf  &  \makecell{Guarantee $g_t$: \\  $temp$ $>$ $ T_t \iff SHUTDOWN$}   \\
      \hline
      PressureSys & Leaf  &  \makecell{Guarantee $g_p$: \\ $pressure$ $>$ $ T_p \iff SHUTDOWN$}    \\
	\hline
      RadiationSys & Leaf  &  \makecell{Guarantee $g_r$: \\ $radiation$ $>$ $ T_r \iff SHUTDOWN$}   \\
      \hline
    \end{tabular}}
  \end{center}

\begin{center}
%\resizebox{\textwidth}{!}{%
    \begin{tabular}{ | c | c | c |}
      \hline
      \thead{Component} & \thead{Layer of Architecture} & \thead{Faults}\\
      \hline
      Temp Sensors (3) & \makecell{Leaf \\ Components} & \makecell{$f_{t}$: fail low}  \\
      \hline
	Pressure Sensors (3) & \makecell{Leaf \\ Components} & \makecell{$f_{p}$: fail low}  \\
      \hline
	Radiation Sensors (3) & \makecell{Leaf \\ Components} & \makecell{$f_{r}$: fail low}  \\
      \hline
    \end{tabular}
  \end{center}

The first two steps of this process are performed top down in conjunction with JKind analysis over a program. Thus, the top layer is analyzed first, then the next and so on. Steps 3 and 4 proceed after all MIVCs for each layer have been generated. We walk through this sensor system example in this fashion. 

\textbf{Step 1a: Preprocessing top layer.} The preprocessing step inserts specific MIVC elements into the Lustre program. The MIVC elements are the model elements considered in the constraint system for a given property. The Safety Annex provides the means to define a fault over the output of a component. This fault is given an unassigned \emph{trigger} Boolean literal in Lustre. If the trigger literal is true, the output of the component is changed. If not, the output remains equivalent to the nominal output of this component~\cite{stewart2020safety, Stewart17:IMBSA}. This trigger in Lustre is called a \emph{fault activation literal}. The IVC elements required in order to perform this transformation are these fault activation literals as well as guarantees. The basic rules used to insert these additional literals into Lustre depend on the analysis layer of that is being formed in Lustre and are as follows. 
\begin{itemize}
\item Leaf layer of analysis: only fault activation literals are added.
\item Middle or top layers: guarantees are added and if a direct subcomponent is a leaf component of the architecture and faults are defined on its outputs, then these faults are also added.
\end{itemize}

At the top level, guarantees of the sensor subsystems are the IVC elements.
$$\boxed{g_t, g_p, g_r}$$

There are distinct constraint systems, one for each property being proved. In this system at the top layer, there is a single property $P$; this results in the following constraint system. 
$$\boxed{C = \{g_t, g_p, g_r, \neg P\}}$$

\textbf{Step 2a: Generate all MIVCs for the constraint system at the top layer.} In order to prove $P$, all three guarantees from the sensor subsystem level are required. 
$$\boxed{MIVC(P) = \{g_t, g_p, g_r\}}$$. 

\textbf{Step 1b: Preprocessing leaf layer.} Model elements for the IVC algorithm consideration are the faults for each sensor, for instance temperature sensors:
$$\boxed{f_{t1}, f_{t2}, f_{t3}}$$ 

The resulting constraint system for the temperature sensor subsystem layer is:
$$\boxed{C = \{\neg f_{t1}, \neg f_{t2}, \neg f_{t3}, \neg g_t\}}$$

\textbf{Step 2b: Generate all MIVCs for the constraint system at the leaf layer.} Due to the majority voting mechanism, the MIVCs show all possible pairs of faults restricted to \emph{false}. This means, if any combination of two faults do not occur, then the guarantees at the temperature sensor subsystem level are satisfied. 
$$\boxed{
	\begin{aligned}
		MIVC_1(g_t) = \{\neg f_{t1}, \neg f_{t2}\} \\
		MIVC_2(g_t) = \{\neg f_{t1}, \neg f_{t3}\} \\
		MIVC_3(g_t) = \{\neg f_{t2}, \neg f_{t3}\}
	\end{aligned}
}$$

At this point, all MIVCs have been successfully generated (which is a requirement of the following algorithms) and we can move on to the generation of minimal cut sets. 

\textbf{Step 3a: Generate MCSs using a hitting set algorithm at the top layer.} Our single MIVC in this case will reveal three associated MCSs. (Notice: $MCS_1 \cap MIVC(P) \neq \emptyset$, and same for $MCS_2$ and $MCS_3$, thus these are hitting sets.)
 $$\boxed{
	\begin{aligned}
		MCS_1(top) = \{g_t\} \\
		MCS_2(top) = \{g_p\} \\
		MCS_3(top) = \{g_r\}
	\end{aligned}
}$$

\textbf{Step 4a: Transform MCSs into MinCutSets at the top layer.} Given that only guarantees are found in the MCSs at this layer, recursion is used to find the faults that cause violation of these guarantees. Using this recursion on $MCS_1$, we show the process further. 

\textbf{Step 3b: Generate MCSs using a hitting set algorithm at the leaf layer.} In step 2b, we found all MIVCs for the contract $g_t$ and send these to the hitting set algorithm. The resulting MCSs are: 
 $$\boxed{
	\begin{aligned}
		MCS_1(leaf) = \{\neg f_{t1}, \neg f_{t2}\} \\
		MCS_2(leaf) = \{\neg f_{t1}, \neg f_{t3}\} \\
		MCS_3(leaf) = \{\neg f_{t2}, \neg f_{t3}\}
	\end{aligned}
}$$

Only constrained faults are found in these MCSs, so we simply remove those constraints and have found the MinCutSets for the contracts $g_t$. These are returned and replace this contract in the top layer $MCS_1(top)$. Here is the end result for $MCS_1(top)$; this can be understood as three of the total minimal cut sets for $P$. 

\begin{equation*}
MCS_1(top) \rightarrow \left\{ \,
\begin{IEEEeqnarraybox}[][c]{l?s}
\IEEEstrut
MinCutSet_1(P) = \{f_{t1}, f_{t2}\}, \\
MinCutSet_2(P) = \{f_{t1},  f_{t3}\}, \\
MinCutSet_3(P) = \{f_{t2}, f_{t3}\}
\IEEEstrut
\end{IEEEeqnarraybox}
\right.
\end{equation*}

After all replacements have been made, we are left with all minimal cut sets for the property of interest ($P$ in this example). 

\end{comment}










%\section{Formalization of the Method}
\label{sec:theory}

\begin{comment}
The main idea that we present in this research is as follows. The MIVCs are MUSs (Minimal Unsatisfiable Subsets) of a constraint system that normally consists of the assumptions and guarantees of system components and the negation of the safety property of interest. The MCSs (Minimal Correction Sets), the sets that contain the minimal ``correction" of the infeasible constraint system, can then be obtained from all MUSs. Here is the key idea that we present: if the constraint system is defined to take into account fault activation literals as well as component contracts, these MCSs and the information they contain can be transformed into MinCutSets. A small example will illustrate this point nicely. 

Let a constraint system $C$ consist of guarantees $g_1$, $g_2$, fault activation literals constrained to $false$, $\neg f_1$, $\neg f_2$, and the safety property $P$ also constrained to $false$, $\neg P$. 
\begin{equation}
    C = \{\neg f_1, \neg f_2, g_1, g_2, \neg P\}
\end{equation}

Now, since we assume that the nominal model proves, it is no surprise that $C$ is UNSAT. (The guarantees are valid, no faults occur, and we want to prove the negation of the safety property.) The MIVC algorithm returns the minimal unsatisfiable subsets, say $MIVC_1 = \{\neg f_1, g_2\}$ and $MIVC_2 = \{\neg f_2\}$. Let us focus on $MIVC_1$. This can be understood in two ways: (1) $MIVC_1$ is a proof core: if $f_1$ does not occur and $g_2$ holds, then the property $P$ can be proven, or (2) $MIVC_1$ is a minimal unsatisfiable subset: it cannot be the case that $f_1$ does not occur and the guarantee $g_2$ holds. 

Next we look at the MCSs for this example:
\begin{center}
    $MCS_1 = \{\neg f_1, \neg f_2\}$,
    $MCS_2 = \{\neg f_2, g_2\}$
\end{center}

This means that $C \setminus MCS_1$ is SAT. So by removing those constraints from $C$, we get a SAT constraint system: $C \setminus MCS_1 = \{f_1, f_2, g_1, g_2, \neg P\}$. When both faults $f_1$ and $f_2$ are active, $\neg P$ can be proven; i.e., this is a cut set for $\neg P$. The minimality of the MCS gives the minimality of the cut set. 
\end{comment}









Throughout the remainder of this section, we provide the formal background necessary to show how this approach works.

In the case of a nominal model augmented with faults, a constraint system can be defined as follows. Let $F$ be the set of all fault activation literals defined in the system and $G$ be the set of component contracts (guarantees of component output behavior). 

\begin{definition}Let $C = \{C_1,C_2,...,C_n\}$ be a constraint system such that for $i \in \{1,...,n\}$, $C_i$ has the following constraints for any $f_j \in F$ and $g_k \in G$ with regard to the top level property $P$: 
\begin{center}
$C_i \in \left\{ \begin{array}{ll}
	f_j :&  false\\
	g_k :& true\\
	P :& false\\
\end{array}\right.$	
\end{center}
\label{def:constraintsystem}
\end{definition}

\begin{comment}
The \aivcalg algorithm collects all minimal unsatisfiable subsets of a given transition system in terms of the \textit{negation} of the top level property~\cite{Ghassabani2017EfficientGO,bendik2018online}. Assuming that the nominal model proves (no faults are active), it is not surprising that the guarantees and the negation of the safety property is UNSAT. The MUSs are the minimal explanation of the infeasibility of this constraint system; equivalently, these are the minimal sets of model elements necessary for proof of the safety property.

We utilize this algorithm by providing not only component contracts as model elements, but also fault activation literals constrained to \textit{false}, i.e., the faults are inactive. Thus the resulting MIVCs (MUSs) will contain the required contracts and constrained fault activation literals necessary to prove the safety property. 

Because of the duality between MUSs and MCSs, all MCSs can be obtained by finding the hitting sets of all MUSs. The MCS can be seen to correct the infeasibility of the constraint system and provides the minimal such correction. By removing the constraints from $C$ that are found in any MCS, $C$ becomes satisfiable. In terms of the constraint system that includes fault activation literals, by \textit{activating} the faults in the MCS and \textit{violating} the contracts in the MCS, we can prove the \textit{negation} of the property $P$. This is the precise information we require to find the minimal cut sets of a system. If the contracts in the MCS are replaced with the faults that cause its violation, the MCS is transformed into a MinCutSet. A high level summary of the steps of this transformation are shown in Figure~\ref{fig:trans}. 

\begin{figure*}[h!]
	%\vspace{-0.1in}
	\begin{center}
		\includegraphics[width=0.8\textwidth]{images/highLevelIdea.PNG}
	\end{center}
	\caption{Steps of the Transformation Process}
	\label{fig:trans}
\end{figure*}

\danielle{End of copy/paste.}

\end{comment}
The theory regarding MIVCs and the duality of MUSs with MCSs has been established~\cite{GhassabaniGW16,Ghassabani2017EfficientGO,liffiton2016fast}. The MCSs are transformed into Minimal Cut Sets according to the following theoretical results. We assume that the nominal model proves a safety property $P$. Then since the nominal model consists of all its component contracts, $G$, and all faults constrained to false, $F$, we can say that $F \cup G$ satisfies $P$ and the constraint system $C = F \cup G \cup \{\neg P\}$ is UNSAT for disjoint sets $F$ and $G$.

Let the set $\overline{MCS}$ be an MCS with all constraints removed.

\begin{lemma}
$\overline{MCS}$ satisfies $\neg P$.

%If the only elements of a MCS are constrained faults, these unconstrained faults are a minimal cut set.
\begin{proof}
Let $\overline{MCS}$ consist of the elements of MCS with constraints removed and let $C = E \cup \{\neg P\}$ for all model elements $E$ (i.e., $E = F \cup G$). Since $C$ is UNSAT, clearly $E$ does not satisfy $\neg P$. \\

Since $MCS \subseteq E$, we can further decompose $E$. Let the set $L_E$ be all elements of $E$ that are not found in $MCS$; the leftover elements. Then $E = L_E \cup MCS$ for disjoint sets $L_E$ and $MCS$. \\

It follows that since $E$ does not satisfy $\neg P$, we know that neither $L_E$ nor $MCS$ satisfies $\neg P$.\\ 

But we also know by the definition of $MCS$ that $C \setminus MCS$ satisfies $\neg P$ (i.e., removing all constraints from the elements in $C$ that are found in the $MCS$ produces a satisfiable constraint system). \\

Let $\overline{MCS}$ be the MCS with all constraints removed. Then $C \setminus MCS$ can be represented as: $C = L_E \cup \overline{MCS} \cup \{\neg P\}$ which is satisfiable. \\

Since $L_E$ does not satisfy $\neg P$, it must be the case that $\overline{MCS}$ does satisfy $\neg P$.

\end{proof}
\label{lem:minCorrSet1}
\end{lemma}
\vspace{-2em}

In order to transform the MCS into a minimal cut set, any contracts found in the MCS must be replaced with faults that cause their violation. We show that making this replacement still satisfies $\neg P$.

\begin{lemma}
	Replacement of a contract $g \in \overline{MCS}$ with the faults that cause its violation satisfies $\neg P$.
	\begin{proof}
	Assume that there exists a $g \in G$ where $g \in MCS$. Given the assumption that $G \cup \{P\}$ is SAT (i.e., the nominal model proves), $\neg g$ can only occur by activation of faults. For the set of all faults in the system, $F$, let $F_G \subseteq F$ such that $F_G$ satisfies $\neg g$ and let $F_G$ be a minimal such set (i.e. $F_G$ is a minimal cut set for $g$). Replace $g \in \overline{MCS}$ with $F_G$; call this new set $MCS_{repl}$. By the assumption that $F_G$ satisfies $\neg g$ and lemma 1, the result is immediate.
	\end{proof}
	\label{lem:corrSet}
\end{lemma}
\vspace{-2em}

\begin{lemma}
$MCS_{repl}$ is minimal.
\begin{proof}
The minimality follows directly from the definition of MCS. 

\end{proof}
\label{lem:minCorrSet}
\end{lemma}
\vspace{-2em}
Once all replacements have been made, the set $MCS_{repl}$ contains only faults. Since $MCS_{repl}$ satisfies $\neg P$, these unconstrained faults are a cut set for $\neg P$. Thus, iterative replacement of each contract in an MCS with a minimal cut set of that contract and removal of all constraints of the elements in MCS results in minimal cut set. 

\begin{theorem}
A minimal correction set can be transformed into a minimal cut set.
\begin{proof}
Result follows directly from lemmas 1-3.
\end{proof}
\end{theorem}

%For more information on the implementation of these theoretical results, see section~\ref{sec:algs}.






\section{Formalization}
\label{sec:formalization}
Given an initial state $I$ and a transition relation $T$ consisting of conjunctive constraints as defined in section~\ref{sec:prelim}. The nominal guarantees of the system, $G$, consist of conjunctive constraints $g \in G$. Given no faults (i.e., nominal system), each $g$ is one of the transition constraints $T_i$ where:

\begin{gather}
T_n = g_1 \land  g_2 \land \cdots \land g_n
\label{eq:Tn}
\end{gather}

We consider an arbitrary layer of analysis of the architecture and assume the property holds of the nominal relation $(I,T_n) \vdash P$. Given that our focus is on safety analysis in the presence of faults, let the set of all faults in the system be  denoted as $F$. A fault $f \in F$ is a deviation from the normal constraint imposed by a guarantee. Without loss of generality, we associate a single fault and an associated fault probability with a guarantee. Each fault $f_i$ is associated with an \emph{activation literal}, $\mathit{af}_i$, that determines whether the fault is active or inactive. %Any ``faults" in a mid-layer are simply violated guarantees, or deviations from normal behavior. 

%The faults in the safety annex are defined on leaf level components. Thus, for the lowest analysis layer, we must take into consideration faults and the guarantees their activation may violate. A fault $f \in F$ is a deviation from the normal constraint imposed by a guarantee. For the purposes of this paper, each guarantee at the leaf layer of analysis has an associated fault. 

To consider the system under the presence of faults, consider a set $GF$ of modified guarantees in the presence of faults and let a mapping be defined from activation literals $\mathit{af}_i \in AF$ to these modified guarantees $\mathit{gf}_i \in GF$. 
\begin{center}
$\sigma : AF \rightarrow GF$ \\
$\mathit{gf}_i = \sigma(\mathit{af}_i) =$ \textit{if} $\mathit{af}_i$ \textit{then} $f_i$ \textit{else} $g_i$\\
\label{eq:sigma}
\end{center}

The transition system is composed of the set of modified guarantees $GF$ and a set of conjunctions assigning each of the activation literals $\mathit{af}_i \in AF$ to false: 

\begin{gather}
T = \mathit{gf}_1 \land \mathit{gf}_2 \land \cdots \land \mathit{gf}_n \land \neg \mathit{af}_1 \land \neg \mathit{af}_2 \land \cdots \land \neg \mathit{af}_n
\label{eq:T}
\end{gather}

\begin{theorem} If $(I,T_n) \vdash P$ for $T_n$ defined in equation~\ref{eq:Tn}, then $(I,T) \vdash P$ for $T$ defined in equation~\ref{eq:T}.
\begin{proof}
By application of successive evaluations of $\sigma$ on each constrained activation literal $\neg \mathit{af}_i$ and the weakening of the antecedent by introduction of the activation literals, the result is immediate.
\end{proof}
\end{theorem}

Consider the elements of $T$ as a set $GF \cup AF$, where $GF$ are the potentially faulty guarantees and $AF$ consists of the activation literals that determine whether a guarantee is faulty. This is a set that is considered by an SMT-solver for satisfiability during the $k$-induction procedures. The posited problem is thus: $GF \land AF \land \neg P$ for the safety property in question. Recall, if this is an \emph{unsatisfiable} constraint system, then $P$ is provable given these constraints. On the other hand, if it is \emph{satisfiable}, then we know that given the constraints in $GF$ and $AF$, $P$ is not provable. These satisfiability constraints contain the information we wish to find. 

Let us view this in terms of the PWR system example and focus on the temperature sensor subsystem. The safety property to be proved is $G_t$, the supporting guarantees are found in each of the three temperature sensors, $g_{ti}$. Faults $f_{ti}$ are defined for each sensor. The transition system is: 
\begin{gather*}
T = \mathit{gf}_{t1} \land \mathit{gf}_{t2} \land \mathit{gf}_{t3}  \land \neg \mathit{af}_{t1} \land \neg \mathit{af}_{t2} \land \neg \mathit{af}_{t3}
\end{gather*}

The MIVCs for this subsystem layer correspond to all pairwise combinations of constrained activation literals. Intuitively, if any two sensor faults do {\em not} occur, then two of the three sensor guarantees are not violated and the system responds appropriately to high temperature; therefore, $G_t$ is provable. 

The MCSs for this subsystem layer happen to also correspond to all pairwise combinations of constrained activation literals. If any two sensor faults {\em do} occur, then two of the three sensor guarantees will be violated and the system does not respond to high temperature as required. This would result in the inability to prove $G_t$. (Note: it is not always the case that the MCSs are the same as the MIVCs -- in this case it is due to majority voting on three sensors.)

\subsection{Transforming MCS into Minimal Cut Set}
The MCSs contain the information needed to find minimal cut sets, but their elements consist of constrained activation literals and/or guarantees. The link between the activation literals, faults, and guarantees is defined through $\sigma$ mapping (equation~\ref{eq:sigma}); $\sigma$ must be applied to each element in an MCS to map back to the associated fault. Without loss of generality, let $MCS = \{\mathit{af}_1, \cdots, \mathit{af}_m\}$. Let $\sigma (MCS) = \{\sigma (\neg \mathit{af}_{1}), \cdots, \sigma (\mathit{af}_{m})\}$ be a mapping where MCS is a minimal correction set with regard to some property $G$ and $MCS  \subseteq AF$. 

\begin{lemma} $\sigma (MCS)$ is a cut set of $G$. 
\begin{proof}
Assume towards contradiction that $\sigma (MCS)$ is not a cut set of $G$. Then $\mathit{gf}_1 \land \cdots \land \mathit{gf}_n \land \mathit{af}_1 \cdots \land \mathit{af}_m \land \neg \mathit{af}_{k+1} \land \neg \mathit{af}_n \land \neg G$ is unsatisfiable. Thus, the \emph{true} activation literals do not affect the provability of $G$. This contradicts $C \setminus MCS$ is satisfiable. 
\label{lemma:cut}
\end{proof}
\end{lemma}

\begin{lemma} $\sigma(MCS)$ is minimal.
\begin{proof}
Assume toward contradiction that $\sigma(MCS)$ is not minimal with regard to $G$. Then there exists $S \subset MCS$ such that $\sigma(S)$ is a minimal cut set of $G$. This implies that the corresponding constraint system $C \setminus S$ is satisfiable. This contradicts the minimality of MCS.
\label{lemma:min}
\end{proof}
\end{lemma}

Minimal cut sets generated by monolithic analysis look only at explicitly defined faults throughout the architecture and attempt through various techniques to find the minimal violating set for a particular property. In this approach, explicit faults are analyzed as well as supporting guarantees. We view violated guarantees as deviations from nominal behavior and refer to them as ``faulty". Thus, this approach provides an overapproximation of the minimal cut sets compared to a monolithic approach. Let $\mathit{MonoCuts}$ be the set of all minimal cut sets using a monolithic approach and let $\mathit{CompCuts}$ be the set of minimal cut sets using the above approach. 

\begin{theorem} $\mathit{MonoCuts} \subseteq \mathit{CompCuts}$.
\begin{proof}
Let $M \in \mathit{MonoCuts}$ where $M$ is a mimimal cut set for safety property $P$. Then all $f_i \in M$, if active simultaneously, violate the property $P$. A direct translation of this system to a $\sigma$ form constraint system gives: $g_1  \land \cdots \land g_n  \land \neg \mathit{af}_{1} \land \cdots \land \neg \mathit{af}_{n} \land \neg P$. 

Without loss of generality, let $M = \{f_1, \dots, f_k\}$. Then we know that $g_1  \land \cdots \land g_n  \land \mathit{af}_1 \land \cdots \land \mathit{af}_k \land \neg \mathit{af}_{k+1} \land \cdots \land \neg \mathit{af}_{n} \land \neg P$ is satisfiable. Then $\{\mathit{af}_1, \dots,\mathit{af}_k\}$ is a correction set for the system and can be mapped by $\sigma$ to a minimal cut set by Lemmas \ref{lemma:cut} - \ref{lemma:min}.
\end{proof}
\end{theorem}
\danielle{This isn't quite right... Monolithic takes *all* guarantees in the model in the proof where as compositional takes only that layer. I do not make the distinction in this proof. Will consider this and rewrite.}





\begin{comment}
\danielle{Seems abrupt here - need lead in to describe WHY the cut sets are pairwise faults.} In terms of the PWR example, the minimal cut sets for the temperature subsystem property $G_t$ consist of all pairwise faults on the temperature sensors; if any two faults occur on the sensors at the same time, we violate the temperature subsystem guarantee. 

Once these lower level minimal cut sets are generated, it is a matter of simple set replacement to find the higher level minimal cut sets. This can be easily seen in our example. An MCS at the top level has the element $G_t$. We systematically replace the contract with the faults that cause their violation. This results in three distinct minimal cut sets for $P$ from the temperature subsystem: $\{f_{t1}, f_{t2}\}, \{f_{t1}, f_{t3}, \{f_{t2}, f_{t3}$. All minimal cut sets for $P$ are given as similar pairwise combinations from each subsystem and total 9 for the entire system.

\danielle{Seems I need a theorem to round it out: that replacement will give min cut sets of safety property. Will think about how to formulate this.}

\end{comment}













\begin{comment}
Given an initial state $I$ and a transition relation $T$ consisting of conjunctive constraints as defined in section~\ref{sec:prelim}. The nominal guarantees of the system, $G$, consist of conjunctive constraints $g \in G$. Given no faults, each $g$ is one of the transition constraints $T_i$ where:

\begin{gather}
T_n = g_1 \land  g_2 \land \cdots \land g_n
\label{eq:Tn}
\end{gather}

We assume the property holds of the nominal relation $(I,T_n) \vdash P$. 

Given that our focus is on safety analysis in the presence of faults, let the faults in the system be the set $F$. A fault $f \in F$ is a deviation from the normal constraint imposed by a guarantee. For the purposes of this paper, each guarantee has an associated fault. Without loss of generality, we associate a single fault and an associated fault probability with a guarantee. Each fault $f_i$ is associated with an \emph{activation literal}, $af_i$, that determines whether the fault is active or inactive. 

To consider the system under the presence of faults, consider a set $GF$ of modified guarantees in the presence of faults and let a mapping be defined from activation literals $af_i \in AF$ to these modified guarantees $gf_i \in GF$. 
\begin{center}
$\sigma : AF \rightarrow GF$ \\
$gf_i = \sigma(af_i) =$ if $af_i$ then $f_i$ else $g_i$
\end{center}

The transition system is composed of the set of modified guarantees $GF$ and a set of conjunctions assigning each of the activation literals $af_i \in AF$ to false: 

\begin{gather}
T = gf_1 \land gf_2 \land \cdots \land gf_n \land \neg af_1 \land \neg af_2 \land \cdots \land \neg af_n
\label{eq:T}
\end{gather}

\begin{lemma} If $(I,T_n) \vdash P$ for $T_n$ defined in equation~\ref{eq:Tn}, then $(I,T) \vdash P$ for $T$ defined in equation~\ref{eq:T}.
\begin{proof}
By application of successive evaluations of $\sigma$ on each constrained activation literal $\neg af_i$, the result is immediate.
\end{proof}
\end{lemma}

Consider the elements of $T$ as a set $GF \cup AF$, where $GF$ are the potentially faulty guarantees and $AF$ consists of the activation literals that determine whether a guarantee is faulty. This is a set that is considered by a SAT-solver for satisfiability during the $k$-induction procedures. The posited problem is thus: $GF \land AF \land \neg P$ for the safety property in question. Recall, if this is an \emph{unsatisfiable} constraint system, then $(I,T) \vdash P$. On the other hand, if it is \emph{satisfiable}, then we know that given the constraints in $GF$ and $AF$, $P$ is not provable. These are the exact constraints we wish to find. 

\subsection{Transform the MIVCs into Minimal Cut Sets}
The \aivcalg algorithm collects all {\em minimal unsatisfiable subsets} (MUSs) of a given transition system in terms of the \textit{negation} of the top level property~\cite{Ghassabani2017EfficientGO,bendik2018online}. Formally, an MUS of a constraint system $C$ is a set $M \subseteq C$ such that $M$ is unsatisfiable and $\forall c \in M$ : $M \setminus \{c\}$ is satisfiable. The MUSs are the minimal explanation of the infeasibility of this constraint system; equivalently, these are the minimal sets of model elements necessary for proof of the safety property.

Returning to our running example, this can be illustrated by the following. Given the constraint system $C = \{g_p, g_t, g_r, \neg P\}$, a minimal explanation of the infeasability of this system is the set $\{g_p, g_t, g_r,\}$. If all three guarantees hold, then $P$ is provable. 

A related set is a {\em minimal correction set} (MCS); a MCS $M$ of a constraint system $C$ is a subset $M\subseteq C$ such that $C \setminus M$ is satisfiable and $\forall S \subset M$ : $C \setminus S$ is unsatisfiable. A MCS can be seen to ``correct'' the infeasability of the constraint system by the removal from $C$ the constraints found in an MCS.

In the case of an UNSAT system, we may ask: what will correct this unsatisfiability? Returning to the PWR example, we can find the MCSs of the constraint system: $MCS_1 = \{g_t\}$, $MCS_2 = \{g_p\}$, $MCS_3 = \{g_r\}$. If any single guarantee is violated, a shut down from that subsystem will not get sent when it should and the safety property $P$ will be violated. 

A duality exists between the MUSs of a constraint system and the MCSs as established by Reiter \cite{reiter1987theory}. This duality is defined in terms of \textit{Minimal Hitting Sets} (\textit{MHS}). A hitting set of a collection of sets $A$ is a set $H$ such that every set in $A$ is ``hit'' by $H$; $H$ contains at least one element from every set in $A$. Every MUS of a constraint system is a minimal hitting set of the system's MCSs, and likewise every MCS is a minimal hitting set of the system's MUSs~\cite{liffiton2016fast, reiter1987theory, de1987diagnosing}.

For the PWR top level constraint system, it can be seen that each of the MCSs intersected with the MUS is nonempty. 

Since we are interested in sets of active faults that cause violation of the safety property, we turn our attention to Minimal Cut Sets. A \textit{Minimal Cut Set} (MinCutSet) is a minimal collection of faults that lead to the violation of the safety property. Furthermore, any subset of a MinCutSet will not cause this property violation. %We define a minimal cut set consistently with much of the research in this field~\cite{fta:survey,historyFTA}.
In this running example, the critical guarantees were seen in the first (top) layer of compositional analysis; the violation of the guarantees found in the MCSs provided a minimal set of supporting contracts that contribute to a top level event (violation of a safety property). We desire to compute the minimal cut sets and so a natural question is how to get from a minimal set of violated guarantees to the faults that cause their violation. 

\subsection{Compositionality}
Compositional analysis proceeds from the top layer downward through the architecture of the system model. Faults are defined on leaf level components. Any middle level analysis will provide the MCSs in terms of violated guarantees, which are not valid elements of a minimal cut set. The lowest level of analysis will contain the faults $f_i$ that violate guarantees $g_i$. 

For illustration, the PWR lowest level of analysis is performed per sensor subsystem. We focus on the temperature subsystem which has the guarantee under analysis $g_t = temp > T_t \iff SHUTDOWN$. Each leaf level component (temperature sensors) have associated fault $f_{ti}: $ fail low. Due to the majority voting mechanism, the MIVCs show all possible pairs of faults restricted to \emph{false}. This means, if any combination of two faults do not occur, then the guarantee at the temperature sensor subsystem level is satisfied. 
\begin{gather*}
		MIVC_1(g_t) = \{\neg f_{t1}, \neg f_{t2}\} \\
		MIVC_2(g_t) = \{\neg f_{t1}, \neg f_{t3}\} \\
		MIVC_3(g_t) = \{\neg f_{t2}, \neg f_{t3}\}
\end{gather*}


The hitting set algorithm produces the following MCSs; if any two faults occur, then the guarantee cannot be proven.
\begin{gather*}
		MCS_1(leaf) = \{\neg f_{t1}, \neg f_{t2}\} \\
		MCS_2(leaf) = \{\neg f_{t1}, \neg f_{t3}\} \\
		MCS_3(leaf) = \{\neg f_{t2}, \neg f_{t3}\}
\end{gather*}

And now we have the information required to determine the faults that cause the violation of guarantees at upper levels. 
\end{comment}





\section{Preliminaries}
\label{sec:prelim}
\newcommand{\bool}[0]{\mathit{bool}}
\newcommand{\reach}[0]{\mathit{R}}
\newcommand{\ite}[3]{\mathit{if}\ {#1}\ \mathit{then}\ {#2}\ \mathit{else}\ {#3}}


%The algorithms in this paper are implemented in the Safety Annex for the Architecture Analysis and Design Language (AADL) and require the Assume-Guarantee Reasoning Environment (AGREE)~\cite{NFM2012:CoGaMiWhLaLu} to annotate the AADL model in order to perform verification using the back-end model checker \jkind~\cite{2017arXiv171201222G}. 

\textbf{Architecture Analysis and Design Language}
We are using the Architectural Analysis and Design Language (AADL) to construct system architecture models of performance-critical, embedded, real-time systems~\cite{AADL_Standard}. %An AADL model describes a system in terms of a hierarchy of components and their interconnections, where each component can either represent a logical entity (e.g., application software functions, data) or a physical entity (e.g., buses, processors). 
Language annexes to AADL provide a rich set of modeling elements for various system design and analysis needs, and the language definition is sufficiently rigorous to support formal analysis tools that allow for early fault detection. Figure~\ref{fig:tempSensor} shows a temperature sensor component defined in AADL. It has a single input (\texttt{Env\_Temp}) and a single output (\texttt{High\_Temp\_Indicator}).  

\begin{figure*}[h!]
	\vspace{-2em}
	\begin{center}
		\includegraphics[width=0.9\textwidth]{images/tempSensoraadlannex.png}
	\end{center}
	\vspace{-2em}
	\caption{Temperature sensor in AADL with AGREE and safety annexes}
	\label{fig:tempSensor}
	\vspace{-2em}
\end{figure*}

%\textbf{Compositional Analysis} 
%One way to structure compositional verification is hierarchically: layers of the system architecture are analyzed independently and their composition demonstrates a system property of interest. Compositional verification partitions the formal analysis of a system architecture into verification tasks that correspond into the decomposition of the architecture~\cite{clarke1989compositional}.  A proof consists of demonstrating that the system property is provable given the contracts of its direct subcomponents and the system assumptions~\cite{cofer2012compositional,clarke1989compositional}. When compared to monolithic analysis (i.e., analysis of the flattened model composed of all components), the compositional approach allows the analysis to scale to much larger systems~\cite{NFM2012:CoGaMiWhLaLu,heckel1998compositional,cofer2012compositional}.

\textbf{Assume Guarantee Reasoning Environment}
The Assume Guarantee Reasoning Environment (AGREE) is a tool for formal analysis of behaviors in AADL models and supports compositional verification~\cite{NFM2012:CoGaMiWhLaLu}.  It is implemented as an AADL annex and is used to annotate AADL components with formal behavioral contracts. Each component's contracts includes assumptions and guarantees about the component's inputs and outputs respectively. AGREE translates an AADL model and the behavioral contracts into Lustre~\cite{Halbwachs91:IEEE} and then queries the \jkind model checker to conduct the back-end analysis~\cite{2017arXiv171201222G}. Figure~\ref{fig:tempSensor} shows the guarantee defined in the temperature sensor written in the AGREE annex.

%\textbf{JKind}
%JKind is an open-source industrial infinite-state inductive model checker for safety properties~\cite{2017arXiv171201222G}. Models and properties in JKind are specified in Lustre~\cite{Halbwachs91:IEEE}, a synchronous dataflow language, using the theories of linear real and integer arithmetic. JKind uses SMT-solvers to prove and falsify multiple properties in parallel.

\textbf{Safety Annex for AADL}
The Safety Annex for AADL provides the ability to reason about faults and faulty component behaviors in AADL models~\cite{Stewart17:IMBSA,stewart2020safety}. In the Safety Annex approach, AGREE is used to define the nominal behavior of system components, faults are introduced into the nominal model, and JKind is used to analyze the behavior of the system in the presence of faults. Faults describe deviations from the nominal behavior and are attached to the outputs of components in the system. An example of a safety annex fault definition on the temperature sensor is shown in Figure~\ref{fig:tempSensor}. The fault is associated with the \texttt{High\_Temp\_Indicator} output and has a probability of occurrence of $1.0 \times 10^{-5}$. The behavior of the output when the fault is active is to report low temperature. 

\subsection{Formal Background}
Given a state space $U$, a transition system $(I,T)$ consists of an
initial state predicate $I : U \to \bool$ and a transition step
predicate $T : U \times U \to \bool$.
We define the notion of
reachability for $(I, T)$ as the smallest predicate $\reach : U \to
\bool$ which satisfies the following formulas:
\begin{gather*}
  \forall u.~ I(u) \Rightarrow \reach(u) \\
  \forall u, u'.~ \reach(u) \land T(u, u') \Rightarrow \reach(u')
\end{gather*}
A safety property $P : U \to \bool$ is a state predicate. A safety
property $P$ holds on a transition system $(I, T)$ if it holds on all
reachable states, i.e., $\forall u.~ \reach(u) \Rightarrow P(u)$,
written as $\reach \Rightarrow P$ for short. When this is the case, we
write $(I, T)\vdash P$. We assume the transition relation has the structure of a top level conjuntion. Given $T(u, u') = T_1(u,u') \land \cdots T_n(u,u')$ we will write $T = \land_{i=1..n}$ for short. By further abuse of notation, $T$ is identified with the set of its top-level conjuncts. Thus, $T_i \in T$ means that $T_i$ is a top-level conjunct of $T$, and $S\subseteq T$ means all top level conjuncts of $S$ are top-level conjuncts of $T$. When a top-level conjunct $T_i$ is removed from $T$, we write $T \setminus \{T_i\}$

The idea behind finding an IVC for a given property P~\cite{GhassabaniGW16} is based on inductive proof methods used in SMT-based model checking, such as $\mathit{k}$-induction and IC3/PDR~\cite{een2011efficient, kahsai2012incremental, cook1971complexity}. Generally, an IVC computation technique aims to determine, for any subset $S \subseteq T$, whether $\mathit{P}$ is provable by $\mathit{S}$. Then, a minimal subset that satisfies $\mathit{P}$ is seen as a minimal proof explanation called a minimal inductive validity core. Ghassabani et al. demonstrate that the minimization process is as hard as model checking~\cite{Ghassabani2017EfficientGO}, so finding a minimal inductive validity core may not be possible for some model checking problems. 

\begin{definition}
Inductive Validity Core (IVC)~\cite{GhassabaniGW16}: $S \subseteq T$ for $(I, T) \vdash P$ is an Inductive Validity Core, denoted by $\mathit{IVC(P,S)}$, iff $\mathit{(I,S)} \vdash P$.
\end{definition}

\begin{definition}
Minimal Inductive Validity Core (MIVC)~\cite{Ghassabani2017EfficientGO}: $S \subseteq T$ is a minimal Inductive Validity Core, denoted by $\mathit{MIVC(P,S)}$, iff $\mathit{IVC(P,S)} \land \forall T_i \in S$. $(I, S \setminus \{T_i\}) \not \vdash P$.
\end{definition}

A $\mathit{k}$-induction model checker utilizes parallel SMT-solving engines at each induction step to glean information about the proof of a safety property. The transition formula is translated into clauses such that satisfiability is preserved~\cite{een2003temporal}. The translated system, consisting of the constrained formulas of the transition system and the negation of the property, is often called a \emph{constraint system}. The \aivcalg algorithm collects all {\em minimal unsatisfiable subsets} (MUSs) of a constraint system generated from a transition system at each induction step.~\cite{Ghassabani2017EfficientGO,bendik2018online}. 

\begin{definition}
A Minimal Unsatisfiable Subset (MUS)~\cite{reiter1987theory} $M$ of a constraint system $C$ is a set $M \subseteq C$ such that $M$ is unsatisfiable and $\forall c \in M$ : $M \setminus \{c\}$ is satisfiable.
\end{definition}
The MUSs are the minimal explanation of the infeasibility of this constraint system; equivalently, these are the minimal sets of model elements necessary for proof of the safety property.

Returning to our running example, this can be illustrated by the following. Given the constraint system $C = \{G_p, G_t, G_r, \neg P\}$, a minimal explanation of the infeasability of this system is the set $\{G_p, G_t, G_r,\}$. If all three guarantees hold, then $P$ is provable. 

A related set is a {\em minimal correction set}: 
\begin{definition}
A Minimal Correction Set (MCS)~\cite{reiter1987theory} $M$ of a constraint system $C$ is a subset $M\subseteq C$ such that $C \setminus M$ is satisfiable and $\forall M' \subset M$ : $C \setminus M'$ is unsatisfiable.
\end{definition}
A MCS can be seen to ``correct'' the infeasability of the constraint system by the removal from $C$ the constraints found in an MCS. In the case of an UNSAT system, we may ask: what will correct this unsatisfiability? Returning to the PWR example, we can find the MCSs of the constraint system $C$: $\mathit{MCS}_1 = \{G_t\}$, $\mathit{MCS}_2 = \{G_p\}$, $\mathit{MCS}_3 = \{G_r\}$. If any single guarantee is violated, a shut down from that subsystem will not get sent when it should and the safety property $P$ will be violated. 

A duality exists between the MUSs of a constraint system and the MCSs as established by Reiter \cite{reiter1987theory}. This duality is defined in terms of \textit{Minimal Hitting Sets} (\textit{MHS}). 

\begin{definition}
A hitting set of a collection of sets $A$ is a set $H$ such that every set in $A$ is ``hit'' by $H$; $H$ contains at least one element from every set in $A$. 
\end{definition}
Every MUS of a constraint system is a minimal hitting set of the system's MCSs, and likewise every MCS is a minimal hitting set of the system's MUSs. This is noted in previous work~\cite{liffiton2016fast, de1987diagnosing} and the proof of such is given by Reiter (Theorem 4.4 and Corollary 4.5)~\cite{reiter1987theory}. For the PWR top level constraint system, it can be seen that each of the MCSs intersected with the MUS is nonempty. This gives the minimal set of guarantees for which, if violated, will cause $P$ to be violated. 













\begin{comment}

For an arbitrary transition system $(I, T)$, computing reachability
can be very expensive or even impossible. Thus, we need a more
effective way of checking if a safety property $P$ is satisfied by the
system. The key idea is to over-approximate reachability. If we can
find an over-approximation that implies the property, then the
property must hold. Otherwise, the approximation needs to be refined.

A good first approximation for reachability is the property itself.
That is, we can check if the following formulas hold:
\begin{gather}
  \forall s.~ I(s) \Rightarrow P(s)
  \label{eq:1-ind-base} \\
  \forall s, s'.~ P(s) \land T(s, s') \Rightarrow P(s')
  \label{eq:1-ind-step}
\end{gather}
If both formulas hold then $P$ is {\em inductive} and holds over the
system. If (\ref{eq:1-ind-base}) fails to hold, then $P$ is violated
by an initial state of the system. If (\ref{eq:1-ind-step}) fails to
hold, then $P$ is too much of an over-approximation and needs to be
refined.

The JKind model checker used in this research uses {\em
  $k$-induction} which unrolls the property over $k$ steps of the
transition system. For example, 1-induction consists of formulas
(\ref{eq:1-ind-base}) and (\ref{eq:1-ind-step}) above, whereas
2-induction consists of the following formulas:
\begin{gather*}
\forall s.~ I(s) \Rightarrow P(s) \\
\forall s, s'.~ I(s) \land T(s, s') \Rightarrow P(s') \\
\forall s, s', s''.~ P(s) \land T(s, s') \land P(s') \land T(s',
  s'') \Rightarrow P(s'')
\end{gather*}
That is, there are two base step checks and one inductive step check.
In general, for an arbitrary $k$, $k$-induction consists of $k$
base step checks and one inductive step check as shown in
Figure~\ref{fig:k-induction} (the universal quantifiers on $s_i$ have
been elided for space). We say that a property is $k$-inductive if it
satisfies the $k$-induction constraints for the given value of $k$.
The hope is that the additional formulas in the antecedent of the
inductive step make it provable.

\begin{figure}
\begin{gather*}
I(s_0) \Rightarrow P(s_0) \\[-2pt]
%
\vdots \\[2pt]
%
I(s_0) \land T(s_0, s_1) \land \cdots \land T(s_{k-2}, s_{k-1})
\Rightarrow P(s_{k-1}) \\[2pt]
%
P(s_0) \land T(s_0, s_1) \land \cdots \land P(s_{k-1}) \land
T(s_{k-1}, s_k) \Rightarrow P(s_k)
\end{gather*}
\caption{$k$-induction formulas: $k$ base cases and one inductive
  step}
\label{fig:k-induction}
\end{figure}

In practice, inductive model checkers often use a combination of the
above techniques. Thus, a typical conclusion is of the form ``$P$ with
lemmas $L_1, \ldots, L_n$ is $k$-inductive''.

\danielle{Cut SAT section, but it seems I still need a bridge between the transition systems here and the constraint system further down. How is this?} Each induction step is sent to an SMT (Satisfiabilty Modulo Theory)-solver to check for \emph{satisfiability}, i.e. there exists a total truth assignment to a given formula that evaluates to true. If there does not exist such an assignment, the formula is considered \emph{unsatisfiable}. %A \emph{constraint system} is a term used to describe the formula with all constraints to the variables. 
A formula is translated into clauses such that satisfiability is preserved~\cite{een2003temporal} and the $k$-inductive model checker utilizes parallel SMT-solving engines to glean proof information at each inductive step. Expression of the base and induction steps of a temporal induction proof as SAT problems is straightforward (Figure~\ref{fig:k-induction}).

\begin{gather*}
I(s_0) \land T(s_0, s_1) \land \cdots \land T(s_{k-2}, s_{k-1})
\land \neg P(s_{k-1})
\end{gather*}

When proving correctness it is shown that the formulas are \emph{unsatisfiable},i.e., the property $P$ is provable.

\end{comment}

%\subsection{The SAT Problem}
%Boolean Satisfiability (SAT) solvers attempt to determine if there exists a total truth assignment to a given propositional formula, that evaluates to TRUE. Generally, a propositional formula is any combination of the disjunction and conjunction of literals (as an example, $a$ and $\neg a$ are literals). For a given unsatisfiable problem, solvers try to generate a proof of unsatisfiability; this is generally more useful than a proof of satisfiability. Such a proof is dependent on identifying a subset of clauses that make the problem unsatisfiable (UNSAT). SAT solvers in model checking work over a constraint system to determine satisfiability. A \textit{constraint system} $C$ is an ordered set of $n$ abstract constraints $\{C_1, C_2, ..., C_n\}$ over a set of variables. The constraint $C_i$ restricts the allowed assignments of these variables in some way~\cite{liffiton2016fast}. Given a constraint system, we require some method of determining, for any subset $S \subseteq C$, whether $S$ is \textit{satisfiable} (SAT) or \textit{unsatisfiable} (UNSAT). When a subset $S$ is SAT, this means that there exists an assignment allowed by all $C_i \in S$; when no such assignment exists, $S$ is considered UNSAT. 

%There are several ways of translating a propositional formula into clauses such that satisfiability is preserved~\cite{een2003temporal}. By performing this translation, $k$-inductive model checkers are able to utilize parallel SAT-solving engines to glean information about the proof of a safety property at each inductive step. Expression of the base and induction steps of a temporal induction proof as SAT problems is straightforward. As an example, we look at an arbitrary base case from Figure~\ref{fig:k-induction}.

%\begin{gather*}
%I(s_0) \land T(s_0, s_1) \land \cdots \land T(s_{k-2}, s_{k-1})
%\land \neg P(s_{k-1})
%\end{gather*}

%When proving correctness it is shown that the formulas are \emph{unsatisfiable}. If an $n^{th}$ inductive-step is unsatisfiable, that means following an $n$-step trace where the property holds, there exists no next state where it fails, i.e., the property $P$ is provable.

\section{Implementation}
\label{sec:impl}
%%%%%%%%%%%%%%%%%%%%%%%%%%%%%%%%%%%%%%%%%%%%%%%%%
%%%%%%%%%%%%%%%%%%%%            ALGORITHM DETAILS
In the formalism section, we viewed the problem from the perspective of an arbitary layer of the architectural analysis. This resulted in explicit faults at the leaf level and violated guarantees (``nondeterministic faults") at the middle/top layers. Each MCS generated at each level gives varied insight into the system. In this implementation section, we describe the mapping of the results from an arbitrary layer into the top/mid or leaf layers of analysis. Minimal cut sets traditionally contain explicitly defined faults as elements; to this end, we implemented a compositional mapping from explicit faults to the guarantees they violate. The end result are the minimal cut sets that contribute to a violation of the top level safety property. 

The \aivcalg algorithm requires \emph{IVC elements} to be explicitly added to the Lustre program; these elements are those of consideration for the MIVC analysis. In this implementation, the IVC elements are added differently depending on the layer. In the leaf architectural level, only explicitly defined faults are added to IVC elements. In middle or top layers, supporting guarantees are added. This is shown in Figure~\ref{fig:layers}. 

\begin{figure*}[h!]
	%\vspace{-2em}
	\begin{center}
		\includegraphics[width=0.9\textwidth]{images/twoLevels.PNG}
	\end{center}
	\vspace{-2em}
	\caption{Illustration of Two Layers of Analysis}
	\label{fig:layers}
	\vspace{-2em}
\end{figure*}

The figure shows an arbitrary architecture with two analysis layers: top and leaf. The top layer analysis adds $G$ as IVC element; the leaf layer analysis adds $f_1$ and $f_2$ as IVC elements. The first layer of analysis shows that $G$ supports the proof of $P$, thus is an MIVC. The second layer of analysis shows that \emph{given the model}, if both $f_1$ and $f_2$ are constrained to false, a proof is found for $G$. 

Each explicit fault defined in the safety annex is added to the Lustre program as described in safety annex implementation~\cite{Stewart17:IMBSA,stewart2020safety} and additionally the constrained faults are added as IVC elements for leaf layer analysis. A requirement of the hitting set algorithm is that to find \emph{all} MCSs, \emph{all} MUSs must be known. Ghassabani et al.~\cite{Ghassabani2017EfficientGO} showed that finding all MIVCs is as hard as model checking; and thus cannot always be found. It is a requirement of this analysis that all MIVCs are computed. Once this analysis is complete at each layer, a hitting set algorithm is used to generate the related MCSs~\cite{gainer2017minimal}. Depending on the layer of analysis, the MCSs contain either faulty (or violated) guarantees or explicitly defined faults.

Since minimal cut sets traditionally contain only explicit faults, we aimed to provide minimal cut sets in this format. A mapping is performed from the bottom up which replaces a guarantee with its corresponding cut set. At the leaf level, only constrained faults are in the MCS and thus can be saved as a mapping from the property under proof consideration to the set of \emph{unconstrained} faults that contribute to its violation. In Figure~\ref{fig:layers}, the mapping would be from $G$ to $\{f_1, f_2\}$ showing that the minimal cut set for $G$ is $\{f_1, f_2\}$. 

We continue in this fashion until all MCSs in that layer are processed. Then we move up; if a guarantee is in the MCS, we check for a mapping from that contract to its minimal cut set. If it exists, we perform iterative replacements (a guarantee can have multiple cut sets).  This mapping is shown in Algorithm~\ref{alg:transform_alg}.

\begin{algorithm}[h]
\DontPrintSemicolon
\SetKwFunction{Init}{Init}
\SetKwFunction{Transform}{Transform}
\SetKwProg{Fn}{Function}{:}{}

\Fn{\Init{$\mathit{List(MCS)}$, $P$}}{
	\For{all $\mathit{MCS} \in \mathit{List(MCS)}$}{	
		$\mathit{\overline{MCS}} \gets $ remove constraints from $\mathit{MCS}$\;
		$\mathit{List(\overline{MCS})} \gets$ append $\mathit{\overline{MCS}}$\; 
	}
	$\mathit{map} \gets \mathit{map(P \rightarrow \emptyset)}$\tcp*{from P to list of min cut sets}
	$\Transform(\mathit{List(\overline{MCS})}, P)$\;
}

\setcounter{AlgoLine}{0}
\Fn{\Transform{$\mathit{List(\overline{MCS})}$,$P$, $\mathit{map}$}}{
	\For{all $\mathit{\overline{MCS}} \in \mathit{List(\overline{MCS})}$}{	
		\eIf{$\neg \exists g \in \mathit{\overline{MCS}}$}{
			$\mathit{map(P \rightarrow }$ append $\mathit{\overline{MCS}})$ \;
		}{
			\For{all $\mathit{cut(g)}$}{
				$\mathit{new \overline{MCS}} \gets $ replace $g$ with $\mathit{cut(g)}$\;
				$\mathit{List(\overline{MCS})} \gets$ append $\mathit{new \overline{MCS}}$\;
			}			
		}
	}
}


%	\caption{Minimal Cut Set Generation Algorithm}
	\caption{Transform MCS into Minimal Cut Sets}
	\label{alg:transform_alg}
\end{algorithm}

The number of replacements $R$ that are made in this algorithm are constrained by the number of minimal cut sets there are for all $\alpha$ contracts within the initial MCS. We call the set of all minimal cut sets for a contract $g$: $\mathit{Cut(g)}$. The following formula defines an upper bound on the number of replacements. The validity of this statement follows directly from the general multiplicative combinatorial principle. 

\begin{lemma}
The number of replacements $R$ is bounded by the following formula:
\begin{gather}
\label{eq:bound}
  R \leq {\displaystyle \sum_{i=1}^{\alpha} }({\displaystyle \prod_{j=1}^{i} |\mathit{Cut(g_j)}|})  
\end{gather}
\begin{proof}
Assume there exists a $g_i \in \mathit{\overline{MCS}}$. The number of replacements made between $g_i$ and its minimal cut sets is at most $|\mathit{Cut(g_i)}|$. We iteratively perform this replacement for all $\alpha$ contracts in $\mathit{\overline{MCS}}$ and make, in the worst case, $|\mathit{Cut(g_1)}| \times |\mathit{Cut(g_2)}| \times \cdots \times |\mathit{Cut(g_\alpha)}|$ replacements.
\label{lemma:bound}
\end{proof}
\end{lemma}

It is also important to note that the cardinality of $\mathit{List(\overline{MCS})}$ is bounded, i.e. the algorithm terminates. Every new set that is generated through some replacement of a contract with its minimal cut set is added to $\mathit{List(\overline{MCS})}$ in order to continue the replacement process for all contracts in $\mathit{\overline{MCS}}$. 
\begin{theorem}
Algorithm~\ref{alg:transform_alg} terminates
\begin{proof}
No infinite sets are generated by the \aivcalg or minimal hitting set algorithms~\cite{Ghassabani2017EfficientGO,murakami2013efficient}; therefore, every MCS produced is finite. Thus, every minimal cut set of every contract is finite. Furthermore, a bound exists on the number of additional sets that are added to $List(I)$: \\
$|\mathit{List(\overline{MCS})}| \leq R$ by lemma~\ref{lemma:bound}.
\end{proof}
\end{theorem}

Given that the worst case size in terms of cut set cardinality and number of replacements can grow quickly, we implemented strategies to prune the size of the cut sets and hence the growth of these intermediate sets. 


\subsection{Pruning to Address Scalability}
The safety annex provides the capability to specify a type of verification in what is called a \textit{fault hypothesis statement}. These come in two forms: maximum number of faults or probabilistic analysis. Algorithm~\ref{alg:transform_alg} is the general approach, but the implementation changes slightly depending on which form of analysis is being performed. This pruning improves performance and diminishes the problem of combinatorial explosions in the size of minimal cut sets for larger models. 

\textbf{Max $N$ Analysis Pruning} This statement restricts the number of faults that can be independently active simultaneously and verification is run with this restriction present. For example, if a max 2 fault hypothesis is specified, two or fewer faults may be active at once. In terms of minimal cut sets, this statement restricts the cardinality of minimal cut sets generated to $N$.

If the number of faults in an intermediate set $\mathit{List(\overline{MCS})}$ exceeds the threshold $N$, any further replacement of remaining contracts in that intermediate set can never decrease the total number of faults in $\mathit{List(\overline{MCS})}$; therefore, this intermediate set is eliminated from consideration.

\textbf{Probabilistic Analysis Pruning} The second type of hypothesis statement restricts the cut sets by use of a probabilistic threshold. Any cut sets with combined probability higher than the given probabilistic threshold are removed from consideration. The allowable combinations of faults are calculated before the transformation algorithm begins; this allows for a pruning of intermediate sets during the transformation. If the faults within an intermediate set are not a subset of any allowable combination, that intermediate set is pruned from consideration and no further replacements are made. 








\begin{comment}

\setcounter{AlgoLine}{0}
\Fn{\FindMIVCs{}}{
	\While{$\mathit{Unexplored} \neq \emptyset$}{
		%$U_{max} \gets$ a maximal $U_{max} \in \mathit{Unexplored}$\;
		$U_{\mathit{max}} \gets $ a maximal set $\in \mathit{Unexplored}$\;
        \eIf{$\Solve(I,U_{\mathit{max}},P)$}{
			$U_{\mathit{IVC}} \gets \approx((I,U_{\mathit{max}}), P)$\;
			$\Shrink(U_{\mathit{IVC}})$\;
		}{
			$\mathit{Unexplored} \gets \mathit{Unexplored} \setminus \mathit{Sub}(U_{\mathit{max}})$\;			
		}
		\While{$\mathit{shrinkingQueue}$ is not empty}{
			$\mathit{U} \gets \Dequeue(\mathit{shrinkingQueue})$\;
			$\Shrink(\mathit{U})$\;
		}
	}
}


The transformation of MIVCs to MinCutSets can only be performed if \emph{all} MIVCs have been generated. It is a requirement of the minimal hitting set algorithm that all MUSs are used to find the MCSs~\cite{liffiton2016fast,gainer2017minimal,murakami2013efficient}. Thus, once all MIVCs have been found and the minimal hitting set algorithm has completed, the MinCutSet generation can begin. 

The MinCutSet generation algorithm begins with a list of MCSs specific to a property. These MCSs may contain a mixture of fault activation literals constrained to \textit{false} and subcomponent contracts constrained to \textit{true}. We remove all constraints from each MCS and call the resulting sets $I$, for \textit{Intermediate} set.  For each of those contracts in $I$, we check to see if we have previously obtained a MinCutSet for that contract. If so, replacement is performed. If not, we recursively call this algorithm to obtain the list of all MinCutSets associated with this subcomponent contract. At a certain point, there will be no more contracts in the set $I$ in which case we have a minimal cut set for the current property. The reason is because at the lowest levels of the system, the only model elements used in the constraint system analyzed by the \aivcalg algorithm are faults. Thus when the contracts at the lowest level are the safety properties for the \aivcalg algorithm, the MUSs contain only faults (likewise the MCSs). When this cut set is obtained for the lowest level properties, it is stored in a lookup table keyed by the given property. Algorithm~\ref{alg:generation_alg} describes this process.


\begin{algorithm}[h]
\SetKwFunction{FMain}{replace}
 \SetKwProg{Fn}{Function}{:}{}

	\Fn{\FMain{$P$}}{
		$List(I)$:= $List(MCS)$ for $P$ with all constraints removed \;
		\For{all $I \in List(I)$}{
			\eIf{there exists contracts $g \in I$}{
				\For{all constrained contracts $g \in I$}{
					\eIf{there exists $MinCutSets$ for $g$ in lookup table}{
						\For{all $minCut(g)$}{
							$I_{repl} = I$ \;
							$I_{repl} :=$ replace $g$ with $minCut(g)$ \;
							add $I_{repl}$ to $List(I)$ \;
						} %end for all cut sets of g
					}{
						replace($g$) \;
					} % end else if no cut sets in lookup table
				} % end for all constrained contracts in I
			}{
				add $I$ as $minCut(g)$ for $P$ \;
			} %end else if there exists contracts in I
		}%end for all I in list(I)
	}
%	\caption{Minimal Cut Set Generation Algorithm}
	\caption{MinCutSets Generation Algorithm}
	\label{alg:generation_alg}
\end{algorithm}

The number of replacements $R$ that are made in this algorithm are constrained by the number of minimal cut sets there are for all $\alpha$ contracts within the initial MCS. 

We call the set of all minimal cut sets for a contract $g$: $Cut(g)$. The following formula defines an upper bound on the number of replacements. The validity of this statement follows directly from the general multiplicative combinatorial principle. The number of replacements $R$ is bounded by the following formula:
\begin{equation}
\label{eq:bound}
  R \leq {\displaystyle \sum_{i=1}^{\alpha} }({\displaystyle \prod_{j=1}^{i} |Cut(g_j)|})  
\end{equation}


It is also important to note that the cardinality of $List(I)$ is bounded, i.e. the algorithm terminates. Every new $I$ that is generated through some replacement of a contract with its minimal cut set is added to $List(I)$ in order to continue the replacement process for all contracts in $I$. Adding to this set requires proof regarding termination.
\begin{theorem}
Algorithm~\ref{alg:generation_alg} terminates
\begin{proof}
No infinite sets are generated by the \aivcalg or minimal hitting set algorithms~\cite{Ghassabani2017EfficientGO,murakami2013efficient}; therefore, every MCS produced is finite. Thus, every $MinCutSet$ of every contract $g$ is finite. Furthermore, a bound exists on the number of additional intermediate sets $I$ that are added to $List(I)$: \\
$|List(I)| \leq R$ (Equation~\ref{eq:bound}).
\end{proof}
\end{theorem}

The reason for this upper bound is that for a contract $g_1$ in MCS, we make $|Cut(g_1)|$ replacements and add the resulting lists to $List(I)$. Then we move to the next contract $g_2$ in $I$. We must additionally make $|Cut(g_1)| \times |Cut(g_2)|$ replacements and add all of these resulting lists to $List(I)$, and so on throughout all contracts. Through the use of basic combinatorial principles, we end with the above formula for the upper bound on the number of additional intermediate sets.


\subsubsection{Pruning to Address Scalability}
The MinCutSets are filtered during this process based on a fault hypothesis given before analysis begins. The Safety Annex provides the capability to specify a type of verification in what is called a \textit{fault hypothesis statement}. These come in two forms: maximum number of faults or probabilistic analysis. Algorithm~\ref{alg:generation_alg} is the general approach, but the implementation changes slightly depending on which form of analysis is being performed. This pruning improves performance and diminishes the problem of combinatorial explosions in the size of minimal cut sets for larger models. \\

\textbf{Max $N$ Analysis Pruning} This statement restricts the number of faults that can be independently active simultaneously and verification is run with this restriction present. For example, if a max 2 fault hypothesis is specified, two or fewer faults may be active at once. In terms of minimal cut sets, this statement restricts the cardinality of minimal cut sets generated.

If the number of faults in an intermediate set $I$ exceeds the threshold $N$, any further replacement of remaining contracts in that intermediate set can never decrease the total number of faults in $I$; therefore, this intermediate set is eliminated from consideration.\\

\textbf{Probabilistic Analysis Pruning} The second type of hypothesis statement restricts the cut sets by use of a probabilistic threshold. Any cut sets with combined probability higher than the given probabilistic threshold are removed from consideration. The allowable combinations of faults are calculated before the transformation algorithm begins; this allows for a pruning of intermediate sets during the transformation. If the faults within an intermediate set are not a subset of any allowable combination, that intermediate set is pruned from consideration and no further replacements are made. 

\end{comment}



\section{Related Work}
\label{sec:related_work}
Formal model based systems engineering (MBSE) methods and tools now permit system level requirements to be specified and analyzed early in the development process~\cite{QFCS15:backes,CIMATTI2015333, NFM2012:CoGaMiWhLaLu, hilt2013:MuWhRaHe}. Design models from which aircraft systems are developed can be integrated into the safety analysis process to help guarantee accurate and consistent results. There are tools that currently support reasoning about faults in architecture description languages such as SysML and AADL. These tools include the AADL Error Model Annex, Version 2 (EMV2)~\cite{EMV2} and HiP-HOPS for EAST-ADL~\cite{CHEN201391}. These approaches primarily utilize \textit{qualitative} reasoning. Faults are enumerated and the propagations through system components are explicitly described. Given many possible faults, these propagation relationships increase in complexity and understandability. Interactions are easily overlooked by analysts and thus not explicitly described. This is also the case with tools like SAML that incorporate both \textit{qualitative} and \textit{quantitative} reasoning~\cite{Gudemann:2010:FQQ:1909626.1909813}.  

In earlier work, an approach to MBSA was demonstrated using the Simulink\textsuperscript{\textregistered} notation~\cite{Joshi05:SafeComp,Joshi05:Dasc}. In this approach, a behavioral model of system dynamics was used to reason about the effects of faults in the system. This approach allows an implicit and natural notion of fault propagation through the system. However, non-functional architectural properties were not captured as Simulink is not designed as an architecture description language. In our approach, we are applying \textit{quantitative} reasoning and implicit fault propagation to a more rich architecture language.  

There are other tools purpose-built for safety analysis, including AltaRica~\cite{PROSVIRNOVA2013127}, smartIFlow~\cite{info8010007} and xSAP~\cite{DBLP:conf/tacas/BittnerBCCGGMMZ16}. These notations are separate from the system development model. Other tools extend existing system models, such as HiP-HOPS~\cite{CHEN201391} and the AADL Error Model Annex, Version 2 (EMV2)~\cite{EMV2}. EMV2 uses enumeration of faults in each component and explicit propagation of faulty behavior to perform safety analysis. The required propagation relationships must be manually added to the system model and can become complex, leading to potential omissions and inconsistencies.

Formal verification tools based on model checking have been used to automate the generation of safety artifacts~\cite{symbAltaRica,10.1007/978-3-540-75596-8-13, DBLP:conf/tacas/BittnerBCCGGMMZ16}. This approach has limitations in terms of scalability and readability of the fault trees generated. Work has been done towards mitigating these limitations by the scalable generation of readable fault trees~\cite{10.1007/978-3-319-11936-6-7}.


\section{Conclusion and Future Work}
We have developed a way to leverage recent research in model checking techniques in order to generate minimal cut sets in a compositional fashion. Using the idea of Minimal Inductive Validity Cores (MIVCs), which are the minimal model elements necessary for a proof of a safety property, we are able to restate the safety property as a top level event and provide faults of components and their contracts as model elements to the \aivcalg algorithm which provides all minimal IVCs that pertain to this property. These are used to generate minimal cut sets. Future work includes leveraging the system information embedded in this approach to generate hierarchical fault trees as well as perform scalability studies that compare this approach with other non-compositional approaches to minimal cut set generation.  %For more details on the tool, models, and approach, see the technical report~\cite{SATechReport}. 
To access the algorithm implementation, Safety Annex users manual, or example models, see the repository~\cite{SAGithub}. 

\vspace{2 mm}
\noindent {\bf Acknowledgments.} This research was funded by NASA contract NNL16AB07T and the University of Minnesota College of Science and Engineering Graduate Fellowship.





%\vspace{-0.40cm}
\bibliographystyle{IEEEtran}

\bibliography{IEEEabrv,biblio}
%\vspace{-7.25cm}
% This ~ seems to fix an odd bibliography alignment issue


%\ifdefined\TECHREPORT
%\appendix
%
%\section{Appendix: Proof of Equivalence}
%\input{appendix}
%\fi

%\section{Appendix: GPCA CENTA Model}
%\label{appendix:gpcacenta}
%\begin{figure}[!ht]
%\begin{center}
%\includegraphics[scale=0.6]{images/sampled_pca.PNG} %[trim = 0 2 0 0, clip=true]{Comp}
%\caption{GPCA AGREE Properties modeled as a Timed Automata} \label{fig:samplepca}
%\end{center}
%\end{figure}

%\balancecolumns

\end{document} 