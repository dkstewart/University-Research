\section{Related Work}
\label{sec:related_work}
Minimal cut sets generated by monolithic analysis look at explicitly defined faults throughout the architecture and attempt through various techniques to find the minimal violating set for a particular property. We outline some of the common monolithic approaches to minimal cut set generation in this section.

The representation of Boolean formulae as Binary Decision Diagrams (BDDs) was first formalized in the mid 1980s~\cite{bryant1986graph} and was extended to the representation of fault trees not many years later~\cite{rauzy1993new}. After this formalization, the BDD approach to FTA provided a new approach to safety analysis. The model is constructed using a BDD, then a second BDD - usually slightly restructured - is used to encode minimal cut sets. Unfortunately, due to the structure of BDDs, the worst case is exponential in size in terms of the number of variables~\cite{bryant1986graph,rauzy1993new}. In industrial sized systems, this is not realistically useful. 

SAT based computation was introduced to address scalability problems in the BDD approach; initially it was used as a preprocessing step to simplify the decision diagram~\cite{bozzano2015safety}, but later was extended to allow for all minimal cut set processing and generation without the use of BDDs~\cite{bozzano2015efficient}. Since then, much research has focused on leveraging the power of model checking in the problems of safety assessment~\cite{bieber2002combination,schafer2003combining,bozzano2003improving,volk2017fast,bozzano2015efficient,stewart2020safety}. 

Bozzano et al. formulated a Bounded Model Checking (BMC) approach to the problem by successively approximating the cut set generation and computations to allow for an ``anytime approximation" in cases when the cut sets were simply too large and numerous to find~\cite{bozzano2015efficient}. These algorithms are implemented in xSAP~\cite{DBLP:conf/tacas/BittnerBCCGGMMZ16} and COMPASS~\cite{compass30toolset}. 

The model based safety assessment tool AltaRica 3.0~\cite{prosvirnova:tel-01119730} performs a series of processing to transform the model into a reachability graph and then compile to Boolean formula in order to compute the minimal cut sets. Other tools such as HiP-HOPS~\cite{papadopoulos2001model} have implemented algorithms that follow the failure propagations in the model and collect information about safety related dependencies and hazards. The Safety Analysis Modeling Language (SAML)~\cite{Gudemann:2010:FQQ:1909626.1909813} provides a safety specific modeling language that can be translated into a number of input languages for model checkers in order to provide model checking support for minimal cut set generation.

To our knowledge, a fully compositional approach to generating fault forests or minimal cut sets has not been introduced.



















