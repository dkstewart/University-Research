\section{Conclusion and Future Work}
We presented a formalism that defines the composition of fault forests by extending the transition system to allow for fault activation literals. This formalism is implemented by leveraging recent research in model checking techniques. Using the idea of minimal inductive validity cores (MIVCs), which are the minimal model elements necessary for a proof of a safety property, we are able to provide fault activation literals as model elements to the \aivcalg algorithm which provides all the MIVCs that pertain to this property. These are used to generate minimal cut sets. Future work includes leveraging the system information embedded in this approach to generate graphical hierarchical fault trees as well as perform scalability studies that compare this approach with other non-compositional approaches to minimal cut set generation.  %For more details on the tool, models, and approach, see the technical report~\cite{SATechReport}. 


\vspace{2 mm}
\noindent {\bf Acknowledgments.} This research was funded by NASA contract NNL16AB07T and the University of Minnesota College of Science and Engineering Graduate Fellowship.


