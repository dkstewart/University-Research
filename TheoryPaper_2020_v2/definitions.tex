\section{Definitions}
\label{sec:definitions}
The Boolean Satisfiability (\sat) problem attempts to determine if there exists a total truth assignment to a given propositional formula, that evaluates to TRUE. Generally, a propositional formula is any combination of the disjunction and conjunction of literals (as an example, $a$ and $\neg a$ are literals). \sat solvers in model checking work over a constraint system to determine satisfiability.

A constraint system $C$ is an ordered set of $n$ abstract constraints $\{C_1, C_2, ..., C_n\}$ over a set of variables. The constraint $C_i$ restricts the allowed assignments of these variables in some way~\cite{liffiton2016fast}. Given a constraint system, we require some method of determining, for any subset $S \subseteq C$, whether $S$ is \textit{satisfiable} (\sat) or \textit{unsatisfiable} (\unsat). When a subset $S$ is SAT, this means that there exists an assignment allowed by all $C_i \in S$; when no such assignment exists, $S$ is considered \unsat. Given a constraint system $C$, there are certain subsets of $C$ that are of interest in terms of satisfiability. Definitions 2-4 are taken from research by Liffiton et. al.~\cite{liffiton2016fast}. 

\begin{definition} : A Minimal Unsatisfiable Subset (MUS) $M$ of a constraint system $C$ is a subset $M \subseteq C$ such that $M$ is unsatisfiable and $\forall c \in M$ : $M \setminus \{c\}$ is satisfiable. 
\end{definition}
\noindent
An MUS is the minimal explanation of the constraint systems infeasability. 
\begin{definition} : A Minimal Correction Set (\textit{MCS}) $M$ of a constraint system $C$ is a subset $M\subseteq C$ such that $C \setminus M$ is satisfiable and $\forall S \subset M$ : $C \setminus S$ is unsatisfiable. 
\end{definition}
\noindent
A \textit{MCS} can be seen to ``correct'' the infeasability of the constraint system by the removal from $C$ the constraints in \textit{MCS}.

A duality exists between the \textit{MUS}s of a constraint system and the \textit{MCS}s as established by Reiter \cite{reiter1987theory}. This duality is defined in terms of \textit{Minimal Hitting Sets} (\textit{MHS}). A hitting set of a collection of sets $A$ is a set $H$ such that every set in $A$ is ``hit'' be $H$; $H$ contains at least one element from every set in $A$. Every \textit{MUS} of a constraint system is a minimal hitting set of the system's \textit{MCS}s, and likewise every \textit{MCS} is a minimal hitting set of the system's \textit{MUS}s~\cite{liffiton2016fast, reiter1987theory, de1987diagnosing}.
\begin{definition}: Given a collection of sets $K$, a hitting set for $K$ is a set $H \subseteq \cup_{S \in K} S$ such that $H \cap S \neq \emptyset$ for each $S  \in K$. A hitting set for $K$ is minimal if and only if no proper subset of it is a hitting set for $K$. 
\end{definition}
\noindent
Since we are interested in sets of active faults that cause violation of the safety property, we turn our attention to Minimal Cut Sets. 
\begin{definition}
A \textit{Minimal Cut Set} (\textit{MinCutSet}) is a minimal collection of faults that lead to the violation of the safety property. Furthermore, any subset of a \textit{MinCutSet} will not cause this property violation. 
\end{definition}
\noindent
We define a minimal cut set consistently with much of the research in this field~\cite{0f356f05e72f43018211b36f97c8854a,historyFTA}.\\

\subsection{Inductive Validity Cores}
Given a complex model, it is often useful to extract traceability information related to the proof, in other words, which portions of the model were necessary to construct the proof. An algorithm was introduced by Ghassabani, et. al to provide Minimal Inductive Validity Cores (\textit{MIVCs}) as a way to determine the model elements necessary for the inductive proof of a safety property for a sequential system~\cite{GhassabaniGW16}. Given a safety property, a model checker can be invoked in order to construct a proof of the property. The \textit{MIVC} generation algorithm can extract traceability information from that proof process and return a minimal set of the model elements required in order to prove the property. Later research extended this algorithm in order to produce all such minimal sets of \textit{MIVC} elements; this is the \aivcalg algorithm~\cite{Ghassabani2017EfficientGO,bendik2018online}.

This algorithm is of interest to us for the following reason. All \textit{MIVCs} are \textit{MUSs} of a constraint system that consists of the assumptions and contracts of system components and the negation of the safety property of interest. The set of all \textit{MIVCs} then contains the minimal explanation of the constraint systems infeasibility in terms of the negation of the safety property; hence, these are the minimal model elements necessary to proof the safety property.

%\subsection{Compositional Analysis} Compositional analysis of systems was introduced in order to address the scalability of model checking large software systems. Monolithic verification and compositional verification are two ways that mathematical verification of component properties can be performed. In monolithic analysis, the model is flattened and the top level properties are proved using only the leaf level contracts of the components. On the other hand, the analysis can be performed compositionally following the architecture hierarchy such that analysis at a higher level is based on the components at the next lower level. The idea is to partition the formal analysis of a system architecture into verification tasks that correspond into the decomposition of the architecture. A component contract is an assume-guarantee pair. Intuitively, the meaning of a pair is: if the assumption is true, then the component will ensure that the guarantee is true. The components of a system are organized hierarchically and each layer of the architecture is viewed a system. For any given layer, the proof consists of demonstrating that the system guarantee is provable given the guarantees of its direct subcomponents and the system assumptions. This proof is performed one layer at a time starting from the top level of the system. When compared to monolithic analysis (i.e., analysis of the flattened model composed of all components), the compositional approach allows the analysis to scale to much larger systems~\cite{NFM2012:CoGaMiWhLaLu}. 
