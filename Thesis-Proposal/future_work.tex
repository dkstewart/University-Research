\section{Future Work}
\label{sec:future_work}

Fault trees are commonly used in all major fields of safety engineering. Fault Tree Analysis (FTA) is a deductive technique where an undesired state called a Top Level Event (TLE) is specified and the system is analyzed for a possible sequence of basic events, usually system faults, that may cause the TLE to occur. A fault tree is a representation of such events which makes use of logical gates to depict the relationships between the TLE and the basic events~\cite{FTATechReport,Bozzano:2010:DSA:1951720}. Due to the importance of FTAs in safety engineering, it is important for the Safety Annex to automatically provide such artifacts. 

In AGREE, there are two types of analyses that can be performed on both the nominal and fault models of the system under test: monolithic and compositional. Monolithic verification uses the contracts found at the leaf level components of a system and uses these to prove the top level contracts. Compositional verification on the other hand utilizes a divide and conquer strategy. The requirements of a system can be decomposed and allocated to the subsystems. The goal is to establish at the system level a top level property. The component verification conditions establish that the assumptions of each component are implied by the system level assumptions and the properties of its sibling components~\cite{hilt2013:MuWhRaHe,QFCS15:backes,NFM2012:CoGaMiWhLaLu}. Since the safety analysis process works for both monolithic and compositional verification in slightly different ways, in order to address the fault tree generation problem, these methods of analysis must be discussed. 

In a monolithic analysis setting, we can compose the tree from what are called Minimal Cut Sets (MCS). An MCT can be viewed as the smallest combination of component failures that can cause the TLE to occur~\cite{Bozzano:2010:DSA:1951720}. MCTs are useful in fault tree generation because they represent simple explanations for the TLE. As an example, an MCT with only one basic event corresponds to a single point of failure of that system. Unfortunately, fault trees generated by this approach do not consider the architectural structure of the model, and can result in fault trees that are quite shallow but very wide. 

In a compositional analysis setting, decomposing the negation of a top level AGREE contract and mapping conjunctions and disjunctions to AND and OR gates provide a way to approximate an initial tree structure. However, to further develop the tree, it requires information from composing the results with the presence of faults. Compositional probabilistic analysis is a topic that needs further exploration in this research before fault trees can be generated from compositional safety analysis approaches. 

Furthermore, we would like to also find out what would be the desired structure of fault tree that is possible to obtain from our model (i.e., the shared architecture and safety model specified in AADL/AGREE/Safety Annex), and acceptable by safety engineers for certification purposes. 
