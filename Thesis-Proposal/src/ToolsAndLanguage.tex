\section{Tools and Modeling Language}

\subsection{Architecture Analysis and Design Language}
We are using the Architectural Analysis and Design Language (AADL) to construct system architecture models.  AADL is an SAE International standard language that provides a unifying framework for describing the system architecture for performance-critical, embedded, real-time systems~\cite{AADL_Standard,FeilerModelBasedEngineering2012}. From its conception, AADL has been designed for the design and construction of avionics systems.  Rather than being merely descriptive, AADL models can be made specific enough to support system-level code generation.  Thus, results from analyses conducted, including the new safety analysis proposed here, correspond to the system that will be built from the model.  
 
An AADL model describes a system in terms of a hierarchy of components and their interconnections, where each component can either represent a logical entity (e.g., application software functions, data) or a physical entity (e.g., buses, processors). An AADL model can be extended with language annexes to provide a richer set of modeling elements for various system design and analysis needs (e.g., performance-related characteristics, configuration settings, dynamic behaviors). The language definition is sufficiently rigorous to support formal analysis tools that allow for early phase error/fault detection.

\subsection{Assume Guarantee Reasoning Environment}
The Assume Guarantee Reasoning Environment (\agree) is a tool for formal analysis of behaviors in AADL models~\cite{NFM2012:CoGaMiWhLaLu}.  It is implemented as an AADL annex and annotates AADL components with formal behavioral contracts. Each component's contracts can include assumptions and guarantees about the component's inputs and outputs respectively, as well as predicates describing how the state of the component evolves over time.

\agree translates an AADL model and the behavioral contracts into Lustre~\cite{Halbwachs91:IEEE} and then queries a user-selected model checker to conduct the back-end analysis. The analysis can be performed compositionally or monolithically.

\textbf{Monolithic vs. Compositional Analysis:} Compositional analysis of systems was introduced in order to address the scalability of model checking large software systems~\cite{pnueli1985transition, heckel1998compositional, NFM2012:CoGaMiWhLaLu}. Monolithic verification and compositional verification are two ways that mathematical verification of component properties can be performed. In monolithic analysis, the model is flattened and the top level properties are proved using the contracts of all components. The analysis can alternatively be performed compositionally following the architecture hierarchy such that analysis at a higher level is based on the components at the next lower level and conducted layer by layer; the components of a system are organized hierarchically and each layer of the architecture is viewed a system. The idea is to partition the formal analysis of a system architecture into verification tasks that correspond into the decomposition of the architecture. 

A component contract in \agree is an assume-guarantee pair. Intuitively, the meaning of a pair is: if the assumption is true, then the component will ensure that the guarantee is true. For any given layer, the proof consists of demonstrating that the system guarantee is provable given the guarantees of its direct subcomponents and the system assumptions. This proof is performed one layer at a time starting from the top level of the system. When compared to monolithic analysis (i.e., analysis of the flattened model composed of all components), the compositional approach allows the analysis to scale to much larger systems~\cite{NFM2012:CoGaMiWhLaLu}. 

\subsection{Safety Annex for AADL}
The Safety Annex for AADL is a tool that provides the ability to reason about faults and faulty component behaviors in AADL models and has been developed throughout the course of this project~\cite{Stewart17:IMBSA,SATechReport, stewart2020safety, nasaFinalReport}. In the Safety Annex approach, formal assume-guarantee contracts are used to define the nominal behavior of system component and the nominal model is verified using AGREE. The Safety Annex weaves faults into the nominal model and analyzes the behavior of the system in the presence of faults. The tool supports behavioral specification of faults and their implicit propagation through behavioral relationships in the model and provides support to capture binding relationships between hardware and software components of the system.