\subsection{Definitions}
\label{sec:definitions}
Intuitively a constraint system contains the contracts that constrain component behavior and faults that are defined over these components. In the case of a nominal model augmented with faults, a constraint system is defined as follows. Let $F$ be the set of all fault activation literals defined in the model and $G$ be the set of all component contracts (guarantees). 

\begin{definition}A constraint system $C = \{C_1,C_2,...,C_n\}$ where for $i \in \{1,...,n\}$, $C_i$ has the following constraints for any $f_j \in F$ and $g_k \in G$ with regard to the top level property $P$: 
\begin{center}
$C_i \in \left\{ \begin{array}{ll}
	f_j :&  inactive\\
	g_k :& true\\
	P :& false\\
\end{array}\right.$	
\end{center}
\label{def:constraintsystem}
\end{definition}

Given a state space $S$, a transition system $(I,T)$ consists of the initial state predicate $I : S \rightarrow \{0,1\}$ and a transition step predicate $T : S \times S \rightarrow \{0,1\}$. Reachability for $(I,T)$ is defined as the smallest predicate $R : S \rightarrow \{0,1\}$ that satisfies the following formulas:
\begin{center}
$\forall s. I(s) \Rightarrow R(s)$\\
$\forall s, s' .  R \land T(s,s') \Rightarrow R(s')$\\
\end{center}
A safety property $\mathcal{P} : S \to \{0,1\}$ is a state predicate. A safety property $\mathcal{P}$ holds on a transition system $(I,T)$ if it holds on all reachable states. More formally, $\forall s . R(s) \Rightarrow \mathcal{P}(s)$. When this is the case, we write $(I,T) \vdash\mathcal{P}$~\cite{Ghassabani2017EfficientGO}. 

Given a transition system that satisfies a safety property $P$, it is possible to find which model elements are necessary for satisfying the safety property through the use of the \textit{All Minimal Inductive Validity Cores All-MIVCs} algorithms~\cite{Ghassabani2017EfficientGO,bendik2018online}. This algorithm collects all minimal unsatisfiable subsets of a given transition system in terms of the negation of the top level property. The minimal unsatisfiable subsets consist of component contracts constrained to \textit{true}. When the constraints on these model elements are removed from the constraint system $C$, this results in an UNSAT system. This can be seen as the minimal explanation of the constraint systems infeasibility. Recall that this constraint system is in terms of the \textit{negation} of the safety property. Thus, this algorithm provides all model elements required for the proof of the safety property. 

We utilize this algorithm by providing not only component contracts (constrained to \textit{true}) as model elements, but also fault activation literals constrained to \textit{false}. Thus the resulting MIVCs will contain the required contracts and constrained fault activation literals in order to prove the safety property. This information is used throughout this section to provide the underlying theory behind the generation of minimal cut sets from all MIVCs. 

Definitions 1-3 are taken from research by Liffiton et. al.~\cite{liffiton2016fast}. 

\begin{definition} : A Minimal Unsatisfiable Subset ($MUS$) of a constraint system $C$ is :\\
 $\{MUS \subseteq C$ $|$ $MUS$ is UNSAT and $\forall c \in MUS$: $MUS \setminus \{c\}$ is SAT$\}$. This is the minimal explaination of the constraint systems infeasability. 
\end{definition}

A closely related set is a \textit{minimal correction set} (MCS). The MCSs describe the minimal set of model elements for which if constraints are removed, the constraint system is satisfied. For constraint system $C$, this corresponds to which faults are not constrained to inactive (and are hence active) and violated contracts which lead to the violation of the safety property. In other words, the minimal set of active faults and/or violated properties that lead to the top level event.  

\begin{definition} : A Minimal Correction Set ($MCS$) of a constraint system $C$ is :\\
 $\{MCS \subseteq C$ $|$ $C \setminus MCS$ is SAT and $\forall S \subset MCS$ : $C \setminus S$ is UNSAT$\}$. A MCS can be seen to ``correct'' the infeasability of the constraint system.
\end{definition}

A duality exists between MUSs of a constraint system and MCSs as established by Reiter \cite{reiter1987theory}. This duality is defined in terms of \textit{minimal hitting sets}. A hitting set of a collection of sets $A$ is a set $H$ such that every set in $A$ is ``hit'' be $H$; $H$ contains at least one element from every set in $A$ \cite{liffiton2016fast}. The MCSs can be generated from the MUSs if all MUSs are known. Thus, the use of the All-MIVCs algorithm is required. 

\begin{definition}: Given a collection of sets $K$, a hitting set for $K$ is a set $H \subseteq \cup_{S \in K} S$ such that $H \cap S \neq \emptyset$ for each $S  \in K$. A hitting set for $K$ is minimal if and only if no proper subset of it is a hitting set for $K$. 
\end{definition}

Utilizing this approach, the MCSs are generated from the MUSs that are provided by the All-MIVCs algorithm~\cite{Ghassabani2017EfficientGO} and a minimal hitting set algorithm developed by Murakami et. al.~\cite{murakami2013efficient,gainer2017minimal}. \\

A Minimal Cut Set ($MinCutSet$) is a minimal collection of faults that lead to the violation of the safety property (or in other words, lead to the top level event in the fault tree). We define a minimal cut set consistently with much of the research in this field~\cite{0f356f05e72f43018211b36f97c8854a,historyFTA}

\begin{definition} :  A Minimal Cut Set can be defined as the set of faults in a system that cause the violation of the safety property. Furthermore, any strict subset of these faults will not cause violation of the safety property. 
\end{definition}







