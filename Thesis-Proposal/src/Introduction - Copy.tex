\chapter{Introduction}
\label{chap:intro}

%Safety critical systems are an increasingly important topic in our modern age. From nuclear power plants and airplanes to heart monitors and automobiles, critical systems are vitally important in our society. With a rise in technological advances also comes an increased need for reliable safety analysis methods and tools. Standardized methods of safety analysis have been used for many years and various tools and methods are currently applied to the topic in both industrial and academic settings. System safety analysis techniques are well-established and are a required activity in the development of safety-critical systems. Model-based systems engineering (MBSE) methods and tools based on formal methods now permit system-level requirements to be specified and analyzed early in the system development process~\cite{NFM2012:CoGaMiWhLaLu,CAV2015:BoCiGrMa}. While model-based development methods are widely used in the aerospace industry, they are only recently being applied to system safety analysis.  

%This paper describes a behavioral approach to safety analysis using an architecture description language called Architecture Analysis and Design Language. We describe a {\em Safety Annex} for the Architecture Analysis and Design Language (AADL)~\cite{FeilerModelBasedEngineering2012} that provides the ability to reason about faults and faulty component behaviors in AADL models. In the Safety Annex approach, we use formal assume-guarantee contracts to define the nominal behavior of system components. The nominal model is then verified using the Assume Guarantee Reasoning Environment (AGREE)~\cite{NFM2012:CoGaMiWhLaLu}. The Safety Annex  provides a way to weave faults into the nominal system model and analyze the behavior of the system in the presence of faults. The Safety Annex also provides a library of common fault node definitions that is customizable to the needs of system and safety engineers. Our approach adapts the work of Joshi et. al~\cite{Joshi05:Dasc} to the AADL modeling language, and provides a domain specific language for the kinds of analysis performed manually in previous work~\cite{Stewart17:IMBSA}. 


%The Safety Annex supports model checking and quantitative reasoning by attaching behavioral faults to components and then using the normal behavioral propagation and proof mechanisms built into the AGREE AADL annex. This allows users to reason about the evolution of faults over time, and produce counterexamples demonstrating how component faults lead to system failures. It can serve as the shared model to capture system design and safety-relevant information, and produce both qualitative and quantitative description of the causal relationship between faults/failures and system safety requirements.

System safety analysis is crucial in the development life cycle of critical systems to ensure adequate safety as well as demonstrate compliance with applicable standards. A prerequisite for any safety analysis is a thorough understanding of the system architecture and the behavior of its components; safety engineers use this understanding to explore the system behavior to ensure safe operation, assess the effect of failures on the overall safety objectives, and construct the accompanying safety analysis artifacts. Developing adequate understanding, especially for software components, is a difficult and time consuming endeavor. Given the increase in model-based development in critical systems~\cite{Joshi05:Dasc,CAV2015:BoCiGrMa,info17:HaLuHo,5979344,Gudemann:2010:FQQ:1909626.1909813}, leveraging the resultant models in the safety analysis process holds great promise in terms of analysis accuracy as well as efficiency.

In this report we describe the \emph{Safety Annex} for the system engineering language AADL (Architecture Analysis and Design Language), a SAE Standard modeling language for Model-Based Systems Engineering (MBSE)~\cite{AADL_Standard}. The Safety Annex allows an analyst to model the failure modes of components and then ``weave'' these failure modes together with the original models developed as part of MBSE. The safety analyst can then leverage the merged behavioral models to propagate %failures
errors through the system to investigate their effect on the safety requirements. %(implicit %failure
%error propagation). 
Determining how %faults
errors propagate through software components is currently a costly and time-consuming element of the safety analysis process. 
\begin{comment} 
The use of behavioral contracts to capture the implicit %fault
error propagation characteristics of software component is a significant benefit for safety analysts.  
In addition, the annex allows modeling of explicit %failure 
error propagation that is not captured through the behavioral models, for example, the effect of a single electrical failure on multiple software components or the effect hardware failure (e.g., an explosion) on multiple behaviorally unrelated components. 
\end{comment}
The use of behavioral contracts to capture the %implicit %fault
error propagation characteristics of software component without the need to add separate propagation specifications (\emph{implicit} error propagation) is a significant benefit for safety analysts.  
In addition, the annex allows modeling of %explicit %failure 
dependent faults that are not captured through the behavioral models (\emph{explicit} error propagation),
%error propagation that is not captured through the behavioral models, 
for example, the effect of a single electrical failure on multiple software components or the effect hardware failure (e.g., an explosion) on multiple behaviorally unrelated components. 
Furthermore, we will describe the tool support enabling engineers to investigate the correctness of the nominal system behavior (where no failures have occurred) as well as the system's resilience to component failures. We illustrate the work with a substantial example drawn from the civil aviation domain.

Our work can be viewed as a continuation of work conducted by Joshi et al.~where they explored model-based safety analysis techniques defined over Simulink/Stateflow~\cite{MathWorks} models~\cite{Joshi05:SafeComp,Joshi07:Hase,Joshi05:Dasc,DBLP:conf/cav/BozzanoCPJKPRT15}. Our current work extends and generalizes this work and provide new modeling and analysis capabilities not previously available.  For example, the Safety Annex allows modeling explicit %fault 
error propagation, supports compositional verification and exploration of the nominal system behavior as well as the system's behavior under failure conditions. Our work is also closely related to the existing safety analysis approaches, in particular, the AADL Error Annex (EMV2)~\cite{EMV2}, COMPASS~\cite{10.1007/978-3-642-04468-7_15}, and AltaRica~\cite{PROSVIRNOVA2013127,BieberERTS2018}. Our approach is significantly different from previous work in that unlike EVM2 we leverage the behavioral modeling for implicit %failure 
error propagation, we provide compositional analysis capabilities not available in COMPASS, and in addition, the Safety Annex  is fully integrated in a model-based development process and environment unlike a stand alone language such as AltaRica. 

%The main contributions of this project and the Safety Annex are:
The main contributions of the Safety Annex and this project are:
\begin{itemize}
\renewcommand{\labelitemi}{\textbullet}
		\item close integration of behavioral fault analysis into the {\em Architecture Analysis and Design Language} AADL, which allows close connection between system and safety analysis and system generation from the model,
		\item support for {\em behavioral specification of faults} and their {\em implicit propagation} (both symmetric and asymmetric) through behavioral relationships in the model, in contrast to existing AADL-based annexes (HiP-HOPS~\cite{CHEN201391}, EMV2~\cite{EMV2}) and other related toolsets (COMPASS~\cite{10.1007/978-3-642-04468-7_15}, Cecilia~\cite{bieber2004safety}, etc.),
		\item additional support to capture binding relationships between hardware and software and logical and physical communications, %and
		\item compute all minimal set of fault combinations that can cause violation of the safety properties to be compared to qualitative and quantitative objectives as part of the safety assessment process, and
		\item guidance on integration into a traditional safety analysis process.
\end{itemize}