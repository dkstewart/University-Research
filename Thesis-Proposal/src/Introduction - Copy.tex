\chapter{Introduction}
\label{chap:intro}

System safety analysis is crucial in the development life cycle of critical systems to ensure adequate safety as well as demonstrate compliance with applicable standards. A prerequisite for any safety analysis is a thorough understanding of the system architecture and the behavior of its components; safety engineers use this understanding to explore the system behavior to ensure safe operation, assess the effect of failures on the overall safety objectives, and construct the accompanying safety analysis artifacts. Developing adequate understanding, especially for software components, is a difficult and time consuming endeavor. Given the increase in model-based development in critical systems~\cite{Joshi05:Dasc,CAV2015:BoCiGrMa,info17:HaLuHo,5979344,Gudemann:2010:FQQ:1909626.1909813}, leveraging the resultant models in the safety analysis process holds great promise in terms of analysis accuracy as well as efficiency.

In this report we describe the \emph{Safety Annex} for the system engineering language AADL (Architecture Analysis and Design Language), a SAE Standard modeling language for Model-Based Systems Engineering (MBSE)~\cite{AADL_Standard}. The Safety Annex allows an analyst to model the failure modes of components and then ``weave'' these failure modes together with the original models developed as part of MBSE. The safety analyst can then leverage the merged behavioral models to propagate errors through the system to investigate their effect on the safety requirements. Determining how errors propagate through software components is currently a costly and time-consuming element of the safety analysis process. 

The use of behavioral contracts to capture the error propagation characteristics of software component without the need to add separate propagation specifications (\emph{implicit} error propagation) is a significant benefit for safety analysts.  
In addition, the annex allows modeling of dependent faults that are not captured through the behavioral models (\emph{explicit} error propagation), for example, the effect of a single electrical failure on multiple software components or the effect hardware failure (e.g., an explosion) on multiple behaviorally unrelated components. Furthermore, the tool enables engineers to investigate the correctness of the nominal system behavior (where no failures have occurred) as well as the system's resilience to component failures. 

\danielle{rework this paragraph}
In related work in model checking, the \emph{compositional} approach has been shown to provide greater scalability in large scale system models. There are numerous tools that can compute \emph{Minimal Cut Sets}, or the minimal cause of a \emph{top level event} (failure of a safety property), but none are able to do so compositionally. \danielle{What is the point of this paragraph??}

\section{Objectives and Significance}
In this project, we are specifically concerned with collecting proof information from the model checker and leveraging this in order to capture system safety information and artifacts. Previous research has provided a way to find all sets of minimal model elements necessary for the proof of a property and we attempt to use these to show vital information about possible modes of failure of a system and ways that active faults in the system can cause the violation of a safety property. \textit{The objective of this dissertation} is to use the elements required for the proof of a property in order to compositionally generate all minimal cut sets for a safety property. The minimal cut sets can then be used to generate commonly used safety analysis artifacts such as Fault Tree Analysis (FTA) or Failure Modes and Effect Analysis (FMEA) tables, but can also be used to calculate the probability of the violation of the safety property. 

While other available tools provide ways to generate minimal cut sets, what we propose is new in the following ways. Due to the implementation of the Safety Annex which utilizes behavioral mechanisms in AGREE (Assume Guarantee Reasoning Environment), the faults can be behaviorally propagated or explicitly propagated. Behavioral propagation allows for discovery of possibly unforeseen effects of failed components. The safety engineer is no longer required to determine how an error will propagate through a complex system, but can instead look at counterexamples and proof traces to see how the system is effected by failures of its components. On the other hand, this does not allow for all possible failure modes. Explicit propagation can describe dependencies between faults such as co-located components or other more specific hardware faults. Given these two possible modes of propagation, the fault model can provide a rich description of the system in question. What is more, other tools calculate minimal cut sets using \emph{monolithic} analysis. This flattens a hierarchical model and utilizes all elements of the model to provide proof of a property. It has been shown that a \emph{compositional} approach to proof is far more scalable due to viewing the model one layer at a time. 

The approach we propose takes advantage of these two aspects: the ability of combined behavioral and explicit error propagation within the model and compositional fault analysis. 

What we propose is a generic and efficient mechanism for extracting all minimal sets of model elements required for the proof of the safety property and then transforming them into faults that describe the basic events which cause the violation of the property of interest. Once these minimal cut sets are generated, probabilistic computation can proceed across them to provide the likelihood of this property violation. 

Compositional generation of minimal cut sets facilitates several useful system and safety engineering tasks. Specifically, it is useful to see how a component failure can effect the overall system behavior and which combinations of failures will violate safety properties. These minimal cut sets provide formal and human-understandable artifacts that can be used in the safety assessment process. Such information is valuable in analyzing safety critical systems and can be used for many purposes in the safety assessment process, such as:

\danielle{these may need to be changed... just throwing something out there.}
\begin{description}
\item [Fault Tree Analysis:] The traditional safety assessment process at the system level is based on ARP4754A and ARP4761 (\danielle{cite}). After the system is examined and potential functional failures are found and classified, the next step is the Preliminary System Safety Assessment (PSSA) which is updated throughout the system development process. A key element of the PSSA is a system level FTA. The FTA is a deductive failure analysis to determine the causes of a specific undesired event in a top-down fashion. For an FTA, a safety analyst begins with a failure condition and systematically examines the system design to determine all credible faults and failure combinations that could cause the undesired event. By using a model of the system, a fault model of possible failure modes, behavioral and explicit propagation, and automated generation of these failure combinations, the analyst is provided with an efficient means of understanding the system behavior and failures and can easily iterate through this process multiple times throughout system development.  
\item [Single Points of Failure:] Requirements can often be stated such that no single point of failure can violate the property. In this case, a scalable automated approach can provide insight that may be overlooked by an analyst. By using proof results, the behavior of the system can be examined when this single fault is active and system changes or development can be realized in order to mitigate these failures. 
\item [Probability of Violation:] The Safety Annex currently provides a way to calculate the combinations of probabilities associated with the faults and test to see if this combination is under a given threshold, but it is also valuable to know what the top level threshold of the system is given the faults and their probabilities. During the fault analysis, this can be calculated while the minimal cut sets are being collected which provides a scalable and efficient means of gathering this information.
\end{description}

\section{Use in Research and System Development}
The traditional safety assessment process at the system level is based on ARP4754A~\cite{SAE:ARP4754A} and ARP4761~\cite{SAE:ARP4761}. It starts with the System level Functional Hazard Assessment (SFHA) examining the functions of the system to identify potential functional failures and classifies the potential hazards associated with them. 

The next step is the Preliminary System Safety Assessment (PSSA), updated throughout the system development process. A key element of the PSSA is a system level Fault Tree Analysis (FTA).  The FTA is a deductive failure analysis to determine the causes of a specific undesired event in a top-down fashion. For an FTA, a safety analyst begins with a failure condition from the SFHA, and systematically examines the system design (e.g., signal flow diagrams provided by system engineers) to determine all credible faults and failure combinations that could cause the undesired event. 

The lack of precise models of the system architecture and its failure modes often forces safety analysts to devote significant effort to gathering architectural details about the system behavior from multiple sources. Furthermore, this investigation typically stops at system level, leaving software function details largely unexplored.

Typically equipped with the domain knowledge about the system, but not detailed knowledge of how the software applications are designed, practicing safety engineers find it a time consuming and involved process to acquire the knowledge about the behavior of the software applications hosted in a system and its impact on the overall system behavior.
Industry practitioners have come to realize the benefits of using models in the safety assessment process, and a revision of the ARP4761 to include Model Based Safety Analysis (MBSA) is under way.

We propose a model-based safety assessment process backed by formal methods to help safety engineers with early detection of the design issues.  This process uses a single unified model to support both system design and safety analysis. It is based on the following steps:

\begin{enumerate}
	\item System engineers capture the critical information in a shared AADL/AGREE model:  high-level hardware and software architecture, nominal behavior at the component level, and safety requirements at the system level.% (e.g., inhibit throttle movement during critical takeoff phase).
	\item System engineers use the backend model checker to check that the safety requirements are satisfied by the nominal design model. 
	\item Safety engineers use the Safety Annex to augment the nominal model with the component failure modes. % (e.g., processor failure, input signal corrupted).  
	In addition, safety engineers specify the fault hypothesis for the analysis which corresponds to how many simultaneous faults the system must be able to tolerate.
	\item Safety engineers use the backend model checker to analyze if the safety requirements and fault tolerance objectives are satisfied by the design in the presence of faults. % (e.g., if the system is resilient to a single failure). 
	If the design does not tolerate the specified number of faults (or probability threshold of fault occurrence), then the tool produces counterexamples leading to safety requirement violation in the presence of faults, %and also
	 as well as all minimal set of fault combinations that can cause the safety requirement to be violated.
	%produces fault trees showing smallest set of faults that may lead to the safety requirement being violated. 
	\item The safety engineers examine the results to assess the validity of the fault combinations and the fault tolerance level of the system design. If a design change is warranted, the model will be updated with the latest design change and the above process is repeated.
\end{enumerate}

\section{Intended Contributions}
We claim that compositionally generated minimal cut sets have potential system safety engineering uses in several phases of the development cycle. However, efficient and effective generation strategies must be proposed to achieve these benefits. The anticipated contributions of the work are therefore as follows:

\begin{itemize}
\item \emph{Behavioral fault propagation through the use of the Safety Annex built on the AGREE model checking capabilities:} This provides a model-based environment that contains the system model in AADL, the behavioral model in AGREE, and the fault model in the Safety Annex. When a fault is activated, a trace of the system is provided through the artifacts generated by a model checker (JKind) and the propagation does not have to be explicitly defined. 

\item \emph{Explicit fault propagation through the use of the Safety Annex built on the AGREE model checking capabilities:} Allowing for explicit propagation and fault dependencies provides a richer fault model and catches those cases that may be impossible to describe using only behavioral propagation, e.g. co-location of components or hardware failures. 

\item \emph{Collecting proof information to compositionally generate minimal cut sets:} The thesis will provide a formal way of generating all minimal cut sets through the transformation of all \emph{Minimal Inductive Validity Cores} (MIVCs). This will then be implemented in the Safety Annex tool. 

\item \emph{Calculating the probabilistic threshold of a safety property given the minimal cut sets:} The thesis will also provide the algorithm used in this calculation and a formal approach to this calculation.

\end{itemize}

\section{Evaluation}We plan to evaluate the approach on a large scale aerospace example and evaluate the overhead of the minimal cut set transformation compared to MIVC computation. \danielle{Evaluation with respect to other similar tools might prove to be difficult since these tools do not perform compositional MinCutSet generation and many only perform explicit \emph{or} behavioral propagation and not both. The baseline system model is also in a different modeling language; whereas we use AADL, related tools do not. One option is to compare EMV2 explicit propagation and FT generation to our approach, but I am not sure why this would be useful, especially since we have been careful to distinguish what we do with EMV2. All of that to say, I am unsure how to write this section.}

Therefore, we investigate the following research questions:
\begin{itemize}
\item \textbf{RQ 1:} Does the mix of behavioral and explicit error propagation provide more information on the state of the system model and the possible modes of failure than just one type of error propagation?

\item \textbf{RQ 2:} Are the algorithms used to transform MIVCs into MinCutSets scalable and efficient?

\item \textbf{RQ 3:} What is the time difference between calculating if a fault combination exceeds a given threshold and what the safety property threshold actually is?

\item \textbf{RQ 4:} Can these MinCutSets provide useful information about the system and its modes of failure that can be used in the certification process?

\end{itemize}

Upon completion of the proposed research, the transformation algorithms and the probabilsitic computations will be implemented in the Safety Annex. The implementation will be benchmarked and evaluated rigorously. The usefulness of the compositional MinCutSet idea will be shown by utilizing its applications into different projects.

\section{Chapters and Organization of the Proposal}
This proposal is organized in three chapters. Chapter 2 broadly discusses related work, the tools and modeling language used in this project, and some useful formal definitions. Chapter 3 describes the proposed approach and outlines the contributions of this dissertation. Lastly, the conclusion summarizes the approach of the project.


