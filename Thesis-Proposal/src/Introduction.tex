\chapter{Introduction}
\label{chap:intro}

System safety analysis is crucial in the development life cycle of critical systems to ensure adequate safety as well as demonstrate compliance with applicable standards. A prerequisite for any safety analysis is a thorough understanding of the system architecture and the behavior of its components; safety engineers use this understanding to explore the system behavior to ensure safe operation, assess the effect of failures on the overall safety objectives, and construct the accompanying safety analysis artifacts. Developing adequate understanding, especially for software components, is a difficult and time consuming endeavor. Given the increase in model-based development in critical systems~\cite{Joshi05:Dasc,CAV2015:BoCiGrMa,info17:HaLuHo,5979344,Gudemann:2010:FQQ:1909626.1909813}, leveraging the resultant models in the safety analysis process holds great promise in terms of analysis accuracy as well as efficiency.

\section{Objectives and Significance}
\begin{itemize}
\item The objective of this dissertation is...
\item The proposal description
\item How the information is used in the safety assessment process
\end{itemize}

\section{Use in Research and System Development}
The certification process in SA and MBSA.

\section{Intended Contributions}
Make itemized list with explanatory paragraphs.

\section{Evaluation}
Research questions to evaluate.

\section{Chapters and Organization of the Proposal}


