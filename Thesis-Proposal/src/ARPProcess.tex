\subsection{ARP Process}
\label{sec:arpprocess}

ARP4761, the Guidelines and Methods for Conducting Safety Assessment Process on Civil Airborne Systems and Equipment, provides guidance on applying development assurance at each hierarchical level throughout the development life cycle of highly-integrated/complex aircraft systems, and has been recognized by the Federal Aviation Administration (FAA) as an acceptable method to establish the assurance process~\cite{SAE:ARP4761}.

\begin{figure*}[h!]
	%\vspace{-0.956in}
	\centering
	\includegraphics[width=0.7\textwidth]{images/Safety_Assessment_Process.png}
	%\vspace{-0.4in}
	\caption{Using the Shared System/Safety Model in the ARP4754A Safety Assessment Process}
	\label{fig:proposed_safety_process}
\end{figure*}

The safety assessment process is part of the development life cycle, and is tightly coupled with the system development and verification processes. It is used to show compliance with certification requirements and for meeting a company's internal safety standards. The guidelines presented in ARP4761 include various industry accepted safety assessment practices. They are summarized here for convenience. 

\begin{itemize}
\item Functional Hazard Assessment (FHA): This process examines aircraft and system functions to identify potential functional failures and classifies the hazards associated with specific failure conditions. This is usually developed early in the development process and is updated throughout. 

\item Preliminary System Safety Assessment (PSSA): This will establish the system safety requirements and provide some indication that the system architecture can meet those safety requirements. This is also updated throughout the development process.

\item System Safety Assessment (SSA): This process collects, analyzes, and documents verification that the system, as implemented, meets the safety requirements established by the PSSA. 

\item Common Cause Analysis (CCA): The CCA establishes physical and functional separation, isolation, and independence requirements between systems and verifies that these requirements have been met.
\end{itemize}

As shown in Figure~\ref{fig:proposed_safety_process}, these processes occur during the development process and are continually updated throughout. The safety engineers then use the acquired understanding to conduct safety analysis, construct the safety analysis artifacts, and compare the analysis results with established safety objectives and safety-related requirements. 

In practice, prior to performing the safety assessment of a system, the safety engineers are often equipped with the domain knowledge about the system, but do not necessarily have detailed knowledge of how the software functions are designed. Acquiring the required knowledge about the behavior and implementation of each software function in a system can be time-consuming. Industry practitioners have come to realize the benefits and importance of using models to assist the safety assessment process (either by augmenting the existing system design model, or by building a separate safety model), and a revision of the ARP4761 to include model based safety analysis is under way. Capturing failure modes in models and generating safety analysis artifacts directly from models could greatly improve communication and synchronization between system designer and safety engineers, and provide the ability to more accurately analyze complex systems. 

A single unified model to conduct both system development and safety analysis can help reduce the gap in comprehending the system behavior and transferring the knowledge between the system designers and the safety analysts. This maintains a living model that captures the current state of the system design as it moves through the system development lifecycle.

A single unified model also allows all participants of the ARP- \\4754 process to be able to communicate and review the system design using a ``single source of truth.''

A model that supports both system design and safety analysis must describe both the system design information (e.g., system architecture, functional behavior) and safety-relevant information (e.g., failure modes, failure rates).  It must do this in a way that keeps the two types of information distinguishable, yet allows them to interact with each other.

Figure~\ref{fig:proposed_safety_process} presents our proposed use of this shared system design and safety analysis model in the context of the ARP4754A Safety Assessment Process Model (derived from Figure 7 of ARP4754A). The shared model is one of the system development artifacts from the ``Development of System Architecture'' and ``Allocation of System Requirements to Item'' activities in the System Development Process, which interacts with the PSSAs and SSAs activities in the Safety Assessment Process. This is seen as a box labeled ``Shared System Design and Safety model'' on the right column of the figure. The shared model can serve as an interface to capture the information from the system design and implementation that is relevant for the safety analysis.