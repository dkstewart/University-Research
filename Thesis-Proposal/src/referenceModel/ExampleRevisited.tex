\subsection{Putting it All Together}\label{sec:exampleRevisited}

In developing the reference model, we have explored a number of relationships
necessary to model our motivating scenario.  For clarity we illustrate many of
these constructs and relationships in Figure~\ref{fig:referenceModel}.  Using
the constructs defined in our reference model, we show an example of the input
model in Listing~\ref{listing:refModelAppliedToIdealProject}.

% \begin{figure}[htb]
%     \centering
%         \includegraphics[width=.47\textwidth]{img/referenceModel}
%     \caption{An abbreviated set of constructs and relationships in the model. 
%         For ease of understanding, an additional construct, Work, has been added
%         to illustrate the complex work relationship.}
%     \label{fig:referenceModel}
% \end{figure}

\begin{figure*}[htb]
    \centering
        \includegraphics[width=\textwidth]{img/referenceModel-extended-horizontalLegend}
    \caption{The constructs and relationships in the reference model. 
        The \textit{work} construct above represents the objects required by
        the agent to complete some work---that is, perform an activity on a work
        package.}
    \label{fig:referenceModel}
\end{figure*}


% We require an action selection criteria, support for reasoning over the
% completeness of an activity, learning models, and communication generation and
% handling.  These considerations here as the agent model will vary depending on
% how much we wish to model for the agent.
% At minumum, we require an action selection criteria needs additional
% information, such as action selection criteria.

\lstdefinelanguage{Spec}
{
    morekeywords={Agent, Knowledge, Product, Description, Artifact, Type,
        ValueGroup, Value, WorkPackage, Backlog, Context, Parent, Prerequisites,
        Activities, Roles, Properties, WorkPackageTypes, ArtifactType,
        ArtifactSet, Activity, Role, Function, OrganizationStructure,
        Pausable, Process, Complexity, Generates},
%
    morecomment=[l]{//}
}

\lstset{language=Spec, captionpos=b, multicols=2, basicstyle=\footnotesize,
frame=single, label={listing:refModelAppliedToIdealProject}} 
\begin{lstlisting}[float, caption={A subset of our scenario's input
expressed using the reference model.}] 
Agent:  aCleverTester 
  OrganizationStructure: ...
  Knowledge:
    testing, level 5
    java7, level 3
    jUnit4, level 4
    androidSDK, level 2
    androidStudio, level 3
    ...
    
Product:  someApp
  ArtifactSet:  myArtifactSet
    Artifact:  app
      Type:  application
      Knowledge:  //Skills
        java7
        androidSDK
        androidStudio
    Artifact:  testSuite
      Type:  test
      Knowledge:  //Skills
        java7
        androidSDK
        androidStudio
        jUnit4
  ValueGroup:  initialUI
    Value:  5
  WorkPackage:  initialView
    ValueGroup:  initialUI
    ArtifactSet:  myArtifactSet
    Complexity:  3
    Prerequisites: -
    Knowledge:  //Domain Knowledge
      -
    ...
    
Process: scrum
  Backlog:  productBacklog
  Backlog:  sprintBacklog
  ...
  Context:  develop
    Parent:  sprint
    Prerequisites:
      <Agent>.Role == teamMember
      notEmpty(sprintBacklog)
    Activities:
      implementTests
      ...
    Roles:  
      developer  
      tester
    Pausable:  true
    Properties:
      exitOn(sprintTimeout)
      pauseOn(scrumMeetingTime)
  Activity:  implementTests
    Role:  Tester
    ArtifactType:  test
    Knowledge:  //General Skills
      testing
    Prerequisites:
      exists(
        <WorkPackage>.ArtifactSet
        ArtifactType: application,
        Function: <WorkPackage>.name)
      complete(specifyTests)
    Generates:  functions
    ...
\end{lstlisting}

Let's walk through how a simulation of our scenario's model might progress. 
For ease of discussion, we assume that the simulation progresses when the
simulation clock ``ticks''.
When the simulation starts, each of the six agents assume a role for the scrum context: the four developers---including \texttt{aCleverTester}---assume the
developer role, the project manager assumes the scrum master role, and the
customer representative assumes the product owner role.
The scrum project is initialized---causing the product owner to populate the
product backlog---and the sprint kick-off meeting is held---populating the
sprint backlog.  Assume the team communicates with each other and decides that
when they enter the \texttt{Sprint Execution} context, they will start with the
\texttt{Develop} child-context.

As a team, they enter the sprint context and the developer agents enter the
\texttt{Develop} context per the context prerequisites.  At scheduled times, the
agents suspend existing actions and enter the \texttt{Scrum Meeting} context.

Since there is only one work package that is available for work (i.e.
not blocked by prerequisites), two agents work on it
(Figure~\ref{fig:dependencyNetwork}).  Assume these agents are
\texttt{aCleverTester}, who takes on the role of tester, and one of the
inexperienced developers takes on the developer role.  Each of these agents work
on the activities defined for their role
(Figure~\ref{fig:testActivityNetwork_annotated}), including both of them
executing the \texttt{Understand Work Package Specification} activity.
Let's fast forward a bit.  Upon completion of an activity,
\texttt{aCleverTester} selects an activity to perform by looking at the
available activities for its role and determines which one it can perform, based
on the prerequisites.  \texttt{aCleverTester} notices that all of the
prerequisites have been satisfied for the \texttt{Implement Tests} activity, so
it performs the \texttt{Implement Tests} activity for the \texttt{initialView}
work package.  As part of the simulation, \texttt{aCleverTester} determines its
outputs for the current tick based upon its knowledge levels compared to the
required knowledge levels, the rules in the activity (e.g. find defect/rework,
generate latent defect/rework), the work package's properties, and other
properties of the agent.  If we assume the simulation progresses in discrete
time intervals (ticks), then the output of the agent at the end of a tick could
be nothing (i.e. continue to perform the activity for the work package),  the
newly completed function for the artifact (in this case the \texttt{TestSuite}),
or other required objects---messages sent to other agents to improve
knowledge, generate rework items, and discovered defects in any of the previously
completed function---to complete the work.  Generated defects and, potentially,
the generated rework are added to the sprint backlog so other agents may work on it.

While working on the project, each agent tracks the amount of time it spent on
this project, in terms of ticks, meanwhile gaining development skill and domain
knowledge.  Upon function completion, the work package is marked complete.  The
process dictates when to award credit for value group completion and in our
scenario, this is done either during the scrum meeting or upon
exiting the \texttt{Sprint} context.

Because no other work packages are available and all roles for the
available work package have been assigned, the other developer agents idle.
This time is not counted toward their time on the project.  However, when there
is work available for the idling agents and the agents are available, the agents
perform work according to their own activity ordering.  If the modeler wishes to
model preference for particular activity orderings---i.e.
strategies---our reference model would support this.

In order to simulate requirements changes, we include a special type of agent
that generates work package changes (including addition and removal) according
to some specified criteria and arrival rate.  Change requests, modeled as a type
of communication, may include information about a change to an existing work
package, changes to the dependency structure, and/or a new work package,
depending on the change.  It is up to the specific implementation to determine
if the generator agent holds onto change requests until queried, if the agent
deposits them into a backlog that other agents can pull from, or if the agent
notifies others of changes without solicitation.  The product owner enacts these
changes to the existing work packages in the product backlog if they have not
been moved to the sprint backlog, otherwise, it schedules rework to change
completed or in-progress work packages that result in the loss of product
functionality.  Because we are using agents, it is possible to model behaviors
in which the product owner agent ignores the prescribed rules and pushes the
changes on the team mid-sprint, however defining agent behavior will be
addressed when we discuss developing the simulation framework.

In a scrum process, the sprint retrospective allows the team to improve the
process based on the team's observation of previous sprints.  While this allows
for process enhancement in real projects, this is beyond the scope of both our
model and planned evaluation environment.  However, this meeting does consume
time and can have non-tangible effects on the team, thus we still include it in
the model and allow the particular agent implementation to model the
non-tangible effects, such as morale improvements.

The simulation progresses in much the same way we have described here, and,
according to some pre-specified condition, the simulation terminates.  The
simulator then computes and returns the project duration, effort, disaggregated
effort, product functionality, and defect density.  

Computing the duration and effort metrics is straight-forward.  As mentioned,
our functionality metric increases when all work packages within a value group
are completed.  As work packages change, we alter the amount of functionality
provided by the product, as necessary.
There are a number of ways to quantify scope decreases.  One such method is to
reduce it proportional to the loss of complexity within the value group.  We
allow this to be defined in the input model.
Defect density is computed by summing up the number of unfixed defects---those
defects still associated with artifact functions at the end of the
simulation---and dividing it by the scope.
