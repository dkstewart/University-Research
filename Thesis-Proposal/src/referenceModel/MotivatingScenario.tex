\subsection{A Motivating Scenario}\label{sec:scenario}
Before constructing our initial reference model, let's discuss a motivating
scenario, based on the author's experience, of a process and target project
environment that we wish to model and ultimately simulate.

A team is going to start a mobile app development project and desires to
evaluate a set of processes to determine which is going to best fit their needs.
The project is time constrained.  The product's requirements decomposition is
partially depicted in Figure~\ref{fig:dependencyNetwork}.  In this illustration,
smaller boxes represent work packages---a specification of a small piece of
function---and the larger boxes represent non-overlapping groupings of those
work packages that will, together, deliver value to the customer upon
completion---e.g. stories for most agile teams.  In our discussion, work
packages do not include process activities.  
%Assume that the team is required to produce a suite of 
%automated regression tests to support long-term maintenance.

\begin{figure}[htb]
    \centering
        \includegraphics[width=0.65\textwidth]{img/workPackageDependencies}
    \caption{The abbreviated work package dependency network (arrows point to
        successors).}
    \label{fig:dependencyNetwork}
\end{figure}


The team is composed of six people: an inexperienced project manager, a
stakeholder that represents the customer, one experienced developer with an
affinity for test driven development (TDD), one experienced developer with
strong testing skills, and two inexperienced developers.

Having seen scrum~\cite{rubin_essential_2012} work for other teams, the project
manager wishes to model and evaluate this process.  To reduce
resistance to process adoption, he plans to allow individuals to continue
to use their preferred personal processes and roles with a common definition of
``done''.  Thus, one developer would act as a tester for others, one would
follow TDD, and the remaining developers would determine their own activity
sequencing, often using the tester to verify their code.

