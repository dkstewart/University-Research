\chapter{Conclusion}
System safety analysis is cruicial in the development of critical systems and the generation of accurate and scalable results is invaluable to the assessment process. Having multiple ways to capture complex dependencies between faults and the behavior of the system in the presence of these faults is important throughout the entire process. The artifacts generated from such analyses that are used in the certification process of such systems must be generated in a scalable way and provide accurate and important information. This project has developed and implemented the Safety Annex for AADL which provides a way to capture complex relationships between faults in a model and analyze their effects behaviorally through either compositional or monolithic analysis. 

Furthermore, we propose the compositional generation of minimal cut sets to be used in the development of various artifacts used in system certification, such as FTA, FMEA, and single point of failure examinations. This generation is done through the collection of proof elements called MIVCs and their transformation. 

Lastly, we propose to use the minimal cut sets and resulting fault trees generated through the transformation algorithms to calculate the probability of the top level event, or violation of the safety property.   

